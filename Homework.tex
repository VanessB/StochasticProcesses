\documentclass[12pt]{article}
\usepackage[a5paper, left=1cm, right=1cm, top=1cm, bottom=1cm]{geometry}

% Ссылки.
\usepackage[unicode=true]{hyperref}

% Улучшенные сноски.
\usepackage{footmisc}

% Продвинутые формулы.
\usepackage{amsmath}

% Продвинутые математические символы.
\usepackage{amssymb}

% Кастомизируемые теоремы.
\usepackage{amsthm}
%\usepackage{thmtools}

% Русский язык.
\usepackage{cmap}
\usepackage[T2A]{fontenc}
\usepackage[utf8]{inputenc}
\usepackage[russian]{babel}

% Кастомизируемые хэдеры и футеры.
%\usepackage{fancyhdr}

% Табуляция перед первым параграфом.
%\usepackage{indentfirst}

% Нижнее подчёркивание с переносами.
\usepackage[normalem]{ulem}

% Графики gnuplot.
\usepackage[shell, subfolder, cleanup]{gnuplottex}

% Работа с плавающими объектами.
\usepackage[section]{placeins}

% Обтекаемые изображения
\usepackage{wrapfig}

% Ячейки на несколько строк.
\usepackage{multirow}
\usepackage{makecell}

% Таблица с регулируемой шириной столбцов и работающими сносками.
\usepackage{tabularx}

% Альтернативные таблицы.
%\usepackage{tabularray}

% Вращение.
\usepackage{rotating}
\usepackage{pdflscape}

% Элементы на следующей странице.
\usepackage{afterpage}

% Задачи.
\usepackage[lastexercise]{exercise}

% Сброс счётчика.
\usepackage{chngcntr}

% Графика TikZ
\usepackage{tikz}
%\usepackage{tikz-qtree}
%\usetikzlibrary{calc}
\usetikzlibrary{trees,calc,arrows.meta,positioning,decorations.pathreplacing,bending,matrix}
%\usetikzlibrary{trees}

% Таблицы Юнга.
\usepackage{ytableau}

% Enum и item на одной строке.
\usepackage[inline]{enumitem}
\setlist[1]{itemsep=0pt} % Величина разрыва между \item


    %%%%%%%%%

    % КОМАНДЫ

    %%%%%%%%%


% Математические символы и прочие дефайны.
%% Математические символы и прочие дефайны.

\def\defarr{\overset{\triangle}{\Longleftrightarrow}} % <<По определению>>
\def\defeq{\overset{\triangle}{=}}                    % <<По определению равно>>
\def\symdiff{\,\triangle\,}                           % <<Симметрическая разность>>
\def\connected{\leftrightsquigarrow}                  % Связность в графах.

\def\boolfun{\mathcal{B}} % Булевы функции.

% Математическое операторы.
\DeclareMathOperator{\diam}{\textnormal{diam}}
\DeclareMathOperator{\rad}{\textnormal{rad}}
\DeclareMathOperator*{\argmin}{\arg\min}
\DeclareMathOperator*{\argmax}{\arg\max}
\DeclareMathOperator*{\dom}{\textnormal{dom}}
\DeclareMathOperator*{\range}{\textnormal{range}}
\DeclareMathOperator*{\closure}{\textnormal{cl}}

% Математические множества.
\def\naturals{\mathbb{N}}
\def\integers{\mathbb{Z}}
\def\rationals{\mathbb{Q}}
\def\reals{\mathbb{R}}
\def\complexes{\mathbb{C}}


% Настройки пакета с упражнениями.
%% Настройки пакета xsim
\xsimsetup{path=xsim/}
%\loadxsimstyle{layouts}
\loadxsimstyle{custom}


% Перевод.
\DeclareExerciseTranslations{exercise}{Russian = Задача}
\DeclareExerciseTranslations{solution}{Russian = Решение задачи}

% Оружение.
\DeclareExerciseType{exercise}{
    exercise-env = exercise ,
    solution-env = solution ,
    exercise-name = \XSIMtranslate{exercise} ,
    exercises-name = \XSIMtranslate{exercises} ,
    solution-name = \XSIMtranslate{solution} ,
    solutions-name = \XSIMtranslate{solutions} ,
    exercise-template = runin ,
    solution-template = runin-sol ,
    exercise-heading = \normalsize \textbf ,
    solution-heading = \normalsize \textbf ,
    counter = TheoremCounter
}

\xsimsetup{solution/print=true}


% Картинки.
% Счётчики таблиц и фигур.
\counterwithin{table}{section}
\counterwithin{figure}{section}



% Специальные обозначения бкув.
%% Специальные обозначения букв.

%\def\N{\mathbb{N}}
%\def\Z{\mathbb{Z}}
%\def\Q{\mathbb{Q}}
%\def\R{\mathbb{R}}
%\def\f{\mathcal{F}}
%\def\l{\mathcal{L}}
%\def\t{\mathcal{T}}
%\def\I{\mathbb{I}}


% Теория вероятностей.
%% Теория вероятностей.
\def\proba{\mathbb{P}}
\def\setfamily{\mathcal{F}}
\def\borel{\mathcal{B}}
\def\trajectories{\mathcal{X}}
\def\indicator{\mathbb{I}}
\def\lebesgue{\mathbb{L}}
\def\iid{\textnormal{н.о.р.с.в.}}

% Моменты.
\DeclareMathOperator{\expect}{\mathbb{E}}
\DeclareMathOperator{\dispersion}{\mathbb{D}}
\newcommand{\covariance}[2]{\textnormal{cov}\left(#1, #2\right)}
\newcommand{\rvcenter}[1]{\mathring{#1}}

% Распределения.
\def\bernoulli{\textnormal{Be}}
\def\binomial{\textnormal{Bi}}
\def\poisson{\textnormal{Po}}
\def\uniform{\textnormal{U}}
\def\expdistr{\textnormal{Exp}}
\def\normal{\mathcal{N}}

% Сходимости.
\DeclareMathOperator*{\limmeansq}{\textnormal{l.i.m.}}
\newcommand{\converges}[1]{\overset{#1}{\longrightarrow}}
\def\convmeansq{\converges{\textnormal{с.к.}}}
\def\convalmost{\converges{\textnormal{п.н.}}}
\def\convproba{\converges{\proba}}
\def\convdistr{\converges{d}}
\def\convnorm{\converges{\| \cdot \|}}

\def\almosteq{\overset{\textnormal{п.н.}}{=}}
\def\distreq{\overset{d}{=}}


% Разделы без номеров.
% Глава без номера.
\newcommand{\silentchapter}[1]{
    \chapter*{#1}
    \markboth{\MakeUppercase{#1}}{#1}
    \addcontentsline{toc}{chapter}{#1}
}

% Секция без номера.
\newcommand{\silentsection}[1]{
    \section*{#1}
    \markboth{\MakeUppercase{#1}}{#1}
    \addcontentsline{toc}{section}{#1}
}

% Подсекция без номера.
\newcommand{\silentsubsection}[1]{
    \subsection*{#1}
    \markboth{\MakeUppercase{#1}}{#1}
    \addcontentsline{toc}{subsection}{#1}
}



% Настройки пакета asmthm.
%% Настройки пакета asmthm.

% Счётчик теорем и прочего.
\newcounter{TheoremCounter}
\counterwithin{TheoremCounter}{section}
%\counterwithin*{TheoremCounter}{subsection}

% Теоремы, определения, замечания и так далее.
\newtheorem{theorem}[TheoremCounter]{Теорема}
\newtheorem{lemma}[TheoremCounter]{Лемма}
\newtheorem{corollary}[TheoremCounter]{Следствие}
\newtheorem{definition}[TheoremCounter]{Определение}
\newtheorem{remark}[TheoremCounter]{Замечание}
\newtheorem{statement}[TheoremCounter]{Утверждение}
\newtheorem{problem}[TheoremCounter]{Задача}
\newtheorem{example}[TheoremCounter]{Пример}



% Правила вывода.
\newcommand{\typerule}[2]{%
    \begin{tabular}{c}
    $#1$ \\
    \hline
    %\midrule
    $#2$
    \end{tabular}%
}


% Inline item.
\makeatletter
\newcommand{\inlineitem}[1][]
{%
    \ifnum\enit@type=\tw@
        {\descriptionlabel{#1}}
        \hspace{\labelsep}%
    \else
        \ifnum\enit@type=\z@
        \refstepcounter{\@listctr}\fi
        \quad\@itemlabel\hspace{\labelsep}%
\fi}
\makeatother


% Выделение в определении
%\newcommand{\defemph}[1]{\textbf{\textit{#1}}}
\DeclareTextFontCommand{\defemph}{\bfseries\em}

% Нумерация русскими буквами.
\renewcommand{\alph}[1]{\asbuk{#1}}

% Пространство между плавающими объектами.
\setlength{\floatsep}{20pt}


% Счётчик для домашних задач.
\newcounter{HomeExercise}
\counterwithin{HomeExercise}{section}

\setlength{\ExerciseSkipBefore}{0.5\baselineskip}
\setlength{\ExerciseSkipAfter}{0.5\baselineskip}
\setlength{\AnswerSkipBefore}{0.0\baselineskip}
\setlength{\AnswerSkipAfter}{0.5\baselineskip}





\title{Случайные процессы: домашние задания}
\author{}
\date{2023}

\pagenumbering{gobble}

\begin{document}

\numberwithin{equation}{section}

\maketitle

\newpage



\silentsection{Домашнее задание на первую неделю}
\stepcounter{section}

\begin{Exercise}[counter=HomeExercise, title={(Задача из канонического задания)}]
    \noindent
    Пусть случайный процесс $ X(\omega, t) = \omega t $, $ t \in [0; 1] $,
    определен на вероятностном пространстве $ (\Omega, \setfamily, \proba) $,
    где $ \Omega = \{1, 2, 3\} $, $ \setfamily $~--- множество всех подмножеств множества $ \Omega $,
    а мера $ \proba $ такова, что $ \proba(\{1\}) = \proba(\{2\}) = \proba(\{3\}) = 1/3 $.
    Построить вторичное (выборочное) вероятностное пространство процесса.
\end{Exercise}


\begin{Exercise}[counter=HomeExercise]
    \noindent
    Случайный процесс $ X $ задан формулой $ X_t = t \cdot \eta $, где $ \eta \sim \uniform_{(0; 1)} $, $ t \in (0; 1) $.
    Найдите $ n $-мерные функции распределения этого процесса.
\end{Exercise}


\begin{Exercise}[counter=HomeExercise]
    \noindent
    Найдите математическое ожидание, дисперсию и корреляционную функцию процесса из предыдущей задачи.
\end{Exercise}


\begin{Exercise}[counter=HomeExercise]
    \noindent
    Пусть дана случайная величина $ \eta \sim \uniform_{[0;1]} $.
    Определим случайный процесс $ X_t = \indicator_{(-\infty; \eta]}(t) $.
    Найдите вероятность, что скачок с единицы до нуля произойдёт на интервале $ [t_0; t_0 + \Delta t] $,
    если достоверно известно, что на $ [0; t_0] $ скачка не было
    (параметр $ \Delta t $ задан и строго меньше $ 1 - t_0 $).
\end{Exercise}


\begin{Exercise}[counter=HomeExercise]
    \noindent
    Пусть $ \xi $ и $ \eta $~--- независимые случайные величины с функциями распределения $ F_\xi(x) $ и $ F_\eta(y) $.
    Пусть $ X $~--- случайный процесс, определённый формулой $ X_t = \xi \cdot t + \eta $.
    Найдите семейство конечномерных распределений процесса.
\end{Exercise}


\begin{Exercise}[counter=HomeExercise]
    \noindent
    Пусть $ X_1 $, $ X_2 $~--- два независимых случайных процесса с корреляционными функциями $ R_{X_1}(t, s) $ и $ R_{X_2}(t, s) $
    и функциями среднего $ m_{X_1}(t) $ и $ m_{X_2}(t) $.
    Найдите корреляционную функцию процесса $ Y = X_1 \cdot X_2 $.
\end{Exercise}



\silentsection{Домашнее задание на вторую неделю}
\stepcounter{section}

\begin{Exercise}[counter=HomeExercise, title={(Задача из канонического задания)}]
    \noindent
    Поток сделок в фирме моделируется с помощью пуассоновского процесса $ K $ c интенсивностью $ \lambda = 100 \; \text{сделок}/\text{час} $.
    Каждая сделка приносит доход $ V_i \sim \uniform_{[a;b]} $, $ a = 10 $, $ b = 100 $ условных единиц денег.
    Считая, что $ K $, $ \{ V_i \}_{i \in \naturals} $~--- независимые в совокупности случайные величины,
    найдите математическое ожидание, дисперсию и характеристическую функцию выручки за время $ t $.
    Докажите, что она имеет асимптотически нормальное распределение.
\end{Exercise}


\begin{Exercise}[counter=HomeExercise, title={(Задача из канонического задания)}]
    \noindent
    Случайный процесс $ X $ представляет собой сумму $ n $ независимых пуассоновских процессов
    с интенсивностями $ \{ \lambda_i \}_{i \in \{1, \ldots, n\}} $.
    Определить тип и параметры процесса $ X $.
\end{Exercise}


\begin{Exercise}[counter=HomeExercise, title={(Задача из канонического задания)}]
    \noindent
    Пусть $ K $~--- пуассоновский случайный процесс интенсивности $ \lambda $,
    а X~--- случайный процесс, полученный в результате удаления из $ K $ всех событий,
    очередной номер которых не кратен $ s $.
    Определить тип и параметры распределения интервала между соседними событиями в случайном процессе $ X $.
\end{Exercise}


\begin{Exercise}[counter=HomeExercise]
    \noindent
    Пусть $ \{ \xi_k \}_{k \in \naturals} $ все независимы в совокупности и имеют одинаковое распределение $ U_{[3;5]} $.
    Покажите, что процесс восстановления, построенный по этим случайным величинам
    (т. е. процесс вида $ X_t = \sup\{n \mid \xi_1 + \ldots + \xi_n \leqslant t\} $)
    не является процессом с независимыми приращениями.
\end{Exercise}


\begin{Exercise}[counter=HomeExercise]
    \noindent
    Пусть $ K $~--- пуассоновский процесс с интенсивностью $ \lambda > 0 $.
    Какие из следующих процессов имеют независимые приращения?
    \begin{enumerate}
        \item $ X_t = K_t - K_0 $, $ t \geqslant 0 $.
        \item $ X_t = K_t \mod 2 $, $ t \geqslant 0 $.
        \item $ X_t = K_{t^2 - t + 1} $, $ t \geqslant 0 $.
        \item $ X_t = K_t^2 $, $ t \geqslant 0 $.
    \end{enumerate}
\end{Exercise}


\begin{Exercise}[counter=HomeExercise]
    \noindent
    Пусть $ K $~--- пуассоновский процесс с интенсивностью $ \lambda > 0 $.
    Найдите вероятность, что в момент времени $ t $ число $ K_t $ чётно.
\end{Exercise}


\begin{Exercise}[counter=HomeExercise]
    \noindent
    Найдите предел при $ t \to +\infty $ (почти наверное) величины $ K_t / t $,
    где $ K $~--- пуассоновский процесс интенсивности $ \lambda \geqslant 0 $.
\end{Exercise}


\begin{Exercise}[counter=HomeExercise, title={(Практическое задание)}]
    \noindent
    Вас приняли на должность системного администратора в известную IT-компанию <<Рога и Копыта>>.
    Одной из ваших задач является стресс-тестирование сетевой инфраструктуры компании.
    Для моделирования потока данных от пользователей вы решили использовать сложный пуассоновский процесс с интенсивностью $ \lambda $.
    Размер $ V_i $ каждого приходящего пакета распределён логнормально:
    \[
        \rho_{V}(x) = \indicator_{[0;+\infty)}(x) \cdot \frac{\exp\left(-\frac{(\ln x - \mu)^2}{2 \sigma^2}\right)}{x \cdot \sigma \sqrt{2 \pi}}
    \]
    Пользуясь результатами, полученными на семинаре, найдите функцию среднего и корреляционную функцию процесса
    (матожидание и дисперсию $ V $ можно взять из справочника).
    Найти вероятностное распределение времени между отправкой $ n $-ого и $ (n+m) $-ого пакета.

    Пусть связь с одним из серверов осуществляется по $ N $ независимым каналам,
    на каждом из которых поток пакетов моделируется согласно процессу выше.
    Найдите вид и параметры процесса, соответствующего суммарному потоку данных на сервер.

    По аналогии с кодом в репозитории курса напишите функцию,
    которая по параметрам процесса ($ \lambda $, $ \mu $, $ \sigma $) моделирует заданное число реализаций.
    Постройте графики реализаций для некоторого набора параметров.

    Зафиксируем $ \sigma^2 = \mu $.
    Взяв в качестве максимальной пропускной способности $ Q_{\text{max}} = \lambda \cdot e^{3 \mu} $,
    путём компьютерного моделирования оценить частоту выхода канала из строя при работе в течение времени $ 100 / \lambda $.
\end{Exercise}


\end{document}
