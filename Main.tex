\documentclass[12pt]{article}
%\usepackage[14pt]{extsizes}
%\usepackage[a5paper, textwidth=12cm, textheight=17cm, headheight=14pt]{geometry}
\usepackage[a4paper, left=2cm, right=2cm, top=2cm, bottom=2cm]{geometry}

% Ссылки.
\usepackage[unicode=true]{hyperref}

% Улучшенные сноски.
\usepackage{footmisc}

% Продвинутые формулы.
\usepackage{amsmath}

% Продвинутые математические символы.
\usepackage{amssymb}

% Кастомизируемые теоремы.
\usepackage{amsthm}
%\usepackage{thmtools}

% Упражнения.
\usepackage[use-files, blank, clear-aux]{xsim}

% Русский язык.
\usepackage{cmap}
\usepackage[T2A]{fontenc}
\usepackage[utf8]{inputenc}
\usepackage[russian]{babel}

% Нижнее подчёркивание с переносами.
\usepackage[normalem]{ulem}

% Графики gnuplot.
\usepackage[shell, subfolder, cleanup]{gnuplottex}

% Работа с плавающими объектами.
\usepackage[section]{placeins}

% Обтекаемые изображения
\usepackage{wrapfig}

% Ячейки на несколько строк.
\usepackage{multirow}
\usepackage{makecell}

% Таблица с регулируемой шириной столбцов и работающими сносками.
\usepackage{tabularx}

% Альтернативные таблицы.
%\usepackage{tabularray}

% Вращение.
\usepackage{rotating}
\usepackage{pdflscape}

% Элементы на следующей странице.
\usepackage{afterpage}

% Сброс счётчика.
\usepackage{chngcntr}

% Графика TikZ
\usepackage{tikz}
\usetikzlibrary{trees,calc,arrows.meta,positioning,decorations.pathreplacing,bending,matrix}

% Enum и item на одной строке.
\usepackage[inline]{enumitem}
\setlist[1]{itemsep=0pt} % Величина разрыва между \item


    %%%%%%%%%

    % КОМАНДЫ

    %%%%%%%%%


% Математические символы и прочие дефайны.
%% Математические символы и прочие дефайны.

\def\defarr{\overset{\triangle}{\Longleftrightarrow}} % <<По определению>>
\def\defeq{\overset{\triangle}{=}}                    % <<По определению равно>>
\def\symdiff{\,\triangle\,}                           % <<Симметрическая разность>>
\def\connected{\leftrightsquigarrow}                  % Связность в графах.

\def\boolfun{\mathcal{B}} % Булевы функции.

% Математическое операторы.
\DeclareMathOperator{\diam}{\textnormal{diam}}
\DeclareMathOperator{\rad}{\textnormal{rad}}
\DeclareMathOperator*{\argmin}{\arg\min}
\DeclareMathOperator*{\argmax}{\arg\max}
\DeclareMathOperator*{\dom}{\textnormal{dom}}
\DeclareMathOperator*{\range}{\textnormal{range}}
\DeclareMathOperator*{\closure}{\textnormal{cl}}

% Математические множества.
\def\naturals{\mathbb{N}}
\def\integers{\mathbb{Z}}
\def\rationals{\mathbb{Q}}
\def\reals{\mathbb{R}}
\def\complexes{\mathbb{C}}

% Функции
\def\blankarg{\, \cdot \,}


% Картинки.
% Счётчики таблиц и фигур.
\counterwithin{table}{section}
\counterwithin{figure}{section}



% Специальные обозначения бкув.
%% Специальные обозначения букв.

%\def\N{\mathbb{N}}
%\def\Z{\mathbb{Z}}
%\def\Q{\mathbb{Q}}
%\def\R{\mathbb{R}}
%\def\f{\mathcal{F}}
%\def\l{\mathcal{L}}
%\def\t{\mathcal{T}}
%\def\I{\mathbb{I}}


% Теория вероятностей.
%% Теория вероятностей.
\def\proba{\mathbb{P}}
\def\setfamily{\mathcal{F}}
\def\borel{\mathcal{B}}
\def\trajectories{\mathcal{X}}
\def\indicator{\mathbb{I}}

% Моменты.
\DeclareMathOperator{\expect}{\mathbb{E}}
\DeclareMathOperator{\dispersion}{\mathbb{D}}
\newcommand{\covariance}[2]{\textnormal{cov}(#1, #2)}
\newcommand{\rvcenter}[1]{\mathring{#1}}

% Распределения.
\def\bernoulli{\textnormal{Be}}
\def\binomial{\textnormal{Bi}}
\def\poisson{\textnormal{Po}}
\def\uniform{\textnormal{U}}
\def\expdistr{\textnormal{Exp}}
\def\normal{\mathcal{N}}

% Сходимости.
\newcommand{\converges}[1]{\overset{#1}{\longrightarrow}}
\def\convmeansq{\converges{\textnormal{с.к.}}}
\def\convalmost{\converges{\textnormal{п.н.}}}
\def\convproba{\converges{\proba}}
\def\convdistr{\converges{d}}

\def\almosteq{\overset{\textnormal{п.н.}}{=}}
\def\distreq{\overset{d}{=}}


% Разделы без номеров.
% Глава без номера.
\newcommand{\silentchapter}[1]{
    \chapter*{#1}
    \markboth{\MakeUppercase{#1}}{#1}
    \addcontentsline{toc}{chapter}{#1}
}

% Секция без номера.
\newcommand{\silentsection}[1]{
    \section*{#1}
    \markboth{\MakeUppercase{#1}}{#1}
    \addcontentsline{toc}{section}{#1}
}

% Подсекция без номера.
\newcommand{\silentsubsection}[1]{
    \subsection*{#1}
    \markboth{\MakeUppercase{#1}}{#1}
    \addcontentsline{toc}{subsection}{#1}
}



% Настройки пакета asmthm.
%% Настройки пакета asmthm.

% Счётчик теорем и прочего.
\newcounter{TheoremCounter}
\counterwithin{TheoremCounter}{section}
%\counterwithin*{TheoremCounter}{subsection}

% Теоремы, определения, замечания и так далее.
\newtheorem{theorem}[TheoremCounter]{Теорема}
\newtheorem{lemma}[TheoremCounter]{Лемма}
\newtheorem{corollary}[TheoremCounter]{Следствие}
\newtheorem{definition}[TheoremCounter]{Определение}
\newtheorem{remark}[TheoremCounter]{Замечание}
\newtheorem{statement}[TheoremCounter]{Утверждение}
\newtheorem{problem}[TheoremCounter]{Задача}
\newtheorem{example}[TheoremCounter]{Пример}


% Настройки пакета с упражнениями.
%% Настройки пакета xsim
\xsimsetup{path=xsim/}
%\loadxsimstyle{layouts}
\loadxsimstyle{custom}


% Перевод.
\DeclareExerciseTranslations{exercise}{Russian = Задача}
\DeclareExerciseTranslations{solution}{Russian = Решение задачи}

% Оружение.
\DeclareExerciseType{exercise}{
    exercise-env = exercise ,
    solution-env = solution ,
    exercise-name = \XSIMtranslate{exercise} ,
    exercises-name = \XSIMtranslate{exercises} ,
    solution-name = \XSIMtranslate{solution} ,
    solutions-name = \XSIMtranslate{solutions} ,
    exercise-template = runin ,
    solution-template = runin-sol ,
    exercise-heading = \normalsize \textbf ,
    solution-heading = \normalsize \textbf ,
    counter = TheoremCounter
}

\xsimsetup{solution/print=true}





% Inline item.
\makeatletter
\newcommand{\inlineitem}[1][]
{%
    \ifnum\enit@type=\tw@
        {\descriptionlabel{#1}}
        \hspace{\labelsep}%
    \else
        \ifnum\enit@type=\z@
        \refstepcounter{\@listctr}\fi
        \quad\@itemlabel\hspace{\labelsep}%
\fi}
\makeatother


% Выделение в определении
\DeclareTextFontCommand{\defemph}{\bfseries\em}

% Нумерация русскими буквами.
\renewcommand{\alph}[1]{\asbuk{#1}}

% Пространство между плавающими объектами.
\setlength{\floatsep}{20pt}




\title{Случайные процессы: семинары}
\author{Бутаков~И.\,Д.}
\date{2023}

%\pagenumbering{gobble}

\begin{document}

\numberwithin{equation}{section}

\maketitle

\tableofcontents

\silentsection{Предисловие} \label{sec:intro}

Перед вами сборник всех семинаров по случайным процессам за авторством Бутакова\,И.\,Д.
Автор выражает благодарность Останину Павлу Антоновичу и Широбокову Максиму Геннадьевичу за предоставленные материалы.

\silentsubsection{Используемые обозначения}

\begin{center}
    \begin{tabularx}{\textwidth}{cl}
        $ \defarr $                      & <<\ldots по определению тогда и только тогда, когда \ldots>> \\
        $ \defeq $                       & <<\ldots по определению равно \ldots>> \\
        \rule{0pt}{16pt}%
        $ (\Omega, \setfamily, \proba) $ & \makecell[l]{вероятностное пространство ($ \Omega $~--- множество исходов, $ \setfamily $~--- $ \sigma $-алгебра, \\ $ \proba $~--- вероятностная мера).} \\
        $ \borel(A) $, $ \borel_A $      & \makecell[l]{Борелевская $ \sigma $-алгебра, определённая на множестве $ A $ (если $ A $ не указано, \\ по умолчанию предполагается $ A = \reals $).} \\
        $ \indicator_A $                 & индикаторная функция множества $ A $. \\
        $ \expect X $                    & математическое ожидание случайной величины $ X $. \\
        $ \dispersion X $                & дисперсия случайной величины $ X $. \\
        $ \rvcenter X $                  & <<центрированная>> случайная величина: $ \rvcenter X = X - \expect X $. \\
        \rule{0pt}{16pt}%
        $ \bernoulli(p) $                & распределение Бернулли с параметром $ p $. \\
        $ \binomial(n, p) $              & биномиальное распределение с параметрами $ n $ и $ p $. \\
        $ \poisson(\lambda) $            & распределение Пуассона с интенсивностью $ \lambda $. \\
        $ \uniform(A)$, $ \uniform_A $   & равномерное распределение на множестве $ A $. \\
        $ \expdistr(\lambda) $           & показательное распределение с параметром $ \lambda $ (интенсивность). \\
        $ \normal(\mu, \sigma^2) $       & нормальное распределение со средним $ \mu $ и дисперсией $ \sigma^2 $. \\
        \rule{0pt}{16pt}%
        $ \convmeansq $                  & сходимость в среднем квадратичном. \\
        $ \convalmost $                  & сходимость почти наверное. \\
        $ \convproba $                   & сходимость по вероятности. \\
        $ \convdistr $                   & сходимость по распределению. \\
        $ \almosteq $                    & равенство почти наверное. \\
        $ \distreq $                     & равенство по распределению.
    \end{tabularx}
\end{center}
          % Введение.
\section{Основные сведения} \label{section:basics}

Случайные процессы~--- математические объекты,
построенные с использованием теории вероятностей для исследования и моделирования реальных явлений,
растянутых во времени и имеющих стохастическую (случайную) природу.

\begin{definition}
    \label{definition:basics:stochastic_process}
    Пусть задано вероятностное пространство $ (\Omega, \setfamily, \proba) $ и множество $ T \subseteq \mathbb{R} $.
    Функция $ X\colon \Omega \times T \to \mathbb{R} $ называется \defemph{случайным процессом},
    если $ \forall t \in T $ функция $ X(\cdot, t) \equiv X_t\colon \Omega \to \mathbb{R} $ измерима
    (то есть является случайной величиной).
\end{definition}

Случайный процесс можно трактовать как семейство случайных величин, параметризованное $ t \in T $.
Параметр $ t $ обычно интерпретируется как время.
Если $ T $ состоит из одного элемента, случайный процесс является обычной случайной величиной,
если $ T $ конечно~--- случайным вектором.
Параметр $ \omega $, как и при описании случайных величин, часто опускается.

\begin{definition}
    \label{definition:basics:stochastic_process_slice}
    При фиксированном $ t_0 \in T $ случайная величина $ X_{t_0} $ называется \defemph{сечением случайного процесса} $ X $.
\end{definition}

\begin{definition}
    \label{definition:basics:stochastic_process_realization}
    При фиксированном $ \omega_0 \in \Omega $ функция $ X(\omega_0, \cdot ) $ называется \defemph{реализацией случайного процесса} $ X $.
\end{definition}

Также случайный процесс можно считать особой случайной величиной, принимающей значения в пространстве функций;
при такой интерпретации, однако, отдельных усилий стоит определить, что такое вероятностное распределение на функциях.
В рамках семинаров данный вопрос освещаться со всей полнотой и строгостью не будет,
поэтому приведём из этой области лишь основные факты и определения,
требующиеся для работы со случайными процессами.

Рассмотрим произвольный случайный процесс $ X $.
В силу единства вероятностного пространства,
любой вектор вида $ (X_{t_1}, \ldots, X_{t_n}) $ (где $ t_i \in T $) является случайным вектором.

\begin{definition}
    \label{definition:basics:finite_distribution}
    Вероятностное распределение вектора вида $ (X_{t_1}, \ldots, X_{t_n}) $ называется \defemph{конечномерным распределением случайного процесса} $ X $.
    Его функция распределения обозначается как $ F_X(x_1, \ldots, x_n; t_1, \ldots, t_n) $.
\end{definition}

Функции распределений векторов, составленных из сечений случайного процесса,
обладают всеми известными вам свойствами функций распределений случайных векторов,
а также ещё двумя дополнительными свойствами:

\begin{statement}
    \label{statement:basics:finite_distribution_properties}
    Функции конечномерных распределений случайного процесса $ X $ обладают следующими свойствами:
    \begin{enumerate}
        \item
            \defemph{(условие симметрии)}
            Для любой перестановки $ k_i $ выполнено равенство
            \[
                F_X(x_1, \ldots, x_n; t_1, \ldots, t_n) = F_X(x_{k_1}, \ldots, x_{k_n}; t_{k_1}, \ldots, t_{k_n})
            \]
        \item
            \defemph{(условие согласованности)}
            Для любого индекса $ k \in \{1, \ldots, n\} $ выполнено
            \[
                \lim_{x_k \to +\infty} F_X(x_1, \ldots, x_n; t_1, \ldots, t_n) = F(x_1, \ldots, x_{k-1}, x_{k+1}, \ldots x_n; t_1, \ldots, t_{k-1}, t_{k+1}, \ldots, t_n)
            \]
    \end{enumerate}
\end{statement}

\begin{theorem}[Колмогорова]
    \label{theorem:basics:finite_distributions_family_define_stochastic_process}
    Пусть имеется семейство распределений случайных векторов,
    удовлетворяющее всем свойствам из утверждения \ref{statement:basics:finite_distribution_properties}.
    Тогда существует вероятностное пространство и заданный на нём случайный процесс,
    семейство конечномерных распределений которого совпадает с данным.
\end{theorem}

Таким образом, случайный процесс можно задавать семейством его конечномерных распределений.
На данном этапе читателю должно стать понятно,
как можно задавать вероятностное распределение на множестве функций
(ответ~--- при помощи специальных семейств конечномерных распределений).

\begin{Exercise}[counter=SecExercise, label={exercise:basics:rv_plus_t}]
    \noindent
    Пусть $ \eta $~--- случайная величина с функцией распределения $ F_\eta $.
    Найти все конечномерные распределения случайного процесса $ X_t = \eta + t $.
\end{Exercise}

\begin{Answer}
    \noindent
    Одномерная функция распределения:
    \[
        F_X(x; t) = \proba \{ \omega \in \Omega \mid X_t < x \} = \proba \{ \eta < x - t \} = F_\eta(x - t)
    \]
    Конечномерная функция распределения:
    \begin{multline*}
        F_X(x_1, \ldots, x_n; t_1, \ldots, t_n) = \proba \bigcap_{i = 1}^n \{ X_{t_i} < x_i \} = \proba \bigcap_{i = 1}^n \{ \eta < x_i - t_i \} = \\
        = \proba \left\{ \eta < \min_i \{x_i - t_i\} \right\} = F_\eta \left(\min_i \{x_i - t_i \} \right)
    \end{multline*}
\end{Answer}

\begin{Exercise}[counter=SecExercise, label={exercise:basics:random_point_om_segment}]
    \noindent
    Пусть дана случайная величина $ \eta \sim \uniform_{[0;1]} $.
    Определим случайный процесс $ X_t = \indicator_{(-\infty; \eta]}(t) $.
    Найдите вид реализаций процесса, его одномерные и двумерные распределения.
\end{Exercise}

\begin{Answer}
    \noindent
    Реализация процесса~--- функция, равная единице при $ t \leqslant \eta $ и нулю при $ t > \eta $, см. рис.~\ref{figure:basics:random_point_on_segment}.
    Одномерная функция распределения:
    \[
        F_X(x;t) = \proba \{ X_t < x \} = \proba \{ \indicator_{(-\infty; \eta]}(t) < x \} =
        \begin{cases}
            0, &\quad x \leqslant 0 \\
            \proba \{\eta < t\}, &\quad 0 < x \leqslant 1 \\
            1, &\quad x > 1
        \end{cases},
    \]
    \[
        \textnormal{где} \quad
        \proba \{\eta < t\} = F_\eta(t) =
        \begin{cases}
            0, &\quad t \leqslant 0 \\
            t, &\quad 0 < t \leqslant 1 \\
            1, &\quad t > 1
        \end{cases}
    \]
    Двумерная функция распределения:
    \[
        F_X(x_1, x_2; t_1, t_2) = \proba \left( \{X_{t_1} < x_1\} \cap \{X_{t_2} < x_2\} \right)
    \]
    Аналогично одномерной функции распределения,
    \begin{enumerate}
        \item
            Если $ x_1 \leqslant 0 $ или $ x_2 \leqslant 0 $, $ F_X(x_1, x_2; t_1, t_2) = 0 $.
        \item
            Если $ x_1 > 1 $ и $ x_2 > 1 $, $ F_X(x_1, x_2; t_1, t_2) = 1 $.
        \item
            Если $ 0 < x_1 \leqslant 1 $ и $ x_2 > 1 $, $ F_X(x_1, x_2; t_1, t_2) = F_X(x_1; t_1) $.
            Аналогично симметричный случай.
        \item
            Если $ 0 < x_1, x_2 \leqslant 1 $,
            \[
                F(x_1, x_2; t_1, t_2) = \proba \left( \{\eta < t_1\} \cap \{\eta < t_2\} \right) =
                \proba \left\{ \eta < \min\{t_1, t_2\} \right\} = F_\eta\left( \min\{t_1, t_2\} \right)
                %\begin{cases}
                %    0, &\quad \min\{t_1, t_2\} < 0 \\
                %    \min\{t_1, t_2\}, &\quad 0 \leqslant \min\{t_1, t_2\} < 1 \\
                %    1, &\quad \min\{t_1, t_2\} > 1
                %\end{cases}
            \]
    \end{enumerate}
\end{Answer}

\begin{figure}[ht!]
    \centering
    \begin{gnuplot}[terminal=epslatex, terminaloptions={color size 12cm,8cm}]
        set xlabel  "$ t $"
        set xrange  [ 0 : 1 ] noreverse writeback
        set ylabel  "$ X_t $"
        set yrange  [ -0.1 : 1.1 ] noreverse writeback

        # Functions

        eta = 0.42
        part1(x) = (x <= eta ? 1.0 : 1/0)
        part2(x) = (x >= eta ? 0.0 : 1/0)
        #indicator(x) = (x <= eta ? 1.0 : 0.0)
        #set samples 1000

        # Grid

        set style line 110 lt 1 lc rgb "#EE5555" lw 8

        set style line 100 lt 1 lc rgb "#444444" lw 1
        set style line 101 lt 1 lc rgb "#CCCCCC" lw 1
        set style line 102 lt 1 lc rgb "#EEEEEE" lw 1

        set style line 105 lt 1 lc rgb "#444444" lw 3

        set grid ytics mytics mxtics xtics ls 100, ls 101

        # Arrows

        unset border
        set arrow from graph 0.0,0.083333 to graph 1.05,0.083333 size screen 0.025,15,60 filled ls 105
        set arrow from graph 0.0,0.0 to graph 0.0,1.05 size screen 0.025,15,60 filled ls 105

        # Plotting

        set key noautotitle
        plot [0:1] part1(x) notitle ls 110, part2(x) t "$ X(\\omega_0, t) $" ls 110
    \end{gnuplot}
    %\vspace{-32pt}
    \caption{График одной из реализаций случайного процесса из задачи \ref{exercise:basics:random_point_om_segment}.}
    \label{figure:basics:random_point_on_segment}
\end{figure}

Существование различных случайных процессов с одними и теми же
вероятностными свойствами приводит к желанию (а иногда и необходимости)
в некотором смысле отождествлять процессы,
у которых конечномерные распределения совпадают.

\begin{definition}
    \label{definition:basics:modification}
    Пусть $ X $ и $ Y $~--- два случайных процесса,
    определённые на одном и том же вероятностном пространстве $ (\Omega, \setfamily, \proba) $ и множестве $ T $.
    Данные процессы называются \defemph{стохастически эквивалентными} в случае равенства почти наверное их реализаций в любой выбранный момент,
    то есть
    \[
        \forall t \in T \quad \proba \{ \omega \in \Omega \mid X(\omega, t) = Y(\omega, t) \} = 1
    \]
    В этом случае $ Y $ называют \defemph{модификацией} процесса $ Y $ (и наоборот).
\end{definition}

\begin{statement}
    \label{statement:basics:finite_distributions_of_modifications}
    Стохастически эквивалентные случайные процессы имеют одинаковое семейство конечномерных распределений.
\end{statement}

Например, такое отождествление полезно для осмысленного определения непрерывного случайного процесса:
\begin{definition}
    \label{definition:basics:continious_stochastic_process}
    Случайный процесс называется \defemph{непрерывным} в случае,
    если существует его модификациея с непрерывными реализациями.
\end{definition}

\begin{Exercise}[counter=SecExercise, label={exercise:basics:continious_stochastic_process}]
    \noindent
    Пусть $ \eta \sim \uniform_{[0;1]} $.
    Определим случайный процесс $ X_t = \indicator_{\{\eta\}}(t) $
    (то есть $ X_t = 1 $ в том и только в том случае, когда $ \eta = t $, и равен $ 0 $ иначе).
    Является ли $ X_t $ непрерывным процессом?
\end{Exercise}

\begin{Answer}
    \noindent
    Да, является.
    Процесс $ Y_t \equiv 0 $ является его модификацией.
\end{Answer}


При исследовании случайных процессов также бывает полезно рассматривать их моменты,
дающие некоторое представление об усреднённом поведении процесса.
В отличие от случайных величин, любые моменты случайного процесса также зависят от времени.

\begin{definition}
    \label{definition:basics:mean_function}
    Если $ \forall t \in T $ существует $ \expect X_t $,
    то функция $ m_X(t) = \expect X_t $ определена и называется \defemph{функцией среднего}.
\end{definition}

Аналогично вводятся функции любых других моментов случайной величины $ X_t $.
При работе со случайными процессами нас также будут интересовать моменты,
<<разнесённые во времени>>.

\begin{definition}
    \label{definition:basics:second_order_moment_functions}
    Если $ \forall t_1, t_2 \in T $ существует $ \expect X_{t_1} X_{t_2} $,
    то функции $ K_X(t_1, t_2) = \expect X_{t_1} X_{t_2} $ и $ R_X(t_1, t_2) = \expect \rvcenter X_{t_1} \rvcenter X_{t_2} $
    определены и называются, соответственно, \defemph{ковариационной} и \defemph{корреляционной функциями}.%
    \footnote{Данные обозначения не являются общепринятыми, а также несколько контринтуитивны; при чтении сторонних источников будьте внимательны.}
\end{definition}

\begin{statement}
    \label{statement:basics:correlation_and_covariation_connection}
    Функции $ K_X(t_1, t_2) $ и $ R_X(t_1, t_2) $ одновременно либо определены, либо не определены,
    причём в первом случае функция $ m_X(t) $ определена и $ R_X(t_1, t_2) = K_X(t_1, t_2) - m_X(t_1) m_X(t_2) $.
\end{statement}

\begin{proof}
    Следует из свойств моментов.
\end{proof}

\begin{Exercise}[counter=SecExercise, label={exercise:basics:random_point_om_segment_moments}]
    \noindent
    Найти корреляционную функцию случайного процесса из задачи \ref{exercise:basics:random_point_om_segment}.
\end{Exercise}

\begin{Answer}
    \noindent
    Для любого $ t_0 $ случайная величина $ X_{t_0} $ может принимать только два значения~--- $ 0 $ или $ 1 $;
    это бернуллиевская случайная величина.
    Найдём параметр её распределения:
    \[
        \proba \{X_t = 1\} = \proba \{t < \eta\} = 1 - F_\eta(t)
    \]
    Следовательно, $ m_X(t) = \expect X_t = 1 - F_\eta(t) $.
    Далее,
    \begin{multline*}
        K_X(t_1, t_2) = \expect X_{t_1} X_{t_2} = 1 \cdot \proba \left( \{X_{t_1} = 1\} \cap \{X_{t_2} = 1\} \right) = \\
        = \proba \left( \{t_1 < \eta\} \cap \{t_2 < \eta\} \right) = 1 - F_\eta\left( \max \{t_1, t_2\} \right)
    \end{multline*}
    Наконец,
    \begin{multline*}
        R_X(t_1, t_2) = 1 - F_\eta\left( \max \{t_1, t_2\} \right) - (1 - F_\eta(t_1)) \cdot (1 - F_\eta(t_2)) = \\
        = F_\eta(t_1) + F_\eta(t_2) - F_\eta(t_1) \cdot F_\eta(t_2) - F_\eta(\max\{t_1, t_2\})
    \end{multline*}
    В частности, если $ t_1, t_2 \in [0; 1] $,
    \[
        R_X(t_1, t_2) = t_1 + t_2 - t_1 t_2 - \max\{t_1, t_2\} = \min\{t_1, t_2\} - t_1 t_2
    \]
\end{Answer}

\begin{Exercise}[counter=SecExercise, label={exercise:basics:cosine_stochastic_process}]
    \noindent
    Пусть $ \xi \sim \normal(0, 1) $ и $ \eta \sim U_{[-\pi; \pi]} $~--- независимые случайные переменные.
    Определим случайный процесс $ X $ следующим образом: $ X_t = \xi \cdot \cos(t + \eta) $, где $ t \in \reals $.
    Найдите функцию среднего и корреляционную функцию процесса.
\end{Exercise}

\begin{Answer}
    \noindent
    Поскольку $ \xi $ и $ \eta $ независимы,
    \[
        m_X(t) = \expect X_t = \expect \xi \cdot \expect \cos(t + \eta) = 0 \cdot \ldots = 0
    \]
    \begin{multline*}
        R_X(t_1, t_2) = K_X(t_1, t_2) - 0 = \expect X_{t_1} X_{t_2} = \expect \xi^2 \cdot \expect \left( \cos(t_1 + \eta) \cdot \cos(t_2 + \eta) \right) = \\
        = 1 \cdot \frac{1}{2} \expect \left( \cos(t_1 - t_2) + \cos(t_1 + t_2 + 2 \eta) \right) = \frac{1}{2} \cos(t_1 - t_2)
    \end{multline*}
\end{Answer}

\begin{Exercise}[counter=SecExercise, label={exercise:basics:cos_and_sin}]
    \noindent
    Пусть $ U $, $ V $ и $ W $~--- независимые в совокупности случайные величины.
    Известно, что $ U $ и $ V $ обладают нулевым матожиданием и дисперсией $ D $,
    а $ W $ распределена с плотностью
    \[
        \rho_W(w) = \frac{2 \lambda}{\pi} \cdot \frac{\indicator_{[0; +\infty)}(w)}{\lambda^2 + w^2}, \quad \lambda > 0
    \]
    Определим случайный процесс $ X_t = U \cos(W t) + V \sin(W t) $.
    Вычислите функцию среднего и корреляционную функцию.
\end{Exercise}

\begin{Answer}
    \noindent
    Поскольку $ U $, $ V $ и $ W $ независимы в совокупоности,
    \[
        m_X(t) = \expect X_t = \expect U \cdot \expect \cos(W t) + \expect V \cdot \expect \sin(W t) = 0 \cdot \ldots + 0 \cdot \ldots = 0
    \]
    Корреляционную функцию удобно искать с помощью формулы полной вероятности в непрерывном случае:
    \[
        R_X(t_1, t_2) = \expect \left( \expect(X_{t_1} X_{t_2} \mid W = w) \right) = \int\limits_\reals \underbrace{\expect(X_{t_1} X_{t_2} \mid W = w)}_{\defeq R(t_1, t_2 \mid w)} \cdot \rho_W(w) \, dw
    \]
    В силу нулевого матожидания,
    \begin{multline*}
        R_X(t_1, t_2) = \expect \left( (U \cos(w t_1) + V \sin(w t_1)) \cdot (U \cos(w t_2) + V \sin(w t_2)) \right) = \\
        = \expect (U^2) \cdot \cos(w t_1) + 2 \cdot \underbrace{\expect U \expect V}_{0} \cdot \ldots + \expect (V^2) \cdot \sin(w t_1) \sin(w t_2) = D \cos(w(t_1 - t_2))
    \end{multline*}
    Наконец,
    \[
        R_X(t_1, t_2) = \int\limits_0^{+\infty} D \cos(w(t_1 - t_2)) \cdot \frac{2 \lambda}{\pi} \frac{1}{\lambda^2 + w^2} \, dw = D e^{-\lambda |t_1 - t_2|}
    \]
    Здесь использовалось значение интеграла Лапласа:
    \[
        \int\limits_0^\infty \frac{\cos(\alpha x)}{1 + x^2} \, dx = \frac{\pi}{2} e^{-|\alpha|}
    \]
\end{Answer}


\begin{definition}
    \label{definition:basics:correlation_coefficient_function}
    \defemph{Функцией коэффициента корреляции} называют функцию
    \[
        r_X(t_1, t_2) = \frac{R_X(t_1, t_2)}{\sqrt{R_X(t_1, t_1) \cdot R_X(t_2, t_2)}} = \frac{\covariance{X_{t_1}}{X_{t_2}}}{\sqrt{\dispersion X_{t_1} \dispersion X_{t_2}}}
    \]
\end{definition}

Данная функция, если определена, принимает значения от $ -1 $ до $ 1 $
и имеет смысл степени \uline{линейной} связи сечений процесса,
соответствующих выбранным моментам времени.

\begin{Exercise}[counter=SecExercise, label={exercise:basics:correlation_coefficient_function}]
    \noindent
    Найти функции коэффициента корреляции для процессов из задач \ref{exercise:basics:cosine_stochastic_process} и \ref{exercise:basics:cos_and_sin}.
\end{Exercise}

\begin{Answer}
    \noindent
    \begin{itemize}
        \item
            Задача \ref{exercise:basics:cosine_stochastic_process}:
            $
                \displaystyle
                r_X(t_1, t_2) = \frac{\frac{1}{2} \cos(t_1 - t_2)}{\sqrt{\frac{1}{2} \cdot \frac{1}{2}}} = \cos(t_1 - t_2)
            $.

            Если взять два произвольных момента времени и начать сдвигать их друг к другу или друг от друга,
            будет наблюдаться периодическая корреляция и декорреляция соответствующих сечений.
        \item
            Задача \ref{exercise:basics:cos_and_sin}:
            $
                \displaystyle
                r_X(t_1, t_2) = \frac{D e^{-\lambda |t_1 - t_2|}}{\sqrt{D e^{-\lambda \cdot 0} \cdot D e^{-\lambda \cdot 0}}} = e^{-\lambda |t_1 - t_2|}
            $.

            Несмотря на схожесть процессов, в данном случае наблюдается корреляция,
            затухающая экспоненциально с ростом разницы между моментами времени,
            в которых взяты сечения.

            Дело в том, что в первом процессе случайным был фазовый сдвиг, а потому реализации процесса <<не расползались>>.
            Во втором же случае случайной является ещё и частота, и линейная связь между разными моментами времени быстро теряется
            (реализации <<декогерируют>>).
    \end{itemize}
\end{Answer}
     % Общие сведения.
\section{Важные примеры случайных процессов} \label{section:special}

В этом разделе речь пойдёт о нескольких процессах особого вида,
наиболее часто встречающихся при исследовании реальных явлений.
Зачастую такие процессы именные.
На их примере мы продолжим практиковаться в решении задач,
а также введём несколько новых теоретических понятий.


\subsection{Пуассоновский процесс} \label{subsection:Poisson}

Данный процесс встречается в реальной жизни довольно часто;
он описывает поток случайных событий, которые регистрируются с некоторой постоянной <<интенсивностью>>.
Например, речь может идти о регистрации космических частиц, о кликах по ссылке,
о запросах к серверу, о проезжающих по магистрали автомобилях.

Пуассоновский процесс можно неформально определить следующим образом:
пусть ось времени разбита на бесконечно малые промежутки $ \Delta t $.
Тогда пуассоновский процесс ведёт себя следующим образом:
в самом начале он равен нулю,
и на каждом последующем шаге по времени может претерпеть скачок на $ + 1 $ с вероятностью $ \lambda \Delta t $.
Параметр $ \lambda $ называется интенсивностью процесса и характеризует <<скорость>> потока событий.
Дадим формальное определение:

\begin{definition}[Явная конструкция пуассоновского процесса]
    \label{definition:special:Poisson_process_explicit_definition}
    Пусть $ \xi_1, \ldots, \xi_k, \ldots \sim \expdistr(\lambda) $ и независимы в совокупности,
    $ \tau_n = \xi_1 + \ldots + \xi_n $.
    Тогда процесс $ K_t = \sup \{ n \mid \tau_n \leqslant t \} $ называется \defemph{пуассоновским процессом с интенсивностью $ \lambda $}.
\end{definition}

Процесс $ K_t $, построенный
%по случайным величинам $ \xi_k $
способом, указанным выше,
называется \defemph{процессом восстановления, построенным по величинам $ \{ \xi_k \}_{k \in \naturals} $}, и отвечает следующей модели:
в нулевой момент включается прибор, который работает время $ \xi_1 $, после чего ломается.
Одновременно с поломкой включается следующий прибор, который работает случайное время $ \xi_2 $, и так далее.
Величина $ K_t $ отражает количество приборов, введённых в эксплуатацию к моменту $ t $.

\begin{figure}[ht!]
    \centering
    \begin{gnuplot}[terminal=epslatex, terminaloptions={color size 16cm,8cm}]
        set xlabel  "$ t $"
        set xrange  [ 0 : * ] noreverse writeback
        set ylabel  "$ K_t $"
        set yrange  [ 0 : * ] noreverse writeback

        # Grid

        set style line 100 lt 1 lc rgb "#444444" lw 1
        set style line 101 lt 1 lc rgb "#CCCCCC" lw 1
        set style line 102 lt 1 lc rgb "#EEEEEE" lw 1

        set style line 105 lt 1 lc rgb "#444444" lw 3

        set mxtics 5
        set mytics 5
        set grid ytics mytics mxtics xtics ls 100, ls 102

        # Plotting

        set datafile separator ','
        set key autotitle columnhead

        filename = './data/Poisson_realizations.csv'

        stats filename nooutput
        n_cols = STATS_columns > 9 ? 9 : STATS_columns  # В палитре по умолчанию всего 8 цветов.

        plot [0:*] for [i=2:n_cols] filename using 1:i with histeps lw 3 notitle
    \end{gnuplot}
    %\vspace{-32pt}
    \caption{Пример пучка реализаций пуассоновского процесса с интенсивностью $ \lambda = 2 $.}
    \label{figure:special:Poisson_proccess_realizations}
\end{figure}


Приведённая явная конструкция возвращает нас к неформальному определению,
использующему дискретное время с шагом $ \Delta t $.
Можно заметить, что экспоненциальное распределение получается как
предел вероятностного распределения случайной величины~---
времени между соседними скачками~--- при $ \Delta t \to +0 $:
\[
    \proba \{ \xi_i \in [t; t+h) \} = \lim_{\Delta t \to +0} \left( 1 - \lambda \Delta t \right)^{\frac{t}{\Delta t}} \cdot \left( \lambda \Delta t \cdot \frac{h}{\Delta t} + o(h) \right) =
    \lambda e^{-\lambda t} (h + o(h))
\]

Пуассоновский процесс можно определить и иначе.
Для этого введём понятие процесса с независимыми приращениями.

\begin{definition}
    \label{definition:special:independent_deltas}
    Случайный процесс $ X $ называется \defemph{процессом с независимыми приращениями},
    если $ \forall n \in \naturals \;\, \forall \{t_i\}_{i=1}^n \subseteq T $ случайные величины
    $ X_{t_n} - X_{t_{n-1}}, \ldots, X_{t_2} - X_{t_1}, X_{t_1} $
    независимы в совокупности.
\end{definition}

\begin{definition}
    \label{definition:special:Poisson_process}
    \defemph{Пуассоновским процессом с интенсивностью $ \lambda > 0 $} называется случайный процесс $ K\colon \Omega \times [0; +\infty) \to \naturals $ такой, что
    \begin{enumerate}
        \item
            $ K_0 \almosteq 0 $.
        \item
            $ K $~--- процесс с независимыми приращениями.
        \item
            $ K_t - K_s \sim \poisson\left( \lambda \cdot (t - s) \right) $ (при $ t > s \geqslant 0 $).
    \end{enumerate}
\end{definition}

\begin{theorem}
    \label{theorem:special:Poisson_process_definitions_equivalence}
    Определения \ref{definition:special:Poisson_process_explicit_definition} и \ref{definition:special:Poisson_process} эквивалентны.
\end{theorem}

\begin{statement}
    \label{statement:special:Poisson_process_properties}
    Пуассоновский процесс обладает следующими свойствами:
    \begin{enumerate}
        \item
            Реализации пуассоновского процесса~--- кусочно-постоянные неубывающие функции со значениями в $ \naturals $.
        \item
            С вероятностью $ 1 $ все скачки пуассоновского процесса равны единице.
        \item
            Время, когда произошёл $ n $-ый скачёк (обозначим его $ \tau_n $) имеет $ \Gamma(n, 1/\lambda) $-распределение:
            \[
                \rho_{\tau_n}(t) = \frac{\lambda^n x^{n-1}}{(n-1)!} e^{-\lambda t} \cdot \indicator_{[0;+\infty)}(t)
            \]
        \item
            Случайные величины $ \{\tau_{n} - \tau_{n-1}\}_{n \in \naturals} $ распределены экспоненциально с параметром $ \lambda $ и независимы.
        \item
            Число событий за конечный период времени конечно с вероятностью $ 1 $.
        \item
            Число событий $ K_{t+h} - K_t $ на промежутке $ (t; t+h] $ зависит лишь от длины промежутка $ h $:
            $ \proba \{ K_{t + h} - K_t = k \} = p(h, k) $
        \item
            Вероятность более чем одного скачка на полуинтервале $ (t; t + h] $ есть $ o(h) $,
            то есть $ \displaystyle \lim_{h \to +0} \proba \{ K_{t+h} - K_t > 1 \} / h = 0 $.
        \item
            Для коротких полуинтервалов $ (t; t+h] $ вероятность того, что на них произойдёт хотя бы один скачок,
            убывает линейно с уменьшением $ h $: $ \proba \{ K_{t+h} - K_t > 0 \} = 1 - e^{-\lambda h} = \lambda h + o(h) $ при $ h \to 0 $.
        \item
            Из определения распределения Пуассона:
            \[
                \proba\{K_t = k\} = \frac{(\lambda t)^k}{k!} e^{-\lambda}
            \]
    \end{enumerate}
\end{statement}

Наконец, приведём ещё одно из альтернативных определений пуассоновского процесса:

\begin{statement}
    \label{statement:special:Poisson_process_alternative_definition}
    Случайный процесс $ K\colon \Omega \times T \to \naturals $ является пуассоновским тогда и только тогда, когда он удовлетворяет следующим свойствам:
    \begin{enumerate}
        \item
            \defemph{(стационарность приращений)}
            $ \proba \{ K_{t + h} - K_t = k \} = p(h, k) $
        \item
            \defemph{(отсутствие последействия)}
            Приращения процесса независимы.
        \item
            \defemph{(ординарность)}
            $ \proba \{ K_{t + h} - K_t > 1 \} \in o(h) $
    \end{enumerate}
\end{statement}


\begin{statement}
    \label{statement:special:Poisson_process_correlation_function}
    Пусть $ K $~--- пуассоновский процесс с интенсивностью $ \lambda $.
    Тогда $ m_K(t) = \lambda t $, $ R_K(t, s) = \lambda \cdot \min \{t, s\} $.
\end{statement}

\begin{proof}
    Так как $ K_t \almosteq K_t - K_0 \sim \poisson(\lambda t) $, $ m_K(t) = \expect K_t = \lambda t $.
    Далее, в силу независимости приращений, при $ t \geqslant s $ имеем
    $ \covariance{K_t}{K_s} = \covariance{K_t - K_s + K_s}{K_s} = 0 + \covariance{K_s}{K_s} = \lambda t $.
    Поэтому $ R_k(t, s) = \lambda \cdot \min \{t, s\} $.
\end{proof}


\begin{Exercise}[counter=SecExercise, label={exercise:special:total_wait_time}]
    \noindent
    Поток прибывающих на железнодорожную станцию пассажиров моделируется пуассоновским процессом $ K $ с интенсивностью $ \lambda $.
    В момент $ t = 0 $ пассажиров нет, в момент $ t = t_0 $ прибывает первый поезд.
    Пусть $ \eta $~--- суммарное время ожидания прибытия поезда всеми пассажирами на станции.
    Найти $ \expect \eta $.
\end{Exercise}

\begin{Answer}
    \noindent
    \[
        \eta = \int\limits_0^{t_0} K_t \, dt, \qquad
        \expect \eta = \int\limits_\Omega d\proba \int\limits_0^{t_0} K_t \, dt = \int\limits_0^{t_0} dt \int\limits_\Omega K_t \, d \proba = \int\limits_0^{t_0} m_K(t) \, dt =
        \int\limits_0^{t_0} \lambda t \, dt = \frac{\lambda t_0^2}{2}
    \]
\end{Answer}



\begin{Exercise}[counter=SecExercise, label={exercise:special:Poisson_process_first_event_conditional}]
    \noindent
    Пусть $ K_t $~--- пуассоновский процесс с интенсивностью $ \lambda $,
    а $ \tau_1 $~--- момент первого скачка.
    Найдите $ \proba\{\tau_1 \leqslant s \mid K_t = 1\} $ при $ 0 < s < t $.
\end{Exercise}

\begin{Answer}
    \noindent
    Событие $ \{ \tau_1 \leqslant s \} $ означает, что первый скачок процесса произошёл не позже момента $ s $.
    Если при этом $ K_t = 1 $, то это означает, что $ K_s = 1 $.
    Тогда
    \begin{multline*}
        \proba \{\tau_1 \leqslant s \mid K_t = 1 \} = \proba \{K_s = 1 \mid K_t = 1\} = \frac{\proba\left( \{K_s = 1\} \cap \{K_t = 1\} \right)}{\proba \{K_t = 1 \}} = \\
        = \frac{\proba\left( \{K_s = 1\} \cap \{K_t - K_s = 0\} \right)}{\proba\{K_t = 1\}}
        = \frac{\frac{\lambda s}{1!} e^{-\lambda s} \cdot \frac{(\lambda(t - s))^0}{0!} e^{-\lambda(t - s)}}{\frac{\lambda t}{1!} e^{-\lambda t}} = \frac{s}{t}
    \end{multline*}
\end{Answer}


\begin{Exercise}[counter=SecExercise, label={exercise:special:Poisson_process_third_event}]
    \noindent
    Пусть $ K $~--- пуассоновский процесс с интенсивностью $ \lambda $, $ \tau_3 $~--- время третьего скачка процесса.
    Найти $ \proba \{\tau_3 \leqslant 2 \} $.
\end{Exercise}

\begin{Answer}
    \noindent
    \[
        \proba \{ \tau_3 \leqslant 2 \} = \proba \{ K_2 \geqslant 3 \} = 1 - \proba \{ K_2 < 3 \} = 1 - e^{-2 \lambda} - \frac{2 \lambda}{1!} e^{-2 \lambda} - \frac{(2 \lambda)^2}{2!} e^{-2 \lambda}
    \]
\end{Answer}


\begin{Exercise}[counter=SecExercise, label={exercise:special:Poisson_process_no_event}]
    \noindent
    Пусть $ \eta \sim \uniform_{[0; 1]} $,
    $ K $~--- пуассоновский процесс с интенсивностью $ \lambda $, и $ \eta $ не зависит от $ K $.
    Найти $ \proba\{K_\eta = K_{\eta + 1}\} $.
\end{Exercise}

\begin{Answer}
    \noindent
    По формуле полной вероятности,
    \[
        \proba \{K_\eta = K_{\eta + 1}\} = \int\limits_\reals \proba\{\underbrace{K_{t + 1} - K_t}_{\sim \poisson(1 \cdot \lambda)} = 0 \mid \eta = t\} \cdot \rho_\eta(t) \, dt =
        \int\limits_0^1 e^{-1 \cdot \lambda} \, dt = e^{-\lambda}
    \]
\end{Answer}

Пуассоновский процесс моделирует лишь поток некоторых событий.
Иногда сами события также имеют сложную и/или случайную природу.
Тогда требуется построить более продвинутую модель,
наследующую от пуассоновского процесса только характер возникновения событий с течением времени.
В качестве примера такой модели можно привести \defemph{сложный (составной) пуассоновский процесс}.
Данный процесс может возникнуть, например, при моделировании покупок в магазине:
каждый покупатель будет появляться на кассе согласно пуассоновскому процессу,
при этом закупаясь на некоторое случайное количество денег.

\begin{definition}
    \label{definition:special:compound_Poisson_process}
    Рассмотрим пуассоновский процесс $ K $ и набор независимых (в совокупности с $ K $) одинаково распределённых случайных величин $ \{ V_k \}_{k \in \naturals} $.
    \defemph{Сложным пуассоновским процессом} называется процесс $ \displaystyle Q_t = \sum_{j = 1}^{K_t} V_j $.
\end{definition}

Это означает следующее: $ Q_0 \almosteq 0 $, и в каждый момент, когда $ K $ испытывает скачок, к $ Q $ добавляется $ V_j $.

\begin{statement}
    \label{statement:special:compound_Poisson_process_independent_deltas}
    Сложный пуассоновский процесс является процессом с независимыми приращениями.
\end{statement}

\begin{proof}
    Следует из независимости приращений $ K $ и независимости $ \{V_j\}_{j \in \naturals} $ в совокупности с $ K_t $.
\end{proof}


\begin{statement}
    \label{statement:special:compound_Poisson_process_characteristic_function}
    Рассмотрим сложный пуассоновский процесс $ Q $ с интенсивностью $ \lambda $,
    определённый по случайным величинам $ \{ V_j \}_{j \in \naturals} $.
    Пусть $ \varphi_V(s) $~--- характеристическая функция случайных величин $ V_j $.
    Тогда характеристичекая функция процесса $ Q $ задаётся формулой
    \[
        \varphi_{Q_t}(s) = e^{(\varphi_V(s) - 1) \cdot \lambda t}
    \]
\end{statement}

\begin{proof}
    \begin{multline*}
        \varphi_{Q_t}(s) = \sum_{k = 0}^\infty \expect \left( e^{i s \cdot Q_t} \mid K_t = k \right) \proba \{ K_t = k \} =
        \sum_{k = 0}^\infty \expect \left( e^{i s \cdot (V_1 + \ldots + V_k)} \right) \cdot \frac{(\lambda t)^k}{k!} e^{-\lambda t} = \\
        = \sum_{k = 0}^\infty (\varphi_V(s))^k \cdot \frac{(\lambda t)^k}{k!} e^{-\lambda t} = e^{\varphi_V(s) \cdot \lambda t} \cdot e^{-\lambda t}
    \end{multline*}
\end{proof}


\begin{corollary}
    \label{corollary:special:compound_Poisson_process_moments}
    Функция среднего и корреляционная функция сложного пуассоновского процесса имеют вид, соответственно,
    \[
        m_Q(t) = \lambda t \cdot \expect V, \qquad
        R_Q(t, s) = \lambda \min\{t,s\} \cdot \expect (V^2)
    \]
\end{corollary}

\begin{proof}
    По свойству характеристической функции,
    \begin{multline*}
        m_Q(t) = \expect Q_t = \left . -i \frac{\partial \varphi_{Q_t}(s)}{\partial s} \right|_{s = 0} =
        \left. -i \frac{\partial}{\partial s} \left( e^{(\varphi_V(s) - 1) \cdot \lambda t} \right) \right|_{s = 0} = \\
        = \lambda t \cdot \underbrace{\left( \left. - i \frac{\partial \varphi_V(s)}{\partial s} \right|_{s = 0} \right)}_{\expect V} \cdot
            \underbrace{e^{(\varphi_V(0) - 1) \cdot \lambda t}}_{e^0} =
        \lambda t \cdot \expect V
    \end{multline*}
    Пользуясь независимостью приращений и полагая $ t \leqslant s $,
    \[
        R_Q(t,s) = \expect Q_t Q_s - m_Q(t) m_Q(s) = \expect \left( Q_t (Q_s - Q_t) \right) + \expect Q_t^2 - \lambda^2 t s \cdot (\expect V)^2
    \]
    \[
        \expect \left( Q_t (Q_s - Q_t) \right) = \expect Q_t \cdot \expect (Q_s - Q_t) = \lambda t \cdot \expect V \cdot \lambda (s - t) \cdot \expect V = \lambda^2 t (s - t) \cdot (\expect V)^2
    \]
    \[
        \expect (Q_t)^2 = \left. (- i)^2 \frac{\partial^2}{\partial s^2} \left( e^{(\varphi_V(s) - 1) \cdot \lambda t} \right) \right|_{s = 0} = (\lambda t)^2 (\expect V)^2 + \lambda t \cdot \expect (V^2)
    \]
    Собирая всё вместе, получаем
    \[
        R_Q(t,s) = \lambda^2 [ \underbrace{t(s - t) + t^2 - ts}_{0} ] \cdot (\expect V)^2 + \lambda t \cdot \expect (V^2) = \lambda t \cdot \expect (V^2)
    \]
    В общем же случае $ R_Q(t, s) = \lambda \min\{t,s\} \cdot \expect (V^2) $.
\end{proof}


\begin{Exercise}[counter=SecExercise, title={(Прореживание пуассоновского процесса)}]
    \noindent
    Пусть $ K_t $~--- пуассоновский процесс с интенсивностью $ \lambda $,
    а случайные величины $ \{V_j\}_{j \in \naturals} $ независимы и имеют распределение Бернулли с параметром $ p $.
    Покажите, что $ Q_t $~--- также пуассоновский процесс с интенсивностью $ p \lambda $.
\end{Exercise}

\begin{Answer}
    \noindent
    Пуассоновский процесс также является сложным пуассоновским процессом с $ V_j \equiv 1 $.
    Тогда характеристическая функция пуассоновского процесса:
    \[
        \varphi_{K_t}(s) = e^{\left(e^{i s} - 1 \right) \cdot \lambda t}
    \]
    Характеристическая функция <<прореженного>> процесса $ Q $:
    \[
        \varphi_{Q_t}(s) = e^{\left(p \, e^{i s} + (1 - p) - 1 \right) \cdot \lambda t} = e^{\left(e^{i s} - 1 \right) \cdot p \lambda t}
    \]
    Имеем характеристическую функцию пуассоновского процесса с интенсивностью $ p \lambda $.
    В общем случае этого недостаточно для того, чтобы утверждать, что процесс пуассоновский;
    нужно равенство характеристических функций всех конечномерных распределений.
    Но мы имеем дело с процессом с независимыми приращениями, поэтому характеристической функции среза нам достаточно
    (это утверждение мы оставим без доказательства).
\end{Answer}




\subsection{Гауссовские процессы} \label{subsection:gaussian}

Гауссовские процессы могут возникать при исследовании броуновского движения,
динамики цен акций, эволюции квантово-механических систем и стохастических космологических моделей.
Также гауссовские процессы часто используется как <<шумовая составляющая>> других случайных процессов.

\begin{definition}
    \label{definition:special:gaussian_process}
    Случайный процесс, все векторы срезов которого являются гауссовскими,
    называется \defemph{гауссовским случайным процессом}.
\end{definition}

Напомним, что гауссовские векторы обладают рядом полезных свойств:
распределение гауссовского вектора полностью задаётся вектором среднего и матрицей ковариации,
а нескоррелированность компонент полностью эквивалентна независимости.
Аналогичные свойства можно доказать и для гауссовских процессов.
Однако для этого требуется ввести следующее определение:

\begin{definition}
    \label{definition:special:positive_semi_definite_function}
    Функция $ g(x, y) \colon X \times X \to \complexes $ называется \defemph{симметричной неотрицательно определённой},
    если $ \forall x, y \in X \;\, g(x, y) = g(y, x) $ и $ \forall n \in \naturals, \; \forall \{x_i\}_{i=1}^n, \{y_j\}_{j=1}^n \subseteq X $
    матрица $ (g(x_i, y_j)) = G_{\{x_i\}, \{y_j\}} $ неотрицательна определена как оператор над $ \complexes^n $,
    то есть $ \forall z \in \complexes^n \; \dotprod{z}{G_{\{x_i\}, \{y_j\}} \, z} \geqslant 0 $.
\end{definition}

\begin{remark}
    \label{remark:special:positive_semi_definite_function_for_reals}
    Если $ g(x, y) $ принимает только вещественные значения,
    в определении \ref{definition:special:positive_semi_definite_function} можно рассматривать только $ z \in \reals^n $.
\end{remark}

\begin{statement}
    \label{statement:special:mean_and_cov_define_gaussian_process}
    Пусть $ m \colon T \to \reals $~--- произвольная функция,
    а $ R \colon T \times T \to \reals $~--- симметричная и неотрицательно определённая функция.
    Тогда существует гауссовский процесс $ X $ такой, что $ \expect X_t = m(t) $, $ \expect \rvcenter X_s \rvcenter X_t = R(s, t) $.
\end{statement}

В курсе теории вероятностей вы уже встречались с неотрицательно определёнными функциями.
В частности, все характеристические функции случайных величин неотрицательно определены.

\begin{statement}
    \label{statement:special:characteristic_function_is_positive_semi_definite}
    Пусть $ \varphi_\xi(s) $~--- характеристическая функция некоторой \uline{случайной величины} $ \xi $.
    Тогда функция $ g(s, t) = \varphi_\xi(t - s) $~--- симметричная и неотрицательно определённая.
\end{statement}


\begin{Exercise}[counter=SecExercise, label={exercise:special:gaussian_from_characteristic_function_of_Cauchy}]
    \noindent
    Существует ли гауссовский процесс с корреляционной функцией $ R(s, t) = e^{-|s - t|} $?
\end{Exercise}

\begin{Answer}
    \noindent
    Да, существует.
    Мы знаем, что $e^{-|t|} $ есть характеристическая функция распределения Коши.
    Поэтому это неотрицательно определённая функция.
    Значит, $ R(s, t) = e^{-|s - t|} $ неотрицательно определена и симметрична.
\end{Answer}


\begin{figure}[ht!]
    \centering
    \begin{gnuplot}[terminal=epslatex, terminaloptions={color size 16cm,6cm}]
        set xlabel  "$ t $"
        set xrange  [ 0 : * ] noreverse writeback
        set ylabel  "$ X_t $"

        # Grid

        set style line 100 lt 1 lc rgb "#444444" lw 1
        set style line 101 lt 1 lc rgb "#CCCCCC" lw 1
        set style line 102 lt 1 lc rgb "#EEEEEE" lw 1

        set style line 105 lt 1 lc rgb "#444444" lw 3

        set mxtics 5
        set mytics 5
        set grid ytics mytics mxtics xtics ls 100, ls 102

        # Plotting

        set datafile separator ','
        set key autotitle columnhead

        filename = './data/Gaussian_te14_realizations.csv'

        stats filename nooutput
        n_cols = STATS_columns > 9 ? 9 : STATS_columns  # В палитре по умолчанию всего 8 цветов.

        plot [0:*] for [i=2:n_cols] filename using 1:i with lines lw 3 notitle
    \end{gnuplot}
    %\vspace{-32pt}
    \caption{Пример пучка реализаций гауссовского процесса из задачи \ref{exercise:special:gaussian_from_characteristic_function_of_Cauchy} (среднее взято за ноль).}
    \label{figure:special:gaussian_proccess_te14_realizations}
\end{figure}


\subsubsection{Винеровский процесс} \label{subsubsection:Wiener}

Винеровский процесс описывает симметричное случайное блуждание, непрерывное во времени,
и также имеет множество важных приложений.
Данный процесс часто возникает в стохастических дифференциальных уравнениях,
а также при построении других гауссовских процессов.

\begin{figure}[ht!]
    \centering
    \begin{gnuplot}[terminal=epslatex, terminaloptions={color size 16cm,10cm}]
        set xlabel  "$ t $"
        set xrange  [ 0 : * ] noreverse writeback
        set ylabel  "$ W_t $"

        # Grid

        set style line 100 lt 1 lc rgb "#444444" lw 1
        set style line 101 lt 1 lc rgb "#CCCCCC" lw 1
        set style line 102 lt 1 lc rgb "#EEEEEE" lw 1

        set style line 105 lt 1 lc rgb "#444444" lw 3

        set mxtics 5
        set mytics 5
        set grid ytics mytics mxtics xtics ls 100, ls 102

        # Plotting

        set datafile separator ','
        set key autotitle columnhead

        filename = './data/Wiener_realizations.csv'

        stats filename nooutput
        n_cols = STATS_columns > 9 ? 9 : STATS_columns  # В палитре по умолчанию всего 8 цветов.

        plot [0:*] for [i=2:n_cols] filename using 1:i with lines lw 3 notitle
    \end{gnuplot}
    %\vspace{-32pt}
    \caption{Пример пучка реализаций винеровского процесса.}
    \label{figure:special:Wiener_proccess_realizations}
\end{figure}

Неформально винеровский процесс можно определить,
введя мелкую сетку дискретного времени с шагом $ \Delta t $.
Пусть процесс стартует из нуля и на каждом очередном шаге по времени
делает скачок на некоторую случайную величину;
математическое ожидание скачка пусть будет равно нулю, а дисперсия~--- $ \Delta t $
(это сделано для того, чтобы дисперсия сечения процесса была равна прошедшему времени и,
таким образом, не зависила от выбора $ \Delta t $).
Полученные случайные блуждания при $ \Delta t \to +0 $ и описываются винеровским процессом.

    % Важные примеры случайных процессов.

\end{document}
