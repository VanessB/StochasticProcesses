\documentclass[12pt]{article}
%\usepackage[14pt]{extsizes}
%\usepackage[a5paper, textwidth=12cm, textheight=17cm, headheight=14pt]{geometry}
\usepackage[a4paper, left=2cm, right=2cm, top=2cm, bottom=2cm]{geometry}

% Ссылки.
\usepackage[unicode=true]{hyperref}

% Улучшенные сноски.
\usepackage{footmisc}

% Продвинутые формулы.
\usepackage{amsmath}

% Продвинутые математические символы.
\usepackage{amssymb}

% Кастомизируемые теоремы.
\usepackage{amsthm}
%\usepackage{thmtools}

% Русский язык.
\usepackage{cmap}
\usepackage[T2A]{fontenc}
\usepackage[utf8]{inputenc}
\usepackage[russian]{babel}

% Кастомизируемые хэдеры и футеры.
%\usepackage{fancyhdr}

% Табуляция перед первым параграфом.
%\usepackage{indentfirst}

% Нижнее подчёркивание с переносами.
\usepackage[normalem]{ulem}

% Графики gnuplot.
\usepackage[shell, subfolder, cleanup]{gnuplottex}

% Работа с плавающими объектами.
\usepackage[section]{placeins}

% Обтекаемые изображения
\usepackage{wrapfig}

% Ячейки на несколько строк.
\usepackage{multirow}
\usepackage{makecell}

% Таблица с регулируемой шириной столбцов и работающими сносками.
\usepackage{tabularx}

% Альтернативные таблицы.
%\usepackage{tabularray}

% Вращение.
\usepackage{rotating}
\usepackage{pdflscape}

% Элементы на следующей странице.
\usepackage{afterpage}

% Задачи.
\usepackage[lastexercise]{exercise}

% Сброс счётчика.
\usepackage{chngcntr}

% Графика TikZ
\usepackage{tikz}
%\usepackage{tikz-qtree}
%\usetikzlibrary{calc}
\usetikzlibrary{trees,calc,arrows.meta,positioning,decorations.pathreplacing,bending,matrix}
%\usetikzlibrary{trees}

% Таблицы Юнга.
\usepackage{ytableau}

% Enum и item на одной строке.
\usepackage[inline]{enumitem}
\setlist[1]{itemsep=0pt} % Величина разрыва между \item


    %%%%%%%%%

    % КОМАНДЫ

    %%%%%%%%%


% Математические символы и прочие дефайны.
%% Математические символы и прочие дефайны.

\def\defarr{\overset{\triangle}{\Longleftrightarrow}} % <<По определению>>
\def\defeq{\overset{\triangle}{=}}                    % <<По определению равно>>
\def\symdiff{\,\triangle\,}                           % <<Симметрическая разность>>
\def\connected{\leftrightsquigarrow}                  % Связность в графах.

\def\boolfun{\mathcal{B}} % Булевы функции.

% Математическое операторы.
\DeclareMathOperator{\diam}{\textnormal{diam}}
\DeclareMathOperator{\rad}{\textnormal{rad}}
\DeclareMathOperator*{\argmin}{\arg\min}
\DeclareMathOperator*{\argmax}{\arg\max}
\DeclareMathOperator*{\dom}{\textnormal{dom}}
\DeclareMathOperator*{\range}{\textnormal{range}}
\DeclareMathOperator*{\closure}{\textnormal{cl}}

% Математические множества.
\def\naturals{\mathbb{N}}
\def\integers{\mathbb{Z}}
\def\rationals{\mathbb{Q}}
\def\reals{\mathbb{R}}
\def\complexes{\mathbb{C}}


% Настройки пакета с упражнениями.
%% Настройки пакета xsim
\xsimsetup{path=xsim/}
%\loadxsimstyle{layouts}
\loadxsimstyle{custom}


% Перевод.
\DeclareExerciseTranslations{exercise}{Russian = Задача}
\DeclareExerciseTranslations{solution}{Russian = Решение задачи}

% Оружение.
\DeclareExerciseType{exercise}{
    exercise-env = exercise ,
    solution-env = solution ,
    exercise-name = \XSIMtranslate{exercise} ,
    exercises-name = \XSIMtranslate{exercises} ,
    solution-name = \XSIMtranslate{solution} ,
    solutions-name = \XSIMtranslate{solutions} ,
    exercise-template = runin ,
    solution-template = runin-sol ,
    exercise-heading = \normalsize \textbf ,
    solution-heading = \normalsize \textbf ,
    counter = TheoremCounter
}

\xsimsetup{solution/print=true}


% Картинки.
% Счётчики таблиц и фигур.
\counterwithin{table}{section}
\counterwithin{figure}{section}



% Специальные обозначения бкув.
%% Специальные обозначения букв.

%\def\N{\mathbb{N}}
%\def\Z{\mathbb{Z}}
%\def\Q{\mathbb{Q}}
%\def\R{\mathbb{R}}
%\def\f{\mathcal{F}}
%\def\l{\mathcal{L}}
%\def\t{\mathcal{T}}
%\def\I{\mathbb{I}}


% Теория вероятностей.
%% Теория вероятностей.
\def\proba{\mathbb{P}}
\def\setfamily{\mathcal{F}}
\def\borel{\mathcal{B}}
\def\trajectories{\mathcal{X}}
\def\indicator{\mathbb{I}}
\def\lebesgue{\mathbb{L}}
\def\iid{\textnormal{н.о.р.с.в.}}

% Моменты.
\DeclareMathOperator{\expect}{\mathbb{E}}
\DeclareMathOperator{\dispersion}{\mathbb{D}}
\newcommand{\covariance}[2]{\textnormal{cov}\left(#1, #2\right)}
\newcommand{\rvcenter}[1]{\mathring{#1}}

% Распределения.
\def\bernoulli{\textnormal{Be}}
\def\binomial{\textnormal{Bi}}
\def\poisson{\textnormal{Po}}
\def\uniform{\textnormal{U}}
\def\expdistr{\textnormal{Exp}}
\def\normal{\mathcal{N}}

% Сходимости.
\DeclareMathOperator*{\limmeansq}{\textnormal{l.i.m.}}
\newcommand{\converges}[1]{\overset{#1}{\longrightarrow}}
\def\convmeansq{\converges{\textnormal{с.к.}}}
\def\convalmost{\converges{\textnormal{п.н.}}}
\def\convproba{\converges{\proba}}
\def\convdistr{\converges{d}}
\def\convnorm{\converges{\| \cdot \|}}

\def\almosteq{\overset{\textnormal{п.н.}}{=}}
\def\distreq{\overset{d}{=}}


% Разделы без номеров.
% Глава без номера.
\newcommand{\silentchapter}[1]{
    \chapter*{#1}
    \markboth{\MakeUppercase{#1}}{#1}
    \addcontentsline{toc}{chapter}{#1}
}

% Секция без номера.
\newcommand{\silentsection}[1]{
    \section*{#1}
    \markboth{\MakeUppercase{#1}}{#1}
    \addcontentsline{toc}{section}{#1}
}

% Подсекция без номера.
\newcommand{\silentsubsection}[1]{
    \subsection*{#1}
    \markboth{\MakeUppercase{#1}}{#1}
    \addcontentsline{toc}{subsection}{#1}
}



% Настройки пакета asmthm.
%% Настройки пакета asmthm.

% Счётчик теорем и прочего.
\newcounter{TheoremCounter}
\counterwithin{TheoremCounter}{section}
%\counterwithin*{TheoremCounter}{subsection}

% Теоремы, определения, замечания и так далее.
\newtheorem{theorem}[TheoremCounter]{Теорема}
\newtheorem{lemma}[TheoremCounter]{Лемма}
\newtheorem{corollary}[TheoremCounter]{Следствие}
\newtheorem{definition}[TheoremCounter]{Определение}
\newtheorem{remark}[TheoremCounter]{Замечание}
\newtheorem{statement}[TheoremCounter]{Утверждение}
\newtheorem{problem}[TheoremCounter]{Задача}
\newtheorem{example}[TheoremCounter]{Пример}



% Правила вывода.
\newcommand{\typerule}[2]{%
    \begin{tabular}{c}
    $#1$ \\
    \hline
    %\midrule
    $#2$
    \end{tabular}%
}


% Inline item.
\makeatletter
\newcommand{\inlineitem}[1][]
{%
    \ifnum\enit@type=\tw@
        {\descriptionlabel{#1}}
        \hspace{\labelsep}%
    \else
        \ifnum\enit@type=\z@
        \refstepcounter{\@listctr}\fi
        \quad\@itemlabel\hspace{\labelsep}%
\fi}
\makeatother


% Выделение в определении
%\newcommand{\defemph}[1]{\textbf{\textit{#1}}}
\DeclareTextFontCommand{\defemph}{\bfseries\em}

% Нумерация русскими буквами.
\renewcommand{\alph}[1]{\asbuk{#1}}

% Пространство между плавающими объектами.
\setlength{\floatsep}{20pt}




\title{Случайные процессы: семинары}
\author{Бутаков~И.\,Д.}
\date{2023}

%\pagenumbering{gobble}

\begin{document}

\numberwithin{equation}{section}

\maketitle

\tableofcontents

\silentsection{Предисловие} \label{sec:intro}

Перед вами сборник всех семинаров по случайным процессам за авторством Бутакова\,И.\,Д.
Автор выражает благодарность Останину Павлу Антоновичу и Широбокову Максиму Геннадьевичу за предоставленные материалы.

\silentsubsection{Используемые обозначения}

\begin{center}
    \begin{tabularx}{\textwidth}{cl}
        $ \defarr $                      & <<\ldots по определению тогда и только тогда, когда \ldots>> \\
        $ \defeq $                       & <<\ldots по определению равно \ldots>> \\
        \rule{0pt}{16pt}%
        $ (\Omega, \setfamily, \proba) $ & \makecell[l]{вероятностное пространство ($ \Omega $~--- множество исходов, $ \setfamily $~--- $ \sigma $-алгебра, \\ $ \proba $~--- вероятностная мера).} \\
        $ \borel(A) $, $ \borel_A $      & \makecell[l]{Борелевская $ \sigma $-алгебра, определённая на множестве $ A $ (если $ A $ не указано, \\ по умолчанию предполагается $ A = \reals $).} \\
        $ \indicator_A $                 & индикаторная функция множества $ A $. \\
        $ \expect X $                    & математическое ожидание случайной величины $ X $. \\
        $ \dispersion X $                & дисперсия случайной величины $ X $. \\
        $ \rvcenter X $                  & <<центрированная>> случайная величина: $ \rvcenter X = X - \expect X $. \\
        \rule{0pt}{16pt}%
        $ \bernoulli(p) $                & распределение Бернулли с параметром $ p $. \\
        $ \binomial(n, p) $              & биномиальное распределение с параметрами $ n $ и $ p $. \\
        $ \poisson(\lambda) $            & распределение Пуассона с интенсивностью $ \lambda $. \\
        $ \uniform(A)$, $ \uniform_A $   & равномерное распределение на множестве $ A $. \\
        $ \expdistr(\lambda) $           & показательное распределение с параметром $ \lambda $ (интенсивность). \\
        $ \normal(\mu, \sigma^2) $       & нормальное распределение со средним $ \mu $ и дисперсией $ \sigma^2 $. \\
        \rule{0pt}{16pt}%
        $ \limmeansq $                   & предел в среднем квадратичном (англ. \textit{limit in means}). \\
        $ \convmeansq $                  & сходимость в среднем квадратичном. \\
        $ \convalmost $                  & сходимость <<почти наверное>>. \\
        $ \convproba $                   & сходимость по вероятности. \\
        $ \convdistr $                   & сходимость по распределению. \\
        $ \almosteq $                    & равенство <<почти наверное>>. \\
        $ \distreq $                     & равенство по распределению.
    \end{tabularx}
\end{center}
          % Введение.
\section{Основные сведения} \label{sec:basics}

Случайные процессы~--- математические объекты,
построенные с использованием теории вероятностей для исследования и моделирования реальных явлений,
растянутых во времени и имеющих стохастическую (случайную) природу.

\begin{definition}
    \label{definition:basics:stochastic_process}
    Пусть задано вероятностное пространство $ (\Omega, \setfamily, \proba) $ и множество $ T \subseteq \mathbb{R} $.
    Функция $ X: \Omega \times T \to \mathbb{R} $ называется \defemph{случайным процессом},
    если $ \forall t \in T $ функция $ X(\cdot, t) \equiv X_t: \Omega \to \mathbb{R} $ измерима
    (то есть является случайной величиной).
\end{definition}

Случайный процесс можно трактовать как семейство случайных величин, параметризованное $ t \in T $.
Параметр $ t $ обычно интерпретируется как время.
Если $ T $ состоит из одного элемента, случайный процесс является обычной случайной величиной,
если $ T $ конечно~--- случайным вектором.
Параметр $ \omega $, как и при описании случайных величин, часто опускается.

\begin{definition}
    \label{definition:basics:stochastic_process_slice}
    При фиксированном $ t_0 \in T $ случайная величина $ X_{t_0} $ называется \defemph{сечением случайного процесса} $ X $.
\end{definition}

\begin{definition}
    \label{definition:basics:stochastic_process_realization}
    При фиксированном $ \omega_0 \in \Omega $ функция $ X(\omega_0, \cdot ) $ называется \defemph{реализацией случайного процесса} $ X $.
\end{definition}

Также случайный процесс можно считать особой случайной величиной, принимающей значения в пространстве функций;
при такой интерпретации, однако, отдельных усилий стоит определить, что такое вероятностное распределение на функциях.
В рамках семинаров данный вопрос освещаться со всей полнотой и строгостью не будет,
поэтому приведём из этой области лишь основные факты и определения,
требующиеся для работы со случайными процессами.

Рассмотрим произвольный случайный процесс $ X $.
В силу единства вероятностного пространства,
любой вектор вида $ (X_{t_1}, \ldots, X_{t_n}) $ (где $ t_i \in T $) является случайным вектором.

\begin{definition}
    \label{definition:basics:finite_distribution}
    Вероятностное распределение вектора вида $ (X_{t_1}, \ldots, X_{t_n}) $ называется \defemph{конечномерным распределением случайного процесса} $ X $.
    Его функция распределения обозначается как $ F_X(x_1, \ldots, x_n; t_1, \ldots, t_n) $.
\end{definition}

Функции распределений векторов, составленных из сечений случайного процесса,
обладают всеми известными вам свойствами функций распределений случайных векторов,
а также ещё двумя дополнительными свойствами:

\begin{statement}
    \label{statement:basics:finite_distribution_properties}
    Функции конечномерных распределений случайного процесса $ X $ обладают следующими свойствами:
    \begin{enumerate}
        \item
            \defemph{(условие симметрии)}
            Для любой перестановки $ k_i $ выполнено равенство
            \[
                F_X(x_1, \ldots, x_n; t_1, \ldots, t_n) = F_X(x_{k_1}, \ldots, x_{k_n}; t_{k_1}, \ldots, t_{k_n})
            \]
        \item
            \defemph{(условие согласованности)}
            Для любого индекса $ k \in \{1, \ldots, n\} $ выполнено
            \[
                \lim_{x_k \to +\infty} F_X(x_1, \ldots, x_n; t_1, \ldots, t_n) = F(x_1, \ldots, x_{k-1}, x_{k+1}, \ldots x_n; t_1, \ldots, t_{k-1}, t_{k+1}, \ldots, t_n)
            \]
    \end{enumerate}
\end{statement}

\begin{theorem}[Колмогорова]
    \label{theorem:basics:finite_distributions_family_define_stochastic_process}
    Пусть имеется семейство распределений случайных векторов,
    удовлетворяющее всем свойствам из утверждения \ref{statement:basics:finite_distribution_properties}.
    Тогда существует вероятностное пространство и заданный на нём случайный процесс,
    семейство конечномерных распределений которого совпадает с данным.
\end{theorem}

Таким образом, случайный процесс можно задавать семейством его конечномерных распределений.
На данном этапе читателю должно стать понятно,
как можно задавать вероятностное распределение на множестве функций
(ответ~--- при помощи специальных семейств конечномерных распределений).

\begin{Exercise}[counter=SecExercise, label={exercise:basics:rv_plus_t}]
    \noindent
    Пусть $ \eta $~--- случайная величина с функцией распределения $ F_\eta $.
    Найти все конечномерные распределения случайного процесса $ X_t = \eta + t $.
\end{Exercise}

\begin{Answer}
    \noindent
    Одномерная функция распределения:
    \[
        F_X(x; t) = \proba \{ \omega \in \Omega \mid X_t < x \} = \proba \{ \eta < x - t \} = F_\eta(x - t)
    \]
    Конечномерная функция распределения:
    \begin{multline*}
        F_X(x_1, \ldots, x_n; t_1, \ldots, t_n) = \proba \bigcap_{i = 1}^n \{ X_{t_i} < x_i \} = \proba \bigcap_{i = 1}^n \{ \eta < x_i - t_i \} = \\
        = \proba \left\{ \eta < \min_i \{x_i - t_i\} \right\} = F_\eta \left(\min_i \{x_i - t_i \} \right)
    \end{multline*}
\end{Answer}

\begin{Exercise}[counter=SecExercise, label={exercise:basics:random_point_om_segment}]
    \noindent
    Пусть дана случайная величина $ \eta \sim \uniform_{[0;1]} $.
    Определим случайный процесс $ X_t = \indicator_{(-\infty; \eta]}(t) $.
    Найдите вид реализаций процесса, его одномерные и двумерные распределения.
\end{Exercise}

\begin{Answer}
    \noindent
    Реализация процесса~--- функция, равная единице при $ t \leqslant \eta $ и нулю при $ t > \eta $.
    Одномерная функция распределения:
    \[
        F_X(x;t) = \proba \{ X_t < x \} = \proba \{ \indicator_{(-\infty; \eta]}(t) < x \} =
        \begin{cases}
            0, &\quad x \leqslant 0 \\
            \proba \{\eta < t\}, &\quad 0 < x \leqslant 1 \\
            1, &\quad x > 1
        \end{cases},
    \]
    \[
        \textnormal{где} \quad
        \proba \{\eta < t\} = F_\eta(t) =
        \begin{cases}
            0, &\quad t \leqslant 0 \\
            t, &\quad 0 < t \leqslant 1 \\
            1, &\quad t > 1
        \end{cases}
    \]
    Двумерная функция распределения:
    \[
        F_X(x_1, x_2; t_1, t_2) = \proba \left( \{X_{t_1} < x_1\} \cap \{X_{t_2} < x_2\} \right)
    \]
    Аналогично одномерной функции распределения,
    \begin{enumerate}
        \item
            Если $ x_1 \leqslant 0 $ или $ x_2 \leqslant 0 $, $ F_X(x_1, x_2; t_1, t_2) = 0 $.
        \item
            Если $ x_1 > 1 $ и $ x_2 > 1 $, $ F_X(x_1, x_2; t_1, t_2) = 1 $.
        \item
            Если $ 0 < x_1 \leqslant 1 $ и $ x_2 > 1 $, $ F_X(x_1, x_2; t_1, t_2) = F_X(x_1; t_1) $.
            Аналогично симметричный случай.
        \item
            Если $ 0 < x_1, x_2 \leqslant 1 $,
            \[
                F(x_1, x_2; t_1, t_2) = \proba \left( \{\eta < t_1\} \cap \{\eta < t_2\} \right) =
                \proba \left\{ \eta < \min\{t_1, t_2\} \right\} = F_\eta\left( \min\{t_1, t_2\} \right)
                %\begin{cases}
                %    0, &\quad \min\{t_1, t_2\} < 0 \\
                %    \min\{t_1, t_2\}, &\quad 0 \leqslant \min\{t_1, t_2\} < 1 \\
                %    1, &\quad \min\{t_1, t_2\} > 1
                %\end{cases}
            \]
    \end{enumerate}
\end{Answer}

Существование различных случайных процессов с одними и теми же
вероятностными свойствами приводит к желанию (а иногда и необходимости)
в некотором смысле отождествлять процессы,
у которых конечномерные распределения совпадают.

\begin{definition}
    \label{definition:basics:modification}
    Пусть $ X $ и $ Y $~--- два случайных процесса,
    определённые на одном и том же вероятностном пространстве $ (\Omega, \setfamily, \proba) $ и множестве $ T $.
    Данные процессы называются \defemph{стохастически эквивалентными} в случае равенства почти наверное их реализаций в любой выбранный момент,
    то есть
    \[
        \forall t \in T \quad \proba \{ \omega \in \Omega \mid X(\omega, t) = Y(\omega, t) \} = 1
    \]
    В этом случае $ Y $ называют \defemph{модификацией} процесса $ Y $ (и наоборот).
\end{definition}

\begin{statement}
    \label{statement:basics:finite_distributions_of_modifications}
    Стохастически эквивалентные случайные процессы имеют одинаковое семейство конечномерных распределений.
\end{statement}

Например, такое отождествление полезно для осмысленного определения непрерывного случайного процесса:
\begin{definition}
    \label{definition:basics:continious_stochastic_process}
    Случайный процесс называется \defemph{непрерывным} в случае,
    если существует его модификациея с непрерывными реализациями.
\end{definition}

\begin{Exercise}[counter=SecExercise, label={exercise:basics:continious_stochastic_process}]
    \noindent
    Пусть $ \eta \sim \uniform_{[0;1]} $.
    Определим случайный процесс $ X_t = \indicator_{\{\eta\}}(t) $
    (то есть $ X_t = 1 $ в том и только в том случае, когда $ \eta = t $, и равен $ 0 $ иначе).
    Является ли $ X_t $ непрерывным процессом?
\end{Exercise}

\begin{Answer}
    \noindent
    Да, является.
    Процесс $ Y_t \equiv 0 $ является его модификацией.
\end{Answer}


При исследовании случайных процессов также бывает полезно рассматривать их моменты,
дающие некоторое представление об усреднённом поведении процесса.
В отличие от случайных величин, любые моменты случайного процесса также зависят от времени.

\begin{definition}
    \label{definition:basics:mean_function}
    Если $ \forall t \in T $ существует $ \expect X_t $,
    то функция $ m_X(t) = \expect X_t $ определена и называется \defemph{функцией среднего}.
\end{definition}

Аналогично вводятся функции любых других моментов случайной величины $ X_t $.
При работе со случайными процессами нас также будут интересовать моменты,
<<разнесённые во времени>>.

\begin{definition}
    \label{definition:basics:second_order_moment_functions}
    Если $ \forall t_1, t_2 \in T $ существует $ \expect X_{t_1} X_{t_2} $,
    то функции $ K_X(t_1, t_2) = \expect X_{t_1} X_{t_2} $ и $ R_X(t_1, t_2) = \expect \rvcenter X_{t_1} \rvcenter X_{t_2} $
    определены и называются, соответственно, \defemph{ковариационной} и \defemph{корреляционной функциями}.%
    \footnote{Данные обозначения не являются общепринятыми, а также несколько контринтуитивны; при чтении сторонних источников будьте внимательны.}
\end{definition}

\begin{statement}
    \label{statement:basics:correlation_and_covariation_connection}
    Функции $ K_X(t_1, t_2) $ и $ R_X(t_1, t_2) $ одновременно либо определены, либо не определены,
    причём в первом случае функция $ m_X(t) $ определена и $ R_X(t_1, t_2) = K_X(t_1, t_2) - m_X(t_1) m_X(t_2) $.
\end{statement}

\begin{proof}
    Следует из свойств моментов.
\end{proof}

\begin{Exercise}[counter=SecExercise, label={exercise:basics:random_point_om_segment_moments}]
    \noindent
    Найти корреляционную функцию случайного процесса из задачи \ref{exercise:basics:random_point_om_segment}.
\end{Exercise}

\begin{Answer}
    \noindent
    Для любого $ t_0 $ случайная величина $ X_{t_0} $ может принимать только два значения~--- $ 0 $ или $ 1 $;
    это бернуллиевская случайная величина.
    Найдём параметр её распределения:
    \[
        \proba \{X_t = 1\} = \proba \{t < \eta\} = 1 - F_\eta(t)
    \]
    Следовательно, $ m_X(t) = \expect X_t = 1 - F_\eta(t) $.
    Далее,
    \begin{multline*}
        K_X(t_1, t_2) = \expect X_{t_1} X_{t_2} = 1 \cdot \proba \left( \{X_{t_1} = 1\} \cap \{X_{t_2} = 1\} \right) = \\
        = \proba \left( \{t_1 < \eta\} \cap \{t_2 < \eta\} \right) = 1 - F_\eta\left( \max \{t_1, t_2\} \right)
    \end{multline*}
    Наконец,
    \begin{multline*}
        R_X(t_1, t_2) = 1 - F_\eta\left( \max \{t_1, t_2\} \right) - (1 - F_\eta(t_1)) \cdot (1 - F_\eta(t_2)) = \\
        = F_\eta(t_1) \cdot F_\eta(t_2) - F_\eta(t_1) - F_\eta(t_2) - F_\eta(\max\{t_1, t_2\})
    \end{multline*}
\end{Answer}

\begin{Exercise}[counter=SecExercise, label={exercise:basics:cosine_stochastic_process}]
    \noindent
    Пусть $ \xi \sim \normal(0, 1) $ и $ \eta \sim U_{[-\pi; \pi]} $~--- независимые случайные переменные.
    Определим случайный процесс $ X $ следующим образом: $ X_t = \xi \cdot \cos(t + \eta) $, где $ t \in \reals $.
    Найдите функцию среднего и корреляционную функцию процесса.
\end{Exercise}

\begin{Answer}
    \noindent
    Поскольку $ \xi $ и $ \eta $ независимы,
    \[
        m_X(t) = \expect X_t = \expect \xi \cdot \expect \cos(t + \eta) = 0 \cdot \ldots = 0
    \]
    \begin{multline*}
        R_X(t_1, t_2) = K_X(t_1, t_2) - 0 = \expect X_{t_1} X_{t_2} = \expect \xi^2 \cdot \expect \left( \cos(t_1 + \eta) \cdot \cos(t_2 + \eta) \right) = \\
        = 1 \cdot \frac{1}{2} \expect \left( \cos(t_1 - t_2) + \cos(t_1 + t_2 + 2 \eta) \right) = \frac{1}{2} \cos(t_1 - t_2)
    \end{multline*}
\end{Answer}

\begin{Exercise}[counter=SecExercise, label={exercise:basics:cos_and_sin}]
    \noindent
    Пусть $ U $, $ V $ и $ W $~--- независимые в совокупности случайные величины.
    Известно, что $ U $ и $ V $ обладают нулевым матожиданием и дисперсией $ D $,
    а $ W $ распределена с плотностью
    \[
        \rho_W(w) = \frac{2 \lambda}{\pi} \cdot \frac{\indicator_{[0; +\infty)}(w)}{\lambda^2 + w^2}, \quad \lambda > 0
    \]
    Определим случайный процесс $ X_t = U \cos(W t) + V \sin(W t) $.
    Вычислите функцию среднего и корреляционную функцию.
\end{Exercise}

\begin{Answer}
    \noindent
    Поскольку $ U $, $ V $ и $ W $ независимы в совокупоности,
    \[
        m_X(t) = \expect X_t = \expect U \cdot \expect \cos(W t) + \expect V \cdot \expect \sin(W t) = 0 \cdot \ldots + 0 \cdot \ldots = 0
    \]
    Корреляционную функцию удобно искать с помощью формулы полной вероятности в непрерывном случае:
    \[
        R_X(t_1, t_2) = \expect \left( \expect(X_{t_1} X_{t_2} \mid W = w) \right) = \int\limits_\reals \underbrace{\expect(X_{t_1} X_{t_2} \mid W = w)}_{\defeq R(t_1, t_2 \mid w)} \cdot \rho_W(w) \, dw
    \]
    В силу нулевого матожидания,
    \begin{multline*}
        R_X(t_1, t_2) = \expect \left( (U \cos(w t_1) + V \sin(w t_1)) \cdot (U \cos(w t_2) + V \sin(w t_2)) \right) = \\
        = \expect (U^2) \cdot \cos(w t_1) + 2 \cdot \underbrace{\expect U \expect V}_{0} \cdot \ldots + \expect (V^2) \cdot \sin(w t_1) \sin(w t_2) = D \cos(w(t_1 - t_2))
    \end{multline*}
    Наконец,
    \[
        R_X(t_1, t_2) = \int\limits_0^{+\infty} D \cos(w(t_1 - t_2)) \cdot \frac{2 \lambda}{\pi} \frac{1}{\lambda^2 + w^2} \, dw = D e^{-\lambda |t_1 - t_2|}
    \]
    Здесь использовалось значение интеграла Лапласа:
    \[
        \int\limits_0^\infty \frac{\cos(\alpha x)}{1 + x^2} \, dx = \frac{\pi}{2} e^{-|\alpha|}
    \]
\end{Answer}
     % Общие сведения.
\section{Важные примеры случайных процессов} \label{section:special}

В этом разделе речь пойдёт о нескольких процессах особого вида,
наиболее часто встречающихся при исследовании реальных явлений.
Зачастую такие процессы именные.
На их примере мы продолжим практиковаться в решении задач,
а также введём несколько новых теоретических понятий.


\subsection{Пуассоновский процесс} \label{subsection:Poisson}

Данный процесс встречается в реальной жизни довольно часто;
он описывает поток случайных событий, которые регистрируются с некоторой постоянной <<интенсивностью>>.
Например, речь может идти о регистрации космических частиц, о кликах по ссылке,
о запросах к серверу, о проезжающих по магистрали автомобилях.
Дадим формальное определение.

\begin{definition}
    \label{definition:special:independent_deltas}
    Случайный процесс $ X $ называется \defemph{процессом с независимыми приращениями},
    если $ \forall n \in \naturals \;\, \forall \{t_i\}_{i=1}^n \subseteq T $ случайные величины
    $ X_{t_n} - X_{t_{n-1}}, \ldots, X_{t_2} - X_{t_1}, X_{t_1} $
    независимы в совокупности.
\end{definition}

\begin{definition}
    \label{definition:special:Poisson_process}
    \defemph{Пуассоновским процессом с интенсивностью $ \lambda > 0 $} называется случайный процесс $ K\colon \Omega \times [0; +\infty) \to \naturals $ такой, что
    \begin{enumerate}
        \item
            $ K_0 \almosteq 0 $.
        \item
            $ K $~--- процесс с независимыми приращениями.
        \item
            $ K_t - K_s \sim \poisson\left( \lambda \cdot (t - s) \right) $ (при $ t > s \geqslant 0 $).
    \end{enumerate}
\end{definition}

Это одно из эквивалентных определений, пуассоновский процесс можно определить и иначе:

\begin{theorem}[Явная конструкция пуассоновского процесса]
    \label{theorem:special:Poisson_process_explicit_definition}
    Пусть $ \xi_1, \ldots, \xi_k, \ldots \sim \expdistr(\lambda) $ и независимы в совокупности,
    $ S_n = \xi_1 + \ldots + \xi_n $.
    Тогда процесс $ X_t = \sup \{ n \mid S_n \leqslant t \} $ есть пуассоновский процесс с интенсивностью $ \lambda $.
\end{theorem}

Процесс $ X_t $, построенный по случайным величинам $ \xi_k $ способом, указанным в теореме,
называется \defemph{процессом восстановления} и отвечает следующей модели:
в нулевой момент включается прибор, который работает время $ \xi_1 $, после чего ломается.
Одновременно с поломкой включается следующий прибор, который работает случайное время $ \xi_2 $, и так далее.
Величина $ X_t $ отражает количество приборов, введённых в эксплуатацию к моменту $ t $.

\begin{statement}
    \label{statement:special:Poisson_process_properties}
    Пуассоновский процесс обладает следующими свойствами:
    \begin{enumerate}
        \item
            Реализации пуассоновского процесса~--- кусочно-постоянные неубывающие функции со значениями в $ \naturals $.
        \item
            С вероятностью $ 1 $ все скачки пуассоновского процесса равны единице.
        \item
            Время, когда произошёл $ n $-ый скачёк (обозначим его $ \tau_n $) имеет $ \Gamma(n, 1/\lambda) $-распределение:
            \[
                \rho_{\tau_n}(t) = \frac{\lambda^n x^{n-1}}{(n-1)!} e^{-\lambda t} \cdot \indicator_{[0;+\infty)}(t)
            \]
        \item
            Случайные величины $ \{\tau_{n} - \tau_{n-1}\}_{n \in \naturals} $ распределены экспоненциально с параметром $ \lambda $ и независимы.
        \item
            Число событий за конечный период времени конечно с вероятностью $ 1 $.
        \item
            Число событий $ K_{t+h} - K_t $ на промежутке $ (t; t+h] $ зависит лишь от длины промежутка $ h $:
            $ \proba \{ K_{t + h} - K_t = k \} = p(h, k) $
        \item
            Вероятность более чем одного скачка на полуинтервале $ (t; t + h] $ есть $ o(h) $,
            то есть $ \displaystyle \lim_{h \to +0} \proba \{ K_{t+h} - K_t > 1 \} = 0 $.
        \item
            Для коротких полуинтервалов $ (t; t+h] $ вероятность того, что на них произойдёт хотя бы один скачок,
            убывает линейно с уменьшением $ h $: $ \proba \{ K_{t+h} - K_t > 0 \} = 1 - e^{-\lambda h} = \lambda h + o(h) $ при $ h \to 0 $.
        \item
            Из определения распределения Пуассона:
            \[
                \proba\{K_t = k\} = \frac{(\lambda t)^k}{k!} e^{-\lambda}
            \]
    \end{enumerate}
\end{statement}

Наконец, приведём ещё одно из альтернативных определений пуассоновского процесса:

\begin{statement}
    \label{statement:special:Poisson_process_alternative_definition}
    Случайный процесс $ K\colon \Omega \times T \to \naturals $ является пуассоновским тогда и только тогда, когда он удовлетворяет следующим свойствам:
    \begin{enumerate}
        \item
            \defemph{(стационарность приращений)}
            $ \proba \{ K_{t + h} - K_t = k \} = p(h, k) $
        \item
            \defemph{(отсутствие последействия)}
            Приращения процесса независимы.
        \item
            \defemph{(ординарность)}
            $ \proba \{ K_{t + h} - K_t > 1 \} \in o(h) $
    \end{enumerate}
\end{statement}
    % Важные примеры случайных процессов.

\end{document}
