\section{Основные сведения} \label{section:basics}

Случайные процессы~--- математические объекты,
построенные с использованием теории вероятностей для исследования и моделирования реальных явлений,
растянутых во времени и имеющих стохастическую (случайную) природу.

\begin{definition}
    \label{definition:basics:stochastic_process}
    Пусть задано вероятностное пространство $ (\Omega, \setfamily, \proba) $ и множество $ T \subseteq \mathbb{R} $.
    Функция $ X\colon \Omega \times T \to \mathbb{R} $ называется \defemph{случайным процессом},
    если $ \forall t \in T $ функция $ X(\blankarg, t) \equiv X_t\colon \Omega \to \mathbb{R} $ измерима
    (то есть является случайной величиной).
\end{definition}

Случайный процесс можно трактовать как семейство случайных величин, параметризованное $ t \in T $.
Параметр $ t $ обычно интерпретируется как время.
Если $ T $ состоит из одного элемента, случайный процесс является обычной случайной величиной,
если $ T $ конечно~--- случайным вектором.
Если $ T $ счётно, говорят о случайном процессе с дискретным временем.
Параметр $ \omega $, как и при описании случайных величин, часто опускается.

\begin{definition}
    \label{definition:basics:stochastic_process_slice}
    При фиксированном $ t_0 \in T $ случайная величина $ X_{t_0} $ называется \defemph{сечением случайного процесса} $ X $.
\end{definition}

\begin{definition}
    \label{definition:basics:stochastic_process_realization}
    При фиксированном $ \omega_0 \in \Omega $ функция $ X(\omega_0, \blankarg) $ называется \defemph{реализацией (траекторией) случайного процесса} $ X $.
\end{definition}

Также случайный процесс можно считать особой случайной величиной, принимающей значения в пространстве функций;
при такой интерпретации, однако, отдельных усилий стоит определить, что такое вероятностное распределение на функциях.
В рамках семинаров данный вопрос освещаться со всей полнотой и строгостью не будет,
поэтому приведём из этой области лишь основные факты и определения,
требующиеся для работы со случайными процессами.

Рассмотрим произвольный случайный процесс $ X $.
В силу единства вероятностного пространства,
любой вектор вида $ (X_{t_1}, \ldots, X_{t_n}) $ (где $ t_i \in T $) является случайным вектором.

\begin{definition}
    \label{definition:basics:finite_distribution}
    Вероятностное распределение вектора вида $ (X_{t_1}, \ldots, X_{t_n}) $ называется \defemph{конечномерным распределением случайного процесса} $ X $.
    Его функция распределения обозначается как $ F_X(x_1, \ldots, x_n; t_1, \ldots, t_n) $.
\end{definition}

Функции распределений векторов, составленных из сечений случайного процесса,
обладают всеми известными вам свойствами функций распределений случайных векторов,
а также ещё двумя дополнительными свойствами:

\begin{statement}
    \label{statement:basics:finite_distribution_properties}
    Функции конечномерных распределений случайного процесса $ X $ обладают следующими свойствами:
    \begin{enumerate}
        \item
            \defemph{(условие симметрии)}
            Для любой перестановки $ k_i $ выполнено равенство
            \[
                F_X(x_1, \ldots, x_n; t_1, \ldots, t_n) = F_X(x_{k_1}, \ldots, x_{k_n}; t_{k_1}, \ldots, t_{k_n})
            \]
        \item
            \defemph{(условие согласованности)}
            Для любого индекса $ k \in \{1, \ldots, n\} $ выполнено
            \[
                \lim_{x_k \to +\infty} F_X(x_1, \ldots, x_n; t_1, \ldots, t_n) = F(x_1, \ldots, x_{k-1}, x_{k+1}, \ldots x_n; t_1, \ldots, t_{k-1}, t_{k+1}, \ldots, t_n)
            \]
    \end{enumerate}
\end{statement}

\begin{theorem}[Колмогорова]
    \label{theorem:basics:finite_distributions_family_define_stochastic_process}
    Пусть имеется семейство распределений случайных векторов,
    удовлетворяющее всем свойствам из утверждения \ref{statement:basics:finite_distribution_properties}.
    Тогда существует вероятностное пространство и заданный на нём случайный процесс,
    семейство конечномерных распределений которого совпадает с данным.
\end{theorem}

Таким образом, случайный процесс можно задавать семейством его конечномерных распределений.
На данном этапе читателю должно стать понятно,
как можно задавать вероятностное распределение на множестве функций
(ответ~--- при помощи специальных семейств конечномерных распределений).

\begin{Exercise}[counter=SecExercise, label={exercise:basics:rv_plus_t}]
    \noindent
    Пусть $ \eta $~--- случайная величина с функцией распределения $ F_\eta $.
    Найти все конечномерные распределения случайного процесса $ X_t = \eta + t $.
\end{Exercise}

\begin{Answer}
    \noindent
    Одномерная функция распределения:
    \[
        F_X(x; t) = \proba \{ \omega \in \Omega \mid X_t < x \} = \proba \{ \eta < x - t \} = F_\eta(x - t)
    \]
    Конечномерная функция распределения:
    \begin{multline*}
        F_X(x_1, \ldots, x_n; t_1, \ldots, t_n) = \proba \bigcap_{i = 1}^n \{ X_{t_i} < x_i \} = \proba \bigcap_{i = 1}^n \{ \eta < x_i - t_i \} = \\
        = \proba \left\{ \eta < \min_i \{x_i - t_i\} \right\} = F_\eta \left(\min_i \{x_i - t_i \} \right)
    \end{multline*}
\end{Answer}

\begin{Exercise}[counter=SecExercise, label={exercise:basics:random_point_om_segment}]
    \noindent
    Пусть дана случайная величина $ \eta \sim \uniform_{[0;1]} $.
    Определим случайный процесс $ X_t = \indicator_{(-\infty; \eta]}(t) $.
    Найдите вид реализаций процесса, его одномерные и двумерные распределения.
\end{Exercise}

\begin{Answer}
    \noindent
    Реализация процесса~--- функция, равная единице при $ t \leqslant \eta $ и нулю при $ t > \eta $, см. рис.~\ref{figure:basics:random_point_on_segment}.
    Одномерная функция распределения:
    \[
        F_X(x;t) = \proba \{ X_t < x \} = \proba \{ \indicator_{(-\infty; \eta]}(t) < x \} =
        \begin{cases}
            0, &\quad x \leqslant 0 \\
            \proba \{\eta < t\}, &\quad 0 < x \leqslant 1 \\
            1, &\quad x > 1
        \end{cases},
    \]
    \[
        \textnormal{где} \quad
        \proba \{\eta < t\} = F_\eta(t) =
        \begin{cases}
            0, &\quad t \leqslant 0 \\
            t, &\quad 0 < t \leqslant 1 \\
            1, &\quad t > 1
        \end{cases}
    \]
    Двумерная функция распределения:
    \[
        F_X(x_1, x_2; t_1, t_2) = \proba \left( \{X_{t_1} < x_1\} \cap \{X_{t_2} < x_2\} \right)
    \]
    Аналогично одномерной функции распределения,
    \begin{enumerate}
        \item
            Если $ x_1 \leqslant 0 $ или $ x_2 \leqslant 0 $, $ F_X(x_1, x_2; t_1, t_2) = 0 $.
        \item
            Если $ x_1 > 1 $ и $ x_2 > 1 $, $ F_X(x_1, x_2; t_1, t_2) = 1 $.
        \item
            Если $ 0 < x_1 \leqslant 1 $ и $ x_2 > 1 $, $ F_X(x_1, x_2; t_1, t_2) = F_X(x_1; t_1) $.
            Аналогично симметричный случай.
        \item
            Если $ 0 < x_1, x_2 \leqslant 1 $,
            \[
                F(x_1, x_2; t_1, t_2) = \proba \left( \{\eta < t_1\} \cap \{\eta < t_2\} \right) =
                \proba \left\{ \eta < \min\{t_1, t_2\} \right\} = F_\eta\left( \min\{t_1, t_2\} \right)
                %\begin{cases}
                %    0, &\quad \min\{t_1, t_2\} < 0 \\
                %    \min\{t_1, t_2\}, &\quad 0 \leqslant \min\{t_1, t_2\} < 1 \\
                %    1, &\quad \min\{t_1, t_2\} > 1
                %\end{cases}
            \]
    \end{enumerate}
\end{Answer}

\begin{figure}[ht!]
    \centering
    \begin{gnuplot}[terminal=epslatex, terminaloptions={color size 12cm,8cm}]
        set xlabel  "$ t $"
        set xrange  [ 0 : 1 ] noreverse writeback
        set ylabel  "$ X_t $"
        set yrange  [ -0.1 : 1.1 ] noreverse writeback

        # Functions

        eta = 0.42
        part1(x) = (x <= eta ? 1.0 : 1/0)
        part2(x) = (x >= eta ? 0.0 : 1/0)
        #indicator(x) = (x <= eta ? 1.0 : 0.0)
        #set samples 1000

        # Grid

        set style line 110 lt 1 lc rgb "#EE5555" lw 8

        set style line 100 lt 1 lc rgb "#444444" lw 1
        set style line 101 lt 1 lc rgb "#CCCCCC" lw 1
        set style line 102 lt 1 lc rgb "#EEEEEE" lw 1

        set style line 105 lt 1 lc rgb "#444444" lw 3

        set grid ytics mytics mxtics xtics ls 100, ls 101

        # Arrows

        unset border
        set arrow from graph 0.0,0.083333 to graph 1.05,0.083333 size screen 0.025,15,60 filled ls 105
        set arrow from graph 0.0,0.0 to graph 0.0,1.05 size screen 0.025,15,60 filled ls 105

        # Plotting

        set key noautotitle
        plot [0:1] part1(x) notitle ls 110, part2(x) t "$ X(\\omega_0, t) $" ls 110
    \end{gnuplot}
    %\vspace{-32pt}
    \caption{График одной из реализаций случайного процесса из задачи \ref{exercise:basics:random_point_om_segment}.}
    \label{figure:basics:random_point_on_segment}
\end{figure}

Задавать процессы на вероятностных пространствах удобно также при помощи \defemph{выборочного (вторичного) пространства}.

\begin{definition}
    \label{definition:basics:sampling_space}
    Рассмотрим определённый на вероятностном пространстве $ (\Omega, \setfamily, \proba) $ случайный процесс $ X \colon \Omega \times T \to \reals $.
    Пусть $ \trajectories $~--- пространство функций, содержащее в себе все траектории $ X(\omega, \blankarg) $ (но не обязательно только их).
    Рассмотрим $ \sigma $-алгебру, порождённую \defemph{цилиндрическими множествами}:
    \[
        \borel_{\trajectories}^T =
        \sigma\left( \left\{ x \in \trajectories \mid x(t_1) \in B_1, \ldots, x(t_n) \in B_n \right\}_{n \in \naturals, \{t_k\}_{k=1}^n \subseteq T, \{B_k\}_{k=1}^n \subseteq \borel} \right)
    \]
    Отображение $ X(\omega, \blankarg) $ определяет измеримое отображение $ (\Omega, \setfamily) $ в $ (\trajectories, \borel_{\trajectories}^T) $:
    \[
        \forall B \in \borel_{\trajectories}^T \quad \{ \omega \in \Omega \mid X(\omega, \blankarg) \in B \} \in \setfamily
    \]
    На пространстве $ (\trajectories, \borel_{\trajectories}^T) $ вероятностную меру можно определить следующим образом:
    \[
        \forall B \in \borel_{\trajectories}^T \quad \proba_X B = \proba \{ \omega \in \Omega \mid X(\omega, \blankarg) \in B \}
    \]
    Вероятностное пространство $ (\trajectories, \borel_{\trajectories}^T, \proba_X) $ называется \defemph{выборочным (вторичным) пространством}.
\end{definition}

\begin{Exercise}[counter=SecExercise, label={exercise:basics:sampling_space}]
    \noindent
    Рассмотрим вероятностное пространство $ \left( \{0, 1, 2, 3\}, 2^{\{0, 1, 2, 3\}}, \proba \right) $,
    где $ \proba: \proba \{0\} = \ldots = \proba \{3\} = \frac{1}{4} $,
    и случайный процесс $ X(\omega, t) = \sin\left( t + \frac{\pi}{2} \omega \right) $, $ t \in T = [0; 2 \pi] $.
    Построить вторичное (выборочное) вероятностное пространство.
\end{Exercise}

\begin{Answer}
    \noindent
    Возьмём в качестве пространства функций $ \trajectories = \left\{ \sin(t), \sin\left( t + \frac{\pi}{2} \right), \sin\left( t + \pi \right), \sin\left( t + \frac{3 \pi}{2} \right) \right\} $.
    Тогда $ \borel_\trajectories^T = 2^\trajectories $,
    так как для любого подмножества $ \trajectories $ можно подобрать цилиндрическое множество, с которым оно совпадает.
    Наконец,
    \[
        \proba_X: \quad \proba_X \left\{ \sin(t) \right\} = \ldots = \proba_X \left\{ \sin\left( t + 3 \pi/2 \right) \right\} = \frac{1}{4}
    \]
\end{Answer}


Существование различных случайных процессов с одними и теми же
вероятностными свойствами приводит к желанию (а иногда и необходимости)
в некотором смысле отождествлять процессы,
у которых конечномерные распределения совпадают.

\begin{definition}
    \label{definition:basics:modification}
    Пусть $ X $ и $ Y $~--- два случайных процесса,
    определённые на одном и том же вероятностном пространстве $ (\Omega, \setfamily, \proba) $ и множестве $ T $.
    Данные процессы называются \defemph{стохастически эквивалентными} в случае равенства почти наверное их реализаций в любой выбранный момент,
    то есть
    \[
        \forall t \in T \quad \proba \{ \omega \in \Omega \mid X(\omega, t) = Y(\omega, t) \} = 1
    \]
    В этом случае $ Y $ называют \defemph{модификацией} процесса $ Y $ (и наоборот).
\end{definition}

\begin{statement}
    \label{statement:basics:finite_distributions_of_modifications}
    Стохастически эквивалентные случайные процессы имеют одинаковое семейство конечномерных распределений.
\end{statement}

Например, такое отождествление полезно для осмысленного определения непрерывного случайного процесса:
\begin{definition}
    \label{definition:basics:continious_stochastic_process}
    Случайный процесс называется \defemph{непрерывным} в случае,
    если существует его модификациея с непрерывными реализациями.
\end{definition}

\begin{Exercise}[counter=SecExercise, label={exercise:basics:continious_stochastic_process}]
    \noindent
    Пусть $ \eta \sim \uniform_{[0;1]} $.
    Определим случайный процесс $ X_t = \indicator_{\{\eta\}}(t) $
    (то есть $ X_t = 1 $ в том и только в том случае, когда $ \eta = t $, и равен $ 0 $ иначе).
    Является ли $ X_t $ непрерывным процессом?
\end{Exercise}

\begin{Answer}
    \noindent
    Да, является.
    Процесс $ Y_t \equiv 0 $ является его модификацией.
\end{Answer}


При исследовании случайных процессов также бывает полезно рассматривать их моменты,
дающие некоторое представление об усреднённом поведении процесса.
В отличие от моментов случайных величин, любые моменты случайного процесса также зависят от времени.

\begin{definition}
    \label{definition:basics:mean_function}
    Если $ \forall t \in T $ существует $ \expect X_t $,
    то функция $ m_X(t) = \expect X_t $ определена и называется \defemph{функцией среднего}.
\end{definition}

Аналогично вводятся функции любых других моментов случайной величины $ X_t $.
При работе со случайными процессами нас также будут интересовать моменты,
<<разнесённые во времени>>.

\begin{definition}
    \label{definition:basics:second_order_moment_functions}
    Если $ \forall t_1, t_2 \in T $ существует $ \expect X_{t_1} X_{t_2} $,
    то функции $ K_X(t_1, t_2) = \expect X_{t_1} X_{t_2} $ и $ R_X(t_1, t_2) = \expect \rvcenter X_{t_1} \rvcenter X_{t_2} $
    определены и называются, соответственно, \defemph{ковариационной} и \defemph{корреляционной функциями}.%
    \footnote{Данные обозначения не являются общепринятыми, а также несколько контринтуитивны; при чтении сторонних источников будьте внимательны.}
\end{definition}

\begin{statement}
    \label{statement:basics:correlation_and_covariation_connection}
    Функции $ K_X(t_1, t_2) $ и $ R_X(t_1, t_2) $ одновременно либо определены, либо не определены,
    причём в первом случае функция $ m_X(t) $ определена и $ R_X(t_1, t_2) = K_X(t_1, t_2) - m_X(t_1) m_X(t_2) $.
\end{statement}

\begin{proof}
    Следует из свойств моментов.
\end{proof}

\begin{Exercise}[counter=SecExercise, label={exercise:basics:random_point_om_segment_moments}]
    \noindent
    Найти корреляционную функцию случайного процесса из задачи \ref{exercise:basics:random_point_om_segment}.
\end{Exercise}

\begin{Answer}
    \noindent
    Для любого $ t_0 $ случайная величина $ X_{t_0} $ может принимать только два значения~--- $ 0 $ или $ 1 $;
    это бернуллиевская случайная величина.
    Найдём параметр её распределения:
    \[
        \proba \{X_t = 1\} = \proba \{t \leqslant \eta\} = 1 - F_\eta(t)
    \]
    Следовательно, $ m_X(t) = \expect X_t = 1 - F_\eta(t) $.
    Далее,
    \begin{multline*}
        K_X(t_1, t_2) = \expect X_{t_1} X_{t_2} = 1 \cdot \proba \left( \{X_{t_1} = 1\} \cap \{X_{t_2} = 1\} \right) = \\
        = \proba \left( \{t_1 \leqslant \eta\} \cap \{t_2 \leqslant \eta\} \right) = 1 - F_\eta\left( \max \{t_1, t_2\} \right)
    \end{multline*}
    Наконец,
    \begin{multline*}
        R_X(t_1, t_2) = 1 - F_\eta\left( \max \{t_1, t_2\} \right) - (1 - F_\eta(t_1)) \cdot (1 - F_\eta(t_2)) = \\
        = F_\eta(t_1) + F_\eta(t_2) - F_\eta(t_1) \cdot F_\eta(t_2) - F_\eta(\max\{t_1, t_2\})
    \end{multline*}
    В частности, если $ t_1, t_2 \in [0; 1] $,
    \[
        R_X(t_1, t_2) = t_1 + t_2 - t_1 t_2 - \max\{t_1, t_2\} = \min\{t_1, t_2\} - t_1 t_2
    \]
\end{Answer}

\begin{Exercise}[counter=SecExercise, label={exercise:basics:cosine_stochastic_process}]
    \noindent
    Пусть $ \xi \sim \normal(0, 1) $ и $ \eta \sim U_{[-\pi; \pi]} $~--- независимые случайные величины.
    Определим случайный процесс $ X $ следующим образом: $ X_t = \xi \cdot \cos(t + \eta) $, где $ t \in \reals $.
    Найдите функцию среднего и корреляционную функцию процесса.
\end{Exercise}

\begin{Answer}
    \noindent
    Поскольку $ \xi $ и $ \eta $ независимы,
    \[
        m_X(t) = \expect X_t = \expect \xi \cdot \expect \cos(t + \eta) = 0 \cdot \ldots = 0
    \]
    \begin{multline*}
        R_X(t_1, t_2) = K_X(t_1, t_2) - 0 = \expect X_{t_1} X_{t_2} = \expect \xi^2 \cdot \expect \left( \cos(t_1 + \eta) \cdot \cos(t_2 + \eta) \right) = \\
        = 1 \cdot \frac{1}{2} \expect \left( \cos(t_1 - t_2) + \cos(t_1 + t_2 + 2 \eta) \right) = \frac{1}{2} \cos(t_1 - t_2)
    \end{multline*}
\end{Answer}

\begin{Exercise}[counter=SecExercise, label={exercise:basics:cos_and_sin}]
    \noindent
    Пусть $ U $, $ V $ и $ W $~--- независимые в совокупности случайные величины.
    Известно, что $ U $ и $ V $ обладают нулевым матожиданием и дисперсией $ D $,
    а $ W $ распределена с плотностью
    \[
        \rho_W(w) = \frac{2 \lambda}{\pi} \cdot \frac{\indicator_{[0; +\infty)}(w)}{\lambda^2 + w^2}, \quad \lambda > 0
    \]
    Определим случайный процесс $ X_t = U \cos(W t) + V \sin(W t) $.
    Вычислите функцию среднего и корреляционную функцию.
\end{Exercise}

\begin{Answer}
    \noindent
    Поскольку $ U $, $ V $ и $ W $ независимы в совокупоности,
    \[
        m_X(t) = \expect X_t = \expect U \cdot \expect \cos(W t) + \expect V \cdot \expect \sin(W t) = 0 \cdot \ldots + 0 \cdot \ldots = 0
    \]
    Корреляционную функцию удобно искать с помощью формулы полной вероятности в непрерывном случае:
    \[
        R_X(t_1, t_2) = \expect \left( \expect(X_{t_1} X_{t_2} \mid W = w) \right) = \int\limits_\reals \underbrace{\expect(X_{t_1} X_{t_2} \mid W = w)}_{\defeq R_X(t_1, t_2 \mid w)} \cdot \rho_W(w) \, dw
    \]
    В силу нулевого матожидания,
    \begin{multline*}
        R_X(t_1, t_2) = \expect \left( (U \cos(w t_1) + V \sin(w t_1)) \cdot (U \cos(w t_2) + V \sin(w t_2)) \right) = \\
        = \expect (U^2) \cdot \cos(w t_1) \cos(w t_2) + 2 \cdot \underbrace{\expect U \expect V}_{0} \cdot \ldots + \expect (V^2) \cdot \sin(w t_1) \sin(w t_2) = D \cos(w(t_1 - t_2))
    \end{multline*}
    Наконец,
    \[
        R_X(t_1, t_2) = \int\limits_0^{+\infty} D \cos(w(t_1 - t_2)) \cdot \frac{2 \lambda}{\pi} \frac{1}{\lambda^2 + w^2} \, dw = D e^{-\lambda |t_1 - t_2|}
    \]
    Здесь использовалось значение интеграла Лапласа:
    \[
        \int\limits_0^\infty \frac{\cos(\alpha x)}{1 + x^2} \, dx = \frac{\pi}{2} e^{-|\alpha|}
    \]
\end{Answer}


\begin{definition}
    \label{definition:basics:correlation_coefficient_function}
    \defemph{Функцией коэффициента корреляции} называют функцию
    \[
        r_X(t_1, t_2) = \frac{R_X(t_1, t_2)}{\sqrt{R_X(t_1, t_1) \cdot R_X(t_2, t_2)}} = \frac{\covariance{X_{t_1}}{X_{t_2}}}{\sqrt{\dispersion X_{t_1} \dispersion X_{t_2}}}
    \]
\end{definition}

Данная функция, если определена, принимает значения от $ -1 $ до $ 1 $
и имеет смысл степени \uline{линейной} связи сечений процесса,
соответствующих выбранным моментам времени.

\begin{Exercise}[counter=SecExercise, label={exercise:basics:correlation_coefficient_function}]
    \noindent
    Найти функции коэффициента корреляции для процессов из задач \ref{exercise:basics:cosine_stochastic_process} и \ref{exercise:basics:cos_and_sin}.
\end{Exercise}

\begin{Answer}
    \noindent
    \begin{itemize}
        \item
            Задача \ref{exercise:basics:cosine_stochastic_process}:
            $
                \displaystyle
                r_X(t_1, t_2) = \frac{\frac{1}{2} \cos(t_1 - t_2)}{\sqrt{\frac{1}{2} \cdot \frac{1}{2}}} = \cos(t_1 - t_2)
            $.

            Если взять два произвольных момента времени и начать сдвигать их друг к другу или друг от друга,
            будет наблюдаться периодическая корреляция и декорреляция соответствующих сечений.
        \item
            Задача \ref{exercise:basics:cos_and_sin}:
            $
                \displaystyle
                r_X(t_1, t_2) = \frac{D e^{-\lambda |t_1 - t_2|}}{\sqrt{D e^{-\lambda \cdot 0} \cdot D e^{-\lambda \cdot 0}}} = e^{-\lambda |t_1 - t_2|}
            $.

            Несмотря на схожесть процессов, в данном случае наблюдается корреляция,
            затухающая экспоненциально с ростом разницы между моментами времени,
            в которых взяты сечения.

            Дело в том, что в первом процессе случайным был фазовый сдвиг, а потому реализации процесса <<не расползались>>.
            Во втором же случае случайной является ещё и частота, и линейная связь между разными моментами времени быстро теряется
            (реализации <<декогерируют>>).
    \end{itemize}
\end{Answer}


Из курса теории вероятностей вы должны помнить,
что случайные величины удобно исследовать при помощи характеристической функции.
Аналогичный объект можно ввести и для случайного процесса.

\begin{definition}
    \label{definition:basics:characteristic_function}
    \defemph{Характеристической функцией случайного процесса} $ X \colon \Omega \times T \to \reals $ называется функция
    $ \varphi_X(t, s) \colon T \times \reals \to \complexes $, $ \varphi_X(t, s) = \varphi_{X_t}(s) \defeq \expect \exp \left( i s \cdot X_t \right) $,
    где $ i^2 = -1 $.
\end{definition}
