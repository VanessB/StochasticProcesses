\section{Элементы стохастического анализа} \label{section:calculus}

Общая цель данного раздела~--- дать стохастические аналоги привычным определениям из математического анализа~---
пределу, производной и интегралу~--- для случайных процессов.
Как мы помним, в теории вероятностей было несколько типов сходимости случайных величин,
поэтому и для случайных процессов есть много вариантов определить вышеуказанное.
Изучаемые в этом курсе варианты не претендуют на полноту охвата.

Начнём с некоторых вспомогательных утверждений, следующих напрямую из функционального анализа.
Вспомним, что множество случайных величин на вероятностном пространстве $ (\Omega, \setfamily, \proba) $
с конечным вторым моментом образует гильбертово пространство $ \lebesgue_2 (\Omega, \setfamily, \proba) $
со скалярным произведением, определённым по формуле $ \displaystyle \dotprod{\xi}{\eta} \defeq \expect (\xi \eta) = \int_\Omega \xi(\omega) \eta(\omega) \, d\omega $.
Пользуясь этим, приведём ряд свойств гильбертовых пространств, полезных для задач теории вероятностей.
\begin{itemize}
    \item
        Неравенство Коши-Буняковского-Шварца: $ \dotprod{\xi}{\eta}^2 \leqslant \dotprod{\xi}{\xi} \cdot \dotprod{\eta}{\eta} $.
        Отсюда, например, $ (\expect \xi)^2 \leqslant \expect (\xi^2) $ (если взять $ \eta \equiv 1 $).
    \item
        Скалярное произведение~--- непрерывная функция обеих своих переменных в смысле топологии,
        порождённой нормой $ \| \xi \| = \sqrt{\dotprod{\xi}{\xi}} $.

        Так как пространство является нормированным, непрерывность также является секвенциальной непрерывностью:
        если $ \xi_n \limarrow[\convnorm]{n \to \infty} \xi $ и $ \eta_n \limarrow[\convnorm]{n \to \infty} \eta $, то и $ \dotprod{\xi_n}{\eta_n} \limarrow{n \to \infty} \dotprod{\xi}{\eta} $.
        В частности, $ \expect (\xi_n \eta_n) \limarrow{n \to \infty} \expect (\xi \eta) $.

        Отсюда следует возможность перестановки передела и математического ожидания:
        \begin{statement}
            \label{statement:calculus:swap_expectation_and_limit}
            $ \displaystyle \lim_{n \to \infty} \expect \xi_n = \expect \limmeansq_{n \to \infty} \xi_n $.
        \end{statement}
    \item
        Можно получить аналог критерия Коши:
        \begin{statement}
            \label{statement:calculus:dotprod_converges_to_const}
            Если для последовательности $ \{ \xi_n \}_{n \in \naturals} $ случайных величин из $ \lebesgue_2 (\Omega, \setfamily, \proba) $
            найдётся константа $ C $ такая, что для всяких подпоследовательностей $ \{ \xi_{n_k} \}_{k \in \naturals} $ и $ \{ \xi_{m_k} \}_{k \in \naturals} $
            выполнено $ \dotprod{\xi_{n_k}}{\xi_{m_k}} \limarrow{k \to \infty} C $,
            то $ \displaystyle \exists \xi \colon \limmeansq_{n \to \infty} \xi_n = \xi $.
        \end{statement}

        \begin{proof}
            \[
                \| \xi_{n_k} - \xi_{m_k} \|^2 = \dotprod{\xi_{n_k}}{\xi_{n_k}} - 2 \dotprod{\xi_{n_k}}{\xi_{m_k}} + \dotprod{\xi_{m_k}}{\xi_{m_k}} \limarrow{k \to \infty} C - 2 \cdot C + C = 0
            \]
            Отсюда по критерию Коши получаем существование предела.
        \end{proof}
\end{itemize}


\subsection{Непрерывность} \label{subsection:continuity}

Ранее мы уже приводили определение непрерывного случайного процесса,
основанное на понятии <<почти всюду>> (см. определение \ref{definition:basics:continious_stochastic_process}).
Приведём аналогичные определения для других типов сходимости.

\begin{definition}
    \label{definition:calculus:continious_stochastic_process_mean_squares}
    Случайный процесс $ X $ называется \defemph{непрерывным в среднем квадратичном в точке $ t \in T $} в случае $ X_{t + \Delta t} \limarrow[\convmeansq]{\Delta t \to 0} X_t $.
\end{definition}

\begin{definition}
    \label{definition:calculus:continious_stochastic_process_probability}
    Случайный процесс $ X $ называется \defemph{непрерывным по вероятности в точке $ t \in T $} в случае $ X_{t + \Delta t} \limarrow[\convproba]{\Delta t \to 0} X_t $.
\end{definition}

\begin{definition}
    \label{definition:calculus:continious_stochastic_process_distribution}
    Случайный процесс $ X $ называется \defemph{непрерывным по распределению в точке $ t \in T $} в случае $ X_{t + \Delta t} \limarrow[\convdistr]{\Delta t \to 0} X_t $.
\end{definition}


По аналогии с классическим математическим анализом далее можно ввести понятия непрерывности на множестве.

В этом курсе мы будем в основном заниматься непрерывностью
(а затем и дифференцируемостью и интегрируемостью) в среднем квадратичном.
Для указанного типа непрерывности есть удобный критерий в терминах функций моментов:

\begin{theorem}[Критерий непрерывности в среднем квадратичном]
    \label{theorem:calculus:mean_squares_continuity_test}
    Следующие условия эквивалентны:
    \begin{enumerate}
        \item
            Случайный процесс второго порядка $ X $ непрерывен в среднем квадратичном в точке~$ t $.
        \item
            Ковариационная функция $ K_X(t, s) $ непрерывна в точке $ (t, t) $. %на диагонали $ t = s $.
        \item
            Корреляционная функция $ R_X(t, s) $ непрерывна в точке $ (t,t) $, %на диагонали $ t = s $,
            функция среднего $ m_X(t) $ непрерывна в точке $ t $.
    \end{enumerate}
\end{theorem}

\begin{exercise}
    \label{exercise:calculus:continuous_step_process}
    Случайный процесс $ X $ определён как $ X_t = \xi \cdot \indicator_{(-\infty;r)}(t) + \eta \cdot \indicator_{[r;+\infty)}(t) $,
    где $ t \in T = [0;1] $, $ \xi $ и $ \eta $~--- независимые одинаково распределённые нормальные случайные величины,
    а $ r $~--- равномерно распределённая по $ T $ случайная величина, не зависящая от $ \xi $ и $ \eta $.
    Исследовать процесс $ X $ на непрерывность в среднем квадратичном.
\end{exercise}

\begin{solution}
    Заметим, что вероятность получить непрерывную реализацию процесса равна $ 0 $,
    поскольку это означало бы, что $ \xi $ и $ \eta $ совпали
    (таким образом, процесс не является непрерывным в смысле определения \ref{definition:basics:continious_stochastic_process}).
    Однако, процесс оказывается непрерывным в среднем квадратическом.
    Для доказательства этого воспользуемся теоремой выше.

    Обозначим $ \expect \xi = \expect \eta = m $, $ \dispersion \xi = \dispersion \eta = D $.
    Функция среднего:
    \[
        m_X(t) = \expect X_t = \expect (X_t \mid t < r) \proba \{t < r\} + \expect (X_t \mid t \geqslant r) \proba \{t \geqslant r\} = \expect \xi \cdot (1 - t) + \expect \eta \cdot t = m
    \]
    Пусть, без ограничения общности, $ s \leqslant t $.
    Тогда корреляционная функция:
    \begin{multline*}
        R_X(t,s) = \expect \left( \rvcenter X_t \rvcenter X_s \right) =
        \expect \left( \rvcenter X_t \rvcenter X_s \Mid r < s \right) \proba \{r < s\} +
        \expect \left( \rvcenter X_t \rvcenter X_s \Mid s \leqslant r < t \right) \proba \{s \leqslant r < t\} + \\
        + \expect \left( \rvcenter X_t \rvcenter X_s \Mid t \leqslant r \right) \proba \{t \leqslant r\} =
        \expect (\rvcenter \xi \rvcenter \xi) \cdot s +
        \expect (\rvcenter \xi \rvcenter \eta) \cdot (t - s) +
        \expect (\rvcenter \eta \rvcenter \eta) \cdot (1 - t) =
        D \cdot (1 + s - t)
    \end{multline*}
    \[
        R_X(t, s) = D \cdot (1 - |t - s|)
    \]
    Функция среднего непрерывна, корреляционная функция непрерывна на диагонали $ s = t $.
    Значит, $ X $ непрерывен в среднем квадратичном.
\end{solution}



\subsection{Дифференцирование} \label{subsection:derivative}

\begin{definition}
    \label{definition:calculus:derivative}
    \defemph{Производной в среднем квадратичном случайного процесса $ X $ в точке $ t $} называется предел
    \[
        X_t' = \limmeansq_{\Delta t \to 0} \frac{X_{t + \Delta t} - X_t}{\Delta t}
    \]
    Если указанный предел существует, процесс $ X $ называют \defemph{дифференцируемым в среднем квадратичном в точке $ t $}.
\end{definition}

Абсолютно аналогично вводятся производные и в смысле других сходимостей.
При этом можно получить следующее утверждение:

\begin{statement}
    \label{statement:calculus:continuity_from_differentiability}
    В любом типе сходимости из дифференцируемости следует непрерывность.
\end{statement}

Как и в случае с непрерывностью в среднем квадратичном,
есть удобный критерий, связывающий дифференцируемость в среднем квадратичном и функции моментов:

\begin{theorem}[Критерий дифференцируемости в среднем квадратичном]
    \label{theorem:calculus:mean_squares_differentiability_test}
    Следующие условия эквивалентны:
    \begin{enumerate}
        \item
            Случайный процесс второго порядка $ X $ дифференцируем в среднем квадратичном в точке $ t $.
        \item
            Существует следующий двойной предел:
            \[
                \lim_{\Delta t, \Delta s \to 0} \frac{1}{\Delta t \Delta s}
                \left( K_X(t + \Delta t, t + \Delta s) - K_X(t + \Delta t, t) - K_X(t, t + \Delta s) + K(t, t) \right)
            \]
        \item
            Функция $ m_X(t) $ дифференцируема в точке $ t $ и существует следующий двойной предел:
            \[
                \lim_{\Delta t, \Delta s \to 0} \frac{1}{\Delta t \Delta s}
                \left( R_X(t + \Delta t, t + \Delta s) - R_X(t + \Delta t, t) - R_X(t, t + \Delta s) + R(t, t) \right)
            \]
    \end{enumerate}
\end{theorem}

Заметим, что предел из утверждения теоремы не является смешанной производной.
Смешанная производная выражается через повторный предел,
мы же имеем дело с двойным пределом.
Из существования данного передела следует существование смешанной производной;
в обратную сторону это неверно.

\begin{exercise}
    \label{exercise:moments_functions_of_derivative}
    Выразить функцию среднего и корреляционную функцию $ X_t' $ через $ m_X(t) $ и $ R_X(t, s) $.
\end{exercise}

\begin{solution}
    \[
        m_{X'}(t) = \expect \limmeansq_{\Delta t \to 0} \frac{X_{t + \Delta t} - X_t}{\Delta t} = \lim_{\Delta t \to 0} \frac{1}{\Delta t} \expect (X_{t + \Delta t} - X_t) =
        \lim_{\Delta t \to 0} \frac{m_X(t + \Delta t) - m_X(t)}{\Delta t} = \frac{d}{d t} m_X(t)
    \]
    \begin{multline*}
        R_{X'}(t, s) = \expect \left[ \limmeansq_{\Delta t \to 0} \frac{\rvcenter X_{t + \Delta t} - \rvcenter X_t}{\Delta t} \cdot
        \limmeansq_{\Delta s \to 0} \frac{\rvcenter X_{s + \Delta s} - \rvcenter X_s}{\Delta s} \right] = \\
        = \lim_{\Delta t, \Delta s \to 0} \frac{1}{\Delta t \Delta s} \left( \expect (\rvcenter X_{t + \Delta t} \rvcenter X_{s + \Delta s}) -
        \expect (\rvcenter X_{t + \Delta t} \rvcenter X_s) - \expect (\rvcenter X_t \rvcenter X_{s + \Delta s}) + \expect (\rvcenter X_t \rvcenter X_s) \right) = \\
        = \frac{\partial^2}{\partial t \partial s} R_X(t, s)
    \end{multline*}
    Отметим, что центрирование случайной величины (<<$ \rvcenter{X}_t' $>>) было сразу внесено внутрь предела и дроби.
    Проверьте сами корректность данного шага.
\end{solution}


\begin{exercise}
    \label{exercise:calculus:Poisson_process_continuity_and_differentiability}
    Исследовать пуассоновский процесс на неперывность и дифференцируемость <<почти наверное>>, в среднем квадратичном, по вероятности и по распределению.
\end{exercise}

\begin{solution}
    Непрерывности <<почти наверное>>, очевидно, нет (с вероятностью $ 1 $ траектории разрывны),
    а потому нет и дифференцируемости (см. утверждение \ref{statement:calculus:continuity_from_differentiability}).

    Непрерывность в среднем квадратичном имеет место, поскольку $ m_K(t) = \lambda t $ и $ R_K(t, s) = \lambda \min \{t, s\} $
    непрерывны (в том числе и $ R_K $~--- на диагонали), а дифференцируемости в среднеквадратичном нет,
    поскольку $ R_K $ не дифференцируема на диагонали.
    Уже отсюда следует непрерывность по вероятности и распределению (вспомните диаграмму следования одного типа сходимости из другого).

    Покажем дифференцируемость по вероятности.
    Из неё последует непрерывность по вероятности, а также дифференцируемость и непрерывность по распределению.
    Для этого надо выбрать кандидата для производной.
    Попробуем взять тождественно нулевой случайный процесс $ X $.
    Нужно проверить, что
    \[
        \forall \varepsilon > 0 \qquad \lim_{\Delta t \to 0} \proba \left\{ \left| \frac{K_{t + \Delta t} - K_t}{\Delta t} - X \right| > \varepsilon \right\} = 0
    \]
    Действительно,
    \begin{multline*}
        \proba \left\{ \left| \frac{K_{t + \Delta t} - K_t}{\Delta t} - 0 \right| > \varepsilon \right\} =
        1 - \proba \left\{ \frac{K_{t + \Delta t} - K_t}{\Delta t} \leqslant \varepsilon \right\} =
        1 - \proba \left\{ \frac{K_{\Delta t} - K_0}{\Delta t} \leqslant \varepsilon \right\} = \\
        = 1 - \proba \{ K_{\Delta t} < \Delta t \cdot \varepsilon \} \leqslant 1 - \proba \{ K_{\Delta t} = 0 \} = 1 - e^{- \lambda \Delta t} \limarrow{\Delta t \to 0} 0
    \end{multline*}
\end{solution}


\begin{exercise}
    \label{exercise:calculus:Wiener_process_diffirentiability}
    Показать, что винеровский процесс ни в какой точке не является дифференцируемым даже по распределению.
\end{exercise}

\begin{solution}
    Рассмотрим $ X_t (\Delta t) = (W_{t + \Delta t} - W_t) / \Delta t $.
    Из свойств винеровского процесса имеем $ X_t(\Delta t) \sim \normal(0, |\Delta t| / |\Delta t|^2) = \normal(0, |\Delta t|^{-1}) $.
    Из сходимости по распределению следует сходимость характеристических функций поточечно к характеристической функции предельного распределения.
    Характеристическая функция случайной величины $ X_t(\Delta t) $~--- $ \varphi(s) = \exp(- s^2 / |2 \Delta t|) $.
    В точке $ s = 0 $ она равна единице, а в $ s \neq 0 $ сходится к нулю при $ \Delta t \to 0 $.
    Значит, $ \varphi $ сходится поточечно к разрывной функции,
    которая не может быть характеристической функцией никакого распределения.
\end{solution}



\subsection{Интегрирование по времени} \label{subsection:time_integration}

\begin{definition}
    \label{definition:calculus:integral_sum}
    Пусть случайный процесс $ X $ определён на отрезке $ [a; b] $.
    Рассмотрим разбиение $ \mathcal{T} $ этого отрезка $ a = t_0 < t_1 < \ldots < t_n = b $,
    где на каждом полуинтервале $ \Delta_k = [t_{k-1}; t_k) $ длины $ \Delta t_k = t_k - t_{k-1} $ взято по точке $ \tau_k $.
    \\[0.25\baselineskip]
    Случайная величина $ Z(\partition) = \sum_{k=1}^n X_{\tau_k} \Delta t_k $ называется
    \defemph{интегральной суммой Римана случайного процесса $ X $, построенной по разбиению $ \partition $}.
    \\[0.25\baselineskip]
    Величина $ d(\partition) = \max \Delta t_k $ называется \defemph{мелкостью разбиения $ \partition $}.
\end{definition}

\begin{definition}
    \label{definition:calculus:integral}
    \defemph{Интегралом Римана процесса $ X $ на отрезке $ [a; b] $ в смысле среднего квадратичного} называется предел
    \[
        \limmeansq_{d(\partition) \to 0} Z(\partition) \defeq \int\limits_a^b X_t \, dt
    \]
    Если указанный предел существует, процесс $ X $ называют \defemph{интегрируемым в среднем квадратичном на отрезке $ [a; b] $}.
\end{definition}

Аналогично математическому анализу вводятся интегралы по бесконечным отрезкам:
берётся предел в нужном смысле (в нашем случае~--- в среднем квадратичном) при стремлении одного из концов в бесконечность.

Как и в случае с непрерывностью и дифференцируемостью в среднем квадратичном,
имеем удобный критерий интегрируемости:

\begin{theorem}[Критерий интегрируемости в среднем квадратичном]
    \label{theorem:calculus:mean_squares_integrability_test}
    Следующие условия эквивалентны:
    \begin{enumerate}
        \item
            Случайный процесс второго порядка $ X $ интегрируем в среднем квадратичном на отрезке $ [a; b] $.
        \item
            Существует и конечен следующий двойной интеграл Римана:
            $ \displaystyle \int\limits_a^b \int\limits_a^b K_X(t, s) \, ds \, dt $.
        \item
            Существуют и конечны следующие интегралы Римана:
            $ \displaystyle \int\limits_a^b m_X(t) \, dt $ и $ \displaystyle \int\limits_a^b \int\limits_a^b R_X(t, s) \, ds \, dt $.
    \end{enumerate}
\end{theorem}

%Можно привести следующее утверждение,
%неверное в общем случае для других типов сходимости:

\begin{statement}
    \label{statement:calculus:integrability_from_continuity}
    Из непрерывности в среднем квадратичном следует интегрируемость в среднем квадратичном.
\end{statement}


\begin{exercise}
    \label{exercise:calculus:crosscorrelation_derivative_integral}
    Рассмотрим $ \lebesgue_2 $-процесс $ X $, дифференцируемый в среднем квадратичном на отрезке $ [a; b] \subseteq T $.
    Пусть $ J_t = \int\limits_a^t X_s \, ds $.
    Требуется найти взаимную корреляционную функцию процессов $ X' $ и $ J $,
    то есть $ R_{X', J}(t, s) = \expect (\rvcenter X_t' \cdot \rvcenter J_s) $.
\end{exercise}

\begin{solution}
    \begin{multline*}
        R_{X', J}(t, s) = \expect \left[ \limmeansq_{\Delta t \to 0} \frac{\rvcenter X_{t + \Delta t} - \rvcenter X_t}{\Delta t} \cdot
        \limmeansq_{d(\partition) \to 0} \sum_{k=1}^n \rvcenter X_{\tau_k} \Delta t_k \right] = \\
        = \lim_{\Delta t, d(\partition) \to 0} \frac{1}{\Delta t}
        \expect \left[ \sum_{k=1}^n \left( \rvcenter X_{t + \Delta t} \rvcenter X_{\tau_k} - \rvcenter X_t \rvcenter X_{\tau_k} \right) \Delta t_k \right] = \\
        = \lim_{\Delta t, d(\partition) \to 0}
        \sum_{k=1}^n \frac{R_X(t + \Delta t, \tau_k) - R_X(t, \tau_k)}{\Delta t} \Delta t_k =
        \int\limits_0^s \frac{\partial R_X(t, \tau)}{\partial t} \, d \tau
    \end{multline*}
\end{solution}


О связи интеграла Римана в среднем квадратичном и потраекторного интеграла говорит следующее замечание:

\begin{remark}
    \label{remark:calculus:trajectories_integrals}
    Если почти все реализации случайного процесса $ X $ интегрируемы по Риману,
    то потраекторный интеграл (интеграл траектории $ X(\omega, \blankarg) $ по времени)
    есть случайная величина (то есть измеримая функция на $ (\Omega, \setfamily) $).
    Если при этом $ X $ интегрируем в среднем квадратичном,
    то потраекторный и среднеквадратичный интегралы совпадают с вероятностью $ 1 $.
\end{remark}
