\silentsection{Предисловие} \label{sec:intro}

Перед вами сборник всех семинаров по случайным процессам за авторством Бутакова\,И.\,Д.
Автор выражает благодарность Останину Павлу Антоновичу и Широбокову Максиму Геннадьевичу за предоставленные материалы.

\silentsubsection{Используемые обозначения}

\begin{center}
    \begin{tabularx}{\textwidth}{cl}
        $ \defarr $                      & <<\ldots по определению тогда и только тогда, когда \ldots>> \\
        $ \defeq $                       & <<\ldots по определению равно \ldots>> \\
        \rule{0pt}{16pt}%
        $ (\Omega, \setfamily, \proba) $ & \makecell[l]{вероятностное пространство ($ \Omega $~--- множество исходов, $ \setfamily $~--- $ \sigma $-алгебра, \\ $ \proba $~--- вероятностная мера).} \\
        $ \borel(A) $, $ \borel_A $      & \makecell[l]{Борелевская $ \sigma $-алгебра, определённая на множестве $ A $ (если $ A $ не указано, \\ по умолчанию предполагается $ A = \reals $).} \\
        $ \indicator_A $                 & индикаторная функция множества $ A $. \\
        $ \expect X $                    & математическое ожидание случайной величины $ X $. \\
        $ \dispersion X $                & дисперсия случайной величины $ X $. \\
        $ \rvcenter X $                  & <<центрированная>> случайная величина: $ \rvcenter X = X - \expect X $. \\
        \rule{0pt}{16pt}%
        $ \bernoulli(p) $                & распределение Бернулли с параметром $ p $. \\
        $ \binomial(n, p) $              & биномиальное распределение с параметрами $ n $ и $ p $. \\
        $ \poisson(\lambda) $            & распределение Пуассона с интенсивностью $ \lambda $. \\
        $ \uniform(A)$, $ \uniform_A $   & равномерное распределение на множестве $ A $. \\
        $ \expdistr(\lambda) $           & показательное распределение с параметром $ \lambda $ (интенсивность). \\
        $ \normal(\mu, \sigma^2) $       & нормальное распределение со средним $ \mu $ и дисперсией $ \sigma^2 $. \\
        \rule{0pt}{16pt}%
        $ \convmeansq $                  & сходимость в среднем квадратичном. \\
        $ \convalmost $                  & сходимость почти наверное. \\
        $ \convproba $                   & сходимость по вероятности. \\
        $ \convdistr $                   & сходимость по распределению. \\
        $ \almosteq $                    & равенство почти наверное. \\
        $ \distreq $                     & равенство по распределению.
    \end{tabularx}
\end{center}
