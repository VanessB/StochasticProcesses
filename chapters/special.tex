\section{Важные примеры случайных процессов} \label{section:special}

В этом разделе речь пойдёт о нескольких процессах особого вида,
наиболее часто встречающихся при исследовании реальных явлений.
Зачастую такие процессы именные.
На их примере мы продолжим практиковаться в решении задач,
а также введём несколько новых теоретических понятий.


\subsection{Пуассоновский процесс} \label{subsection:Poisson}

Данный процесс встречается в реальной жизни довольно часто;
он описывает поток случайных событий, которые регистрируются с некоторой постоянной <<интенсивностью>>.
Например, речь может идти о регистрации космических частиц, о кликах по ссылке,
о запросах к серверу, о проезжающих по магистрали автомобилях.
Дадим формальное определение.

\begin{definition}
    \label{definition:special:independent_deltas}
    Случайный процесс $ X $ называется \defemph{процессом с независимыми приращениями},
    если $ \forall n \in \naturals \;\, \forall \{t_i\}_{i=1}^n \subseteq T $ случайные величины
    $ X_{t_n} - X_{t_{n-1}}, \ldots, X_{t_2} - X_{t_1}, X_{t_1} $
    независимы в совокупности.
\end{definition}

\begin{definition}
    \label{definition:special:Poisson_process}
    \defemph{Пуассоновским процессом с интенсивностью $ \lambda > 0 $} называется случайный процесс $ K\colon \Omega \times [0; +\infty) \to \naturals $ такой, что
    \begin{enumerate}
        \item
            $ K_0 \almosteq 0 $.
        \item
            $ K $~--- процесс с независимыми приращениями.
        \item
            $ K_t - K_s \sim \poisson\left( \lambda \cdot (t - s) \right) $ (при $ t > s \geqslant 0 $).
    \end{enumerate}
\end{definition}

Это одно из эквивалентных определений, пуассоновский процесс можно определить и иначе:

\begin{theorem}[Явная конструкция пуассоновского процесса]
    \label{theorem:special:Poisson_process_explicit_definition}
    Пусть $ \xi_1, \ldots, \xi_k, \ldots \sim \expdistr(\lambda) $ и независимы в совокупности,
    $ S_n = \xi_1 + \ldots + \xi_n $.
    Тогда процесс $ X_t = \sup \{ n \mid S_n \leqslant t \} $ есть пуассоновский процесс с интенсивностью $ \lambda $.
\end{theorem}

Процесс $ X_t $, построенный по случайным величинам $ \xi_k $ способом, указанным в теореме,
называется \defemph{процессом восстановления} и отвечает следующей модели:
в нулевой момент включается прибор, который работает время $ \xi_1 $, после чего ломается.
Одновременно с поломкой включается следующий прибор, который работает случайное время $ \xi_2 $, и так далее.
Величина $ X_t $ отражает количество приборов, введённых в эксплуатацию к моменту $ t $.

\begin{statement}
    \label{statement:special:Poisson_process_properties}
    Пуассоновский процесс обладает следующими свойствами:
    \begin{enumerate}
        \item
            Реализации пуассоновского процесса~--- кусочно-постоянные неубывающие функции со значениями в $ \naturals $.
        \item
            С вероятностью $ 1 $ все скачки пуассоновского процесса равны единице.
        \item
            Время, когда произошёл $ n $-ый скачёк (обозначим его $ \tau_n $) имеет $ \Gamma(n, 1/\lambda) $-распределение:
            \[
                \rho_{\tau_n}(t) = \frac{\lambda^n x^{n-1}}{(n-1)!} e^{-\lambda t} \cdot \indicator_{[0;+\infty)}(t)
            \]
        \item
            Случайные величины $ \{\tau_{n} - \tau_{n-1}\}_{n \in \naturals} $ распределены экспоненциально с параметром $ \lambda $ и независимы.
        \item
            Число событий за конечный период времени конечно с вероятностью $ 1 $.
        \item
            Число событий $ K_{t+h} - K_t $ на промежутке $ (t; t+h] $ зависит лишь от длины промежутка $ h $:
            $ \proba \{ K_{t + h} - K_t = k \} = p(h, k) $
        \item
            Вероятность более чем одного скачка на полуинтервале $ (t; t + h] $ есть $ o(h) $,
            то есть $ \displaystyle \lim_{h \to +0} \proba \{ K_{t+h} - K_t > 1 \} = 0 $.
        \item
            Для коротких полуинтервалов $ (t; t+h] $ вероятность того, что на них произойдёт хотя бы один скачок,
            убывает линейно с уменьшением $ h $: $ \proba \{ K_{t+h} - K_t > 0 \} = 1 - e^{-\lambda h} = \lambda h + o(h) $ при $ h \to 0 $.
        \item
            Из определения распределения Пуассона:
            \[
                \proba\{K_t = k\} = \frac{(\lambda t)^k}{k!} e^{-\lambda}
            \]
    \end{enumerate}
\end{statement}

Наконец, приведём ещё одно из альтернативных определений пуассоновского процесса:

\begin{statement}
    \label{statement:special:Poisson_process_alternative_definition}
    Случайный процесс $ K\colon \Omega \times T \to \naturals $ является пуассоновским тогда и только тогда, когда он удовлетворяет следующим свойствам:
    \begin{enumerate}
        \item
            \defemph{(стационарность приращений)}
            $ \proba \{ K_{t + h} - K_t = k \} = p(h, k) $
        \item
            \defemph{(отсутствие последействия)}
            Приращения процесса независимы.
        \item
            \defemph{(ординарность)}
            $ \proba \{ K_{t + h} - K_t > 1 \} \in o(h) $
    \end{enumerate}
\end{statement}


\begin{statement}
    \label{statement:special:Poisson_process_correlation_function}
    Пусть $ K $~--- пуассоновский процесс с интенсивностью $ \lambda $.
    Тогда $ m_K(t) = \lambda t $, $ R_K(t, s) = \lambda \cdot \min \{t, s\} $.
\end{statement}

\begin{proof}
    Так как $ K_t \almosteq K_t - K_0 \sim \poisson(\lambda t) $, $ m_K(t) = \expect K_t = \lambda t $.
    Далее, в силу независимости приращений, при $ t \geqslant s $ имеем
    $ \covariance{K_t}{K_s} = \covariance{K_t - K_s + K_s}{K_s} = 0 + \covariance{K_s}{K_s} = \lambda t $.
    Поэтому $ R_k(t, s) = \lambda \cdot \min \{t, s\} $.
\end{proof}


\begin{Exercise}[counter=SecExercise, label={exercise:special:total_wait_time}]
    \noindent
    Поток прибывающих на железнодорожную станцию пассажиров моделируется пуассоновским процессом $ K $ с интенсивностью $ \lambda $.
    В момент $ t = 0 $ пассажиров нет, в момент $ t = t_0 $ прибывает первый поезд.
    Пусть $ \eta $~--- суммарное время ожидания прибытия поезда всеми пассажирами на станции.
    Найти $ \expect \eta $.
\end{Exercise}

\begin{Answer}
    \noindent
    \[
        \eta = \int\limits_0^{t_0} K_t \, dt, \qquad
        \expect \eta = \int\limits_\Omega d\proba \int\limits_0^{t_0} K_t \, dt = \int\limits_0^{t_0} dt \int\limits_\Omega K_t \, d \proba = \int\limits_0^{t_0} m_K(t) \, dt =
        \int\limits_0^{t_0} \lambda t \, dt = \frac{\lambda t_0^2}{2}
    \]
\end{Answer}



\begin{Exercise}[counter=SecExercise, label={exercise:special:Poisson_process_first_event_conditional}]
    \noindent
    Пусть $ K_t $~--- пуассоновский процесс с интенсивностью $ \lambda $,
    а $ \tau_1 $~--- момент первого скачка.
    Найдите $ \proba\{\tau_1 \leqslant s \mid K_t = 1\} $ при $ 0 < s < t $.
\end{Exercise}

\begin{Answer}
    \noindent
    Событие $ \{ \tau_1 \leqslant s \} $ означает, что первый скачок процесса произошёл не позже момента $ s $.
    Если при этом $ K_t = 1 $, то это означает, что $ K_s = 1 $.
    Тогда
    \begin{multline*}
        \proba \{\tau_1 \leqslant s \mid K_t = 1 \} = \proba \{K_s = 1 \mid K_t = 1\} = \frac{\proba\left( \{K_s = 1\} \cap \{K_t = 1\} \right)}{\proba \{K_t = 1 \}} = \\
        = \frac{\proba\left( \{K_s = 1\} \cap \{K_t - K_s = 0\} \right)}{\proba\{K_t = 1\}}
        = \frac{\frac{\lambda s}{1!} e^{-\lambda s} \cdot \frac{(\lambda(t - s))^0}{0!} e^{-\lambda(t - s)}}{\frac{\lambda t}{1!} e^{-\lambda t}} = \frac{s}{t}
    \end{multline*}
\end{Answer}


\begin{Exercise}[counter=SecExercise, label={exercise:special:Poisson_process_third_event}]
    \noindent
    Пусть $ K $~--- пуассоновский процесс с интенсивностью $ \lambda $, $ \tau_3 $~--- время третьего скачка процесса.
    Найти $ \proba \{\tau_3 \leqslant 2 \} $.
\end{Exercise}

\begin{Answer}
    \noindent
    \[
        \proba \{ \tau_3 \leqslant 2 \} = \proba \{ K_2 \geqslant 3 \} = 1 - \proba \{ K_2 < 3 \} = 1 - e^{-2 \lambda} - \frac{2 \lambda}{1!} e^{-2 \lambda} - \frac{(2 \lambda)^2}{2!} e^{-2 \lambda}
    \]
\end{Answer}


\begin{Exercise}[counter=SecExercise, label={exercise:special:Poisson_process_no_event}]
    \noindent
    Пусть $ \eta \sim \uniform_{[0; 1]} $,
    $ K $~--- пуассоновский процесс с интенсивностью $ \lambda $, и $ \eta $ не зависит от $ K $.
    Найти $ \proba\{K_\eta = K_{\eta + 1}\} $.
\end{Exercise}

\begin{Answer}
    \noindent
    По формуле полной вероятности,
    \[
        \proba \{K_\eta = K_{\eta + 1}\} = \int\limits_\reals \proba\{\underbrace{K_{t + 1} - K_t}_{\sim \poisson(1 \cdot \lambda)} = 0 \mid \eta = t\} \cdot \rho_\eta(t) \, dt =
        \int\limits_0^1 e^{-1 \cdot \lambda} \, dt = e^{-\lambda}
    \]
\end{Answer}

Пуассоновский процесс моделирует лишь поток некоторых событий.
Иногда сами события также имеют сложную и/или случайную природу.
Тогда требуется построить более продвинутую модель,
наследующую от пуассоновского процесса только характер возникновения событий с течением времени.
В качестве примера такой модели можно привести \defemph{сложный (составной) пуассоновский процесс}.

\begin{definition}
    \label{definition:special:compound_Poisson_process}
    Рассмотрим пуассоновский процесс $ K $ и набор независимых одинаково распределённых случайных величин $ \{ V_k \}_{k \in \naturals} $.
    \defemph{Сложным пуассоновским процессом} называется процесс $ \displaystyle Q_t = \sum_{j = 1}^{K_t} V_j $.
\end{definition}

Это означает следующее: $ Q_0 \almosteq 0 $, и в каждый момент, когда $ K $ испытывает скачок, к $ Q $ добавляется $ V_j $.

\begin{statement}
    \label{statement:special:compound_Poisson_process_independent_deltas}
    Сложный пуассоновский процесс является процессом с независимыми приращениями.
\end{statement}

\begin{proof}
    Следует из независимости приращений $ K $ и независимости $ \{V_j\}_{j \in \naturals} $.
\end{proof}


\begin{statement}
    \label{statement:special:compound_Poisson_process_characteristic_function}
    Рассмотрим сложный пуассоновский процесс $ Q $, определённый по пуассоновскому процессу $ K $ с интенсивностью $ \lambda $
    и случайным величинам $ \{ V_j \}_{j \in \naturals} $.
    Пусть $ \varphi_V(s) $~--- характеристическая функция случайных величин $ V_j $.
    Тогда характеристичекая функция процесса $ Q $ задаётся формулой
    \[
        \varphi_{Q_t}(s) = e^{(\varphi_V(s) - 1) \cdot \lambda t}
    \]
\end{statement}

\begin{proof}
    \begin{multline*}
        \varphi_{Q_t}(s) = \sum_{k = 0}^\infty \expect \left( e^{i s \cdot Q_t} \mid K_t = k \right) \proba \{ K_t = k \} =
        \sum_{k = 0}^\infty \expect \left( e^{i s \cdot (V_1 + \ldots + V_k)} \right) \cdot \frac{(\lambda t)^k}{k!} e^{-\lambda t} = \\
        = \sum_{k = 0}^\infty (\varphi_V(s))^k \cdot \frac{(\lambda t)^k}{k!} e^{-\lambda t} = e^{\varphi_V(s) \cdot \lambda t} \cdot e^{-\lambda t}
    \end{multline*}
\end{proof}


\begin{corollary}
    \label{corollary:special:compound_Poisson_process_moments}
    Функция среднего и корреляционная функция сложного пуассоновского процесса имеют вид, соответственно,
    \[
        m_Q(t) = \lambda t \cdot \expect V, \qquad
        R_Q(t, s) = \lambda \min\{t,s\} \cdot \expect (V^2)
    \]
\end{corollary}

\begin{proof}
    По свойству характеристической функции,
    \begin{multline*}
        m_Q(t) = \expect Q_t = \left . -i \frac{\partial \varphi_{Q_t}(s)}{\partial s} \right|_{s = 0} =
        \left. -i \frac{\partial}{\partial s} \left( e^{(\varphi_V(s) - 1) \cdot \lambda t} \right) \right|_{s = 0} = \\
        = \lambda t \cdot \underbrace{\left( \left. - i \frac{\partial \varphi_V(s)}{\partial s} \right|_{s = 0} \right)}_{\expect V} \cdot
            \underbrace{e^{(\varphi_V(0) - 1) \cdot \lambda t}}_{e^0} =
        \lambda t \cdot \expect V
    \end{multline*}
    Пользуясь независимостью приращений и полагая $ t \leqslant s $,
    \[
        R_Q(t,s) = \expect Q_t Q_s - m_Q(t) m_Q(s) = \expect \left( Q_t (Q_s - Q_t) \right) + \expect Q_t^2 - \lambda^2 t s \cdot (\expect V)^2
    \]
    \[
        \expect \left( Q_t (Q_s - Q_t) \right) = \expect Q_t \cdot \expect (Q_s - Q_t) = \lambda t \cdot \expect V \cdot \lambda (s - t) \cdot \expect V = \lambda^2 t (s - t) \cdot (\expect V)^2
    \]
    \[
        \expect (Q_t)^2 = \left. (- i)^2 \frac{\partial^2}{\partial s^2} \left( e^{(\varphi_V(s) - 1) \cdot \lambda t} \right) \right|_{s = 0} = (\lambda t)^2 (\expect V)^2 + \lambda t \cdot \expect (V^2)
    \]
    Собирая всё вместе, получаем
    \[
        R_Q(t,s) = \lambda^2 [ \underbrace{t(s - t) + t^2 - ts}_{0} ] \cdot (\expect V)^2 + \lambda t \cdot \expect (V^2) = \lambda t \cdot \expect (V^2)
    \]
    В общем же случае $ R_Q(t, s) = \lambda \min\{t,s\} \cdot \expect (V^2) $.
\end{proof}


\begin{Exercise}[counter=SecExercise, title={(Прореживание пуассоновского процесса)}]
    \noindent
    Пусть $ K_t $~--- пуассоновский процесс с интенсивностью $ \lambda $,
    а случайные величины $ \{V_j\}_{j \in \naturals} $ независимы и имеют распределение Бернулли с параметром $ p $.
    Покажите, что $ Q_t $~--- также пуассоновский процесс с интенсивностью $ p \lambda $.
\end{Exercise}

\begin{Answer}
    \noindent
    Пуассоновский процесс также является сложным пуассоновским процессом с $ V_j \equiv 1 $.
    Тогда характеристическая функция пуассоновского процесса:
    \[
        \varphi_{K_t}(s) = e^{\left(e^{i s} - 1 \right) \cdot \lambda t}
    \]
    Характеристическая функция <<прореженного>> процесса $ Q $:
    \[
        \varphi_{Q_t}(s) = e^{\left(p \, e^{i s} + (1 - p) - 1 \right) \cdot \lambda t} = e^{\left(e^{i s} - 1 \right) \cdot p \lambda t}
    \]
    Что и требовалось доказать.
\end{Answer}
