\section{Важные примеры случайных процессов} \label{section:special}

В этом разделе речь пойдёт о нескольких процессах особого вида,
наиболее часто встречающихся при исследовании реальных явлений.
Зачастую такие процессы именные.
На их примере мы продолжим практиковаться в решении задач,
а также введём несколько новых теоретических понятий.


\subsection{Пуассоновский процесс} \label{subsection:Poisson}

Данный процесс встречается в реальной жизни довольно часто;
он описывает поток случайных событий, которые регистрируются с некоторой постоянной <<интенсивностью>>.
Например, речь может идти о регистрации космических частиц, о кликах по ссылке,
о запросах к серверу, о проезжающих по магистрали автомобилях.
Дадим формальное определение.

\begin{definition}
    \label{definition:special:independent_deltas}
    Случайный процесс $ X $ называется \defemph{процессом с независимыми приращениями},
    если $ \forall n \in \naturals \;\, \forall \{t_i\}_{i=1}^n \subseteq T $ случайные величины
    $ X_{t_n} - X_{t_{n-1}}, \ldots, X_{t_2} - X_{t_1}, X_{t_1} $
    независимы в совокупности.
\end{definition}

\begin{definition}
    \label{definition:special:Poisson_process}
    \defemph{Пуассоновским процессом с интенсивностью $ \lambda > 0 $} называется случайный процесс $ K\colon \Omega \times [0; +\infty) \to \naturals $ такой, что
    \begin{enumerate}
        \item
            $ K_0 \almosteq 0 $.
        \item
            $ K $~--- процесс с независимыми приращениями.
        \item
            $ K_t - K_s \sim \poisson\left( \lambda \cdot (t - s) \right) $ (при $ t > s \geqslant 0 $).
    \end{enumerate}
\end{definition}

Это одно из эквивалентных определений, пуассоновский процесс можно определить и иначе:

\begin{theorem}[Явная конструкция пуассоновского процесса]
    \label{theorem:special:Poisson_process_explicit_definition}
    Пусть $ \xi_1, \ldots, \xi_k, \ldots \sim \expdistr(\lambda) $ и независимы в совокупности,
    $ S_n = \xi_1 + \ldots + \xi_n $.
    Тогда процесс $ X_t = \sup \{ n \mid S_n \leqslant t \} $ есть пуассоновский процесс с интенсивностью $ \lambda $.
\end{theorem}

Процесс $ X_t $, построенный по случайным величинам $ \xi_k $ способом, указанным в теореме,
называется \defemph{процессом восстановления} и отвечает следующей модели:
в нулевой момент включается прибор, который работает время $ \xi_1 $, после чего ломается.
Одновременно с поломкой включается следующий прибор, который работает случайное время $ \xi_2 $, и так далее.
Величина $ X_t $ отражает количество приборов, введённых в эксплуатацию к моменту $ t $.

\begin{statement}
    \label{statement:special:Poisson_process_properties}
    Пуассоновский процесс обладает следующими свойствами:
    \begin{enumerate}
        \item
            Реализации пуассоновского процесса~--- кусочно-постоянные неубывающие функции со значениями в $ \naturals $.
        \item
            С вероятностью $ 1 $ все скачки пуассоновского процесса равны единице.
        \item
            Время, когда произошёл $ n $-ый скачёк (обозначим его $ \tau_n $) имеет $ \Gamma(n, 1/\lambda) $-распределение:
            \[
                \rho_{\tau_n}(t) = \frac{\lambda^n x^{n-1}}{(n-1)!} e^{-\lambda t} \cdot \indicator_{[0;+\infty)}(t)
            \]
        \item
            Случайные величины $ \{\tau_{n} - \tau_{n-1}\}_{n \in \naturals} $ распределены экспоненциально с параметром $ \lambda $ и независимы.
        \item
            Число событий за конечный период времени конечно с вероятностью $ 1 $.
        \item
            Число событий $ K_{t+h} - K_t $ на промежутке $ (t; t+h] $ зависит лишь от длины промежутка $ h $:
            $ \proba \{ K_{t + h} - K_t = k \} = p(h, k) $
        \item
            Вероятность более чем одного скачка на полуинтервале $ (t; t + h] $ есть $ o(h) $,
            то есть $ \displaystyle \lim_{h \to +0} \proba \{ K_{t+h} - K_t > 1 \} = 0 $.
        \item
            Для коротких полуинтервалов $ (t; t+h] $ вероятность того, что на них произойдёт хотя бы один скачок,
            убывает линейно с уменьшением $ h $: $ \proba \{ K_{t+h} - K_t > 0 \} = 1 - e^{-\lambda h} = \lambda h + o(h) $ при $ h \to 0 $.
        \item
            Из определения распределения Пуассона:
            \[
                \proba\{K_t = k\} = \frac{(\lambda t)^k}{k!} e^{-\lambda}
            \]
    \end{enumerate}
\end{statement}

Наконец, приведём ещё одно из альтернативных определений пуассоновского процесса:

\begin{statement}
    \label{statement:special:Poisson_process_alternative_definition}
    Случайный процесс $ K\colon \Omega \times T \to \naturals $ является пуассоновским тогда и только тогда, когда он удовлетворяет следующим свойствам:
    \begin{enumerate}
        \item
            \defemph{(стационарность приращений)}
            $ \proba \{ K_{t + h} - K_t = k \} = p(h, k) $
        \item
            \defemph{(отсутствие последействия)}
            Приращения процесса независимы.
        \item
            \defemph{(ординарность)}
            $ \proba \{ K_{t + h} - K_t > 1 \} \in o(h) $
    \end{enumerate}
\end{statement}
