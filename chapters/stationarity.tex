\section{Стационарность} \label{section:stationarity}

Ранее мы встречались с процессами со \defemph{стационарными приращениями},
то есть с процессами, у которых распределение приращений не зависит от моментов времени,
в которых они взяты, а зависит только от промежутка между сечениями,
разность которых рассматривается в качестве приращения.
Это, например, пуассоновский и винеровский процесс,
где $ K_{t + \Delta t} - K_t \sim \poisson(\lambda \Delta t) $ и $ W_{t + \Delta t} - W_t \sim \normal(0, \Delta t) $, соответственно.

Можно ввести аналогичное определение для всего процесса в целом,
которое будет отражать некоторую инвариантность процесса относительно сдвига по времени.

\begin{definition}
    \label{definition:stationarity:strong_stationarity}
    Случайный процесс $ X $ называется \defemph{стационарным (в узком смысле)},
    если его конечномерные распределения не зависят от одновременного сдвига моментов времени на одно и то же число $ \Delta t $,
    то есть векторы $ (X_{t_1}, \ldots, X_{t_n}) $ и $ (X_{t_1 + \Delta t}, \ldots, X_{t_n + \Delta t}) $
    имеют одинаковое распределение для любых $ n \in \naturals $, $ \{ t_k \}_{k=1}^n \subseteq \{ t \mid (t \in T) \wedge (t + \Delta t \in T) \} $.
\end{definition}

\begin{definition}
    \label{definition:stationarity:weak_stationarity}
    Случайный процесс $ X $ называется \defemph{стационарным в широком смысле},
    если $ m_X(t) = const $, а $ R_X(t, s) $ зависит только от разности $ t - s $.
\end{definition}

\begin{remark}
    \label{remark:stationarity:strong_stationarity_from_weak}
    Из стационарности следует стационарность в широком смысле.
\end{remark}


Для стационарного (в широком смысле) процесса корреляционную функцию
чаще всего пишут в форме $ R_X(\tau) $,
подразумевая под $ \tau $ разность $ t - s $,
поскольку фактически $ R_X(s, t) $ однозначно определяется функцией одной переменной.
В силу симметричности $ R_X(s, t) $ эта новая функция $ R_X(\tau) $ оказывается чётной.

Для стационарного в широком смысле процесса существенно упрощаются критерии непрерывности,
дифференцируемости и интегрируемости в среднеквадратичном.
К примеру, такая непрерывность стационарного процесса равносильна непрерывности $ R_X(\tau) $ в нуле,
а дифференцируемость в среднеквадратичном сразу следует из непрерывности в нуле функции $ R_X(\tau) $.

\begin{statement}
    \label{statement:stationary:stationary_derivative}
    Пусть стационарный в широком процесс $ X $ дифференцируем в среднем квадратичном.
    Тогда $ X' $~--- также стационарный в широком смысле процесс.
\end{statement}

\begin{proof}
    Вспомним, что $ \displaystyle m_{X'}(t) = \frac{d}{d t} m_X(t) $ и $ \displaystyle R_{X'}(t, s) = \frac{\partial^2 R_X(t, s)}{\partial t \, \partial s} $.
    Тогда, в силу стационарности в широком смысле,
    \[
        m_{X'}(t) = \frac{d}{d t} \, const = 0, \qquad
        R_{X'}(t, s) = \frac{\partial^2 R_X(t - s)}{\partial t \, \partial s} = - \left. \frac{d^2 R_X(\tau)}{d \tau^2} \right|_{\tau = t - s}
    \]
    Отсюда видно, что $ m_{X'} $~--- константа,
    а $ R_{X'} $ зависит только от $ t - s $.
\end{proof}


\begin{remark}
    \label{remark:stationarity:gaussian_stationarity}
    Для гауссовских процессов стационарность в широком и узком смыслах эквивалентны.
\end{remark}


\begin{exercise}
    \label{exercise:stationarity:wiener_process}
    Показать, что винеровский процесс $ W $ не стационарен ни в каком смысле,
    а процесс $ Y_t = W_{t + \Delta t} - W_t $ ($ t, \Delta t \geqslant 0 $) стационарен в обоих смыслах.
\end{exercise}

\begin{solution}
    Дисперсия винеровского процесса зависит от времени,
    поэтому $ W $ сам по себе не стационарен.
    Рассмотрим теперь $ Y $:
    \[
        m_Y(t) = \expect W_{t + \Delta t} - \expect W_t = 0 - 0 = 0
    \]
    \begin{multline*}
        R_Y(t, s) = \covariance{W_{t + \Delta t} - W_t}{W_{s + \Delta t} - W_s} = \\ =
        \min \{t + \Delta t, s + \Delta t\} - \min \{t + \Delta t, s\} - \min \{t, s + \Delta t\} + \min \{t, s\}
    \end{multline*}
    Используя $ 2 \min \{a, b\} = a + b - |a - b| $, получаем
    \[
        R_Y(t, s) = - |t - s| + \frac{1}{2} \left( |t - s + \Delta t| + |t - s - \Delta t| \right) = f(t - s)
    \]
    Согласно замечанию \ref{remark:stationarity:gaussian_stationarity},
    имеем стационарность как в широком, так и в узком смыслах.
\end{solution}


\begin{exercise}
    \label{exercise:stationarity:cosine}
    Дан случайный процесс $ Z_t = A \cos (B t + \varphi) $ ($ t \geqslant 0 $),
    где $ A $, $ B $ и $ \varphi $~--- случайные величины,
    $ \varphi \sim U_{[0; 2\pi]} $ и не зависит от $ (A, B) $.
    Исследовать процесс $ Z $ на стационарность в обоих смыслах.
\end{exercise}

\begin{solution}
    Зафиксируем $ (A, B) = (a, b) $, $ \Delta t > 0 $.
    Так как $ \varphi $ не зависит от $ (A, B) $,
    распределение данной случайной величины осталось тем же (равномерным на отрезке $ [0; 2\pi] $).
    Обозначим $ \varphi' = \varphi + B \Delta t \mod 2 \pi $.
    Распределение $ \varphi' $~--- равномерное на отрезке $ [0; 2\pi] $ независимо от $ a $, $ b $ и $ \Delta t $.
    Значит, $ \varphi' \sim U_{[0;2\pi]} $ также не зависит от $ (A,B) $.
    В таком случае, если ввести $ Y_t = Z_{t + \Delta t} = A \cos (B t + \varphi') $,
    то вектор $ (Z_{t_1 + \Delta t}, \ldots, Z_{t_n + \Delta t}) $ равен вектору $ (Y_{t_1}, \ldots, Y_{t_n}) $,
    который имеет то же распределение, что и $ (Z_{t_1}, \ldots, Z_{t_n}) $,
    так как $ (A, B, \varphi) $ и $ (A, B, \varphi') $ распределены одинаково.
    Отсюда следует, что $ Z $ стационарен по определению.
\end{solution}
