\section{Стационарность} \label{section:stationarity}

Ранее мы встречались с процессами со \defemph{стационарными приращениями},
то есть с процессами, у которых распределение приращений не зависит от моментов времени,
в которых они взяты, а зависит только от промежутка между сечениями,
разность которых рассматривается в качестве приращения.
Это, например, пуассоновский и винеровский процесс,
где $ K_{t + \Delta t} - K_t \sim \poisson(\lambda \Delta t) $ и $ W_{t + \Delta t} - W_t \sim \normal(0, \Delta t) $, соответственно.

Можно ввести аналогичное определение для всего процесса в целом,
которое будет отражать некоторую инвариантность процесса относительно сдвига по времени.

\begin{definition}
    \label{definition:stationarity:strong_stationarity}
    Случайный процесс $ X $ называется \defemph{стационарным (в узком смысле)},
    если его конечномерные распределения не зависят от одновременного сдвига моментов времени на одно и то же число $ \Delta t $,
    то есть векторы $ (X_{t_1}, \ldots, X_{t_n}) $ и $ (X_{t_1 + \Delta t}, \ldots, X_{t_n + \Delta t}) $
    имеют одинаковое распределение для любых $ n \in \naturals $, $ \{ t_k \}_{k=1}^n \subseteq \{ t \mid (t \in T) \wedge (t + \Delta t \in T) \} $.
\end{definition}

\begin{definition}
    \label{definition:stationarity:weak_stationarity}
    Случайный процесс $ X $ называется \defemph{стационарным в широком смысле},
    если $ m_X(t) = const $, а $ R_X(t, s) $ зависит только от разности $ t - s $.
\end{definition}

\begin{remark}
    \label{remark:stationarity:strong_stationarity_from_weak}
    Из стационарности следует стационарность в широком смысле.
\end{remark}


Для стационарного (в широком смысле) процесса корреляционную функцию
чаще всего пишут в форме $ R_X(\tau) $,
подразумевая под $ \tau $ разность $ t - s $,
поскольку фактически $ R_X(s, t) $ однозначно определяется функцией одной переменной.
В силу симметричности $ R_X(s, t) $ эта новая функция $ R_X(\tau) $ оказывается чётной.

Для стационарного в широком смысле процесса существенно упрощаются критерии непрерывности,
дифференцируемости и интегрируемости в среднеквадратичном.
К примеру, такая непрерывность стационарного процесса равносильна непрерывности $ R_X(\tau) $ в нуле,
а дифференцируемость в среднеквадратичном сразу следует из непрерывности в нуле функции $ R_X(\tau) $.

\begin{statement}
    \label{statement:stationary:stationary_derivative}
    Пусть стационарный в широком процесс $ X $ дифференцируем в среднем квадратичном.
    Тогда $ X' $~--- также стационарный в широком смысле процесс.
\end{statement}

\begin{proof}
    Вспомним, что $ \displaystyle m_{X'}(t) = \frac{d}{d t} m_X(t) $ и $ \displaystyle R_{X'}(t, s) = \frac{\partial^2 R_X(t, s)}{\partial t \, \partial s} $.
    Тогда, в силу стационарности в широком смысле,
    \[
        m_{X'}(t) = \frac{d}{d t} \, const = 0, \qquad
        R_{X'}(t, s) = \frac{\partial^2 R_X(t - s)}{\partial t \, \partial s} = - \left. \frac{d^2 R_X(\tau)}{d \tau^2} \right|_{\tau = t - s}
    \]
    Отсюда видно, что $ m_{X'} $~--- константа,
    а $ R_{X'} $ зависит только от $ t - s $.
\end{proof}


\begin{remark}
    \label{remark:stationarity:gaussian_stationarity}
    Для гауссовских процессов стационарность в широком и узком смыслах эквивалентны.
\end{remark}


\begin{exercise}
    \label{exercise:stationarity:wiener_process}
    Показать, что винеровский процесс $ W $ не стационарен ни в каком смысле,
    а процесс $ Y_t = W_{t + \Delta t} - W_t $ ($ t, \Delta t \geqslant 0 $) стационарен в обоих смыслах.
\end{exercise}

\begin{solution}
    Дисперсия винеровского процесса зависит от времени,
    поэтому $ W $ сам по себе не стационарен.
    Рассмотрим теперь $ Y $:
    \[
        m_Y(t) = \expect W_{t + \Delta t} - \expect W_t = 0 - 0 = 0
    \]
    \begin{multline*}
        R_Y(t, s) = \covariance{W_{t + \Delta t} - W_t}{W_{s + \Delta t} - W_s} = \\ =
        \min \{t + \Delta t, s + \Delta t\} - \min \{t + \Delta t, s\} - \min \{t, s + \Delta t\} + \min \{t, s\}
    \end{multline*}
    Используя $ 2 \min \{a, b\} = a + b - |a - b| $, получаем
    \[
        R_Y(t, s) = - |t - s| + \frac{1}{2} \left( |t - s + \Delta t| + |t - s - \Delta t| \right) = f(t - s)
    \]
    Согласно замечанию \ref{remark:stationarity:gaussian_stationarity},
    имеем стационарность как в широком, так и в узком смыслах.
\end{solution}


\begin{exercise}
    \label{exercise:stationarity:cosine}
    Дан случайный процесс $ Z_t = A \cos (B t + \varphi) $ ($ t \geqslant 0 $),
    где $ A $, $ B $ и $ \varphi $~--- случайные величины,
    $ \varphi \sim U_{[0; 2\pi]} $ и не зависит от $ (A, B) $.
    Исследовать процесс $ Z $ на стационарность в обоих смыслах.
\end{exercise}

\begin{solution}
    Зафиксируем $ (A, B) = (a, b) $, $ \Delta t > 0 $.
    Так как $ \varphi $ не зависит от $ (A, B) $,
    распределение данной случайной величины осталось тем же (равномерным на отрезке $ [0; 2\pi] $).
    Обозначим $ \varphi' = \varphi + B \Delta t \mod 2 \pi $.
    Распределение $ \varphi' $~--- равномерное на отрезке $ [0; 2\pi] $ независимо от $ a $, $ b $ и $ \Delta t $.
    Значит, $ \varphi' \sim U_{[0;2\pi]} $ также не зависит от $ (A,B) $.
    В таком случае, если ввести $ Y_t = Z_{t + \Delta t} = A \cos (B t + \varphi') $,
    то вектор $ (Z_{t_1 + \Delta t}, \ldots, Z_{t_n + \Delta t}) $ равен вектору $ (Y_{t_1}, \ldots, Y_{t_n}) $,
    который имеет то же распределение, что и $ (Z_{t_1}, \ldots, Z_{t_n}) $,
    так как $ (A, B, \varphi) $ и $ (A, B, \varphi') $ распределены одинаково.
    Отсюда следует, что $ Z $ стационарен по определению.
\end{solution}


\subsection{Спектральная теория стационарности} \label{subsection:stationarity:spectral_theory}

В рамках данного раздела мы разберём теорию,
полезную для исследования стационарных случайных процессов.
Для большей общности мы будем рассматривать комплекснозначные случайные процессы
(то есть функции из $ \Omega \times T $ в $ \complexes $).
Практически все определения для обычных случайных процессов переносятся на комплекснозначные без изменений.
Необходимо, однако, внести небольшую корректировку в определение ковариацонной и корреляционной функции:

\begin{definition}
    \label{definition:stationarity:second_order_moment_functions_compex}
    Если $ \forall t_1, t_2 \in T $ существует и конечно $ \expect X_{t_1} X_{t_2}^* $,
    то функции $ K_X(t_1, t_2) = \expect X_{t_1} X_{t_2}^* $ и $ R_X(t_1, t_2) = \expect \rvcenter X_{t_1} \rvcenter X_{t_2}^* $
    определены и называются, соответственно, \defemph{ковариационной} и \defemph{корреляционной функциями}
    комплекснозначного случайного процесса $ X $.
\end{definition}

С таким определением утверждение \ref{statement:special:correlation_function_is_semi_definite} остаётся в силе,
только симметричность заменяется на эрмитовость: $ K_X(t,s) = K_X^*(s,t) $, $ R_X(t,s) = R_X^*(s,t) $.
Если процесс также является стационарным,
то по аналогии с вещественным случаем можно рассматривать $ R_X(t, s) $ как функцию одного аргумента: $ R_X(t - s) = R_X(\tau) $.

Свойства корреляционной функции стационарного процесса схожи
со свойствами характеристической функции некоторой случайной величины.
Впрочем, последняя всегда равномерно непрерывна,
а корреляционная функция не обязана даже быть непрерывной.

\begin{theorem}[Бохнер-Х\'{и}нчин]
    \label{theorem:stationarity:stationary_process_correlation_function_spectral_representation}
    Непрерывная функция $ R_X(\tau) $ является корреляционной
    для некоторого стационарного и непрерывного в среднем квадратичном процесса тогда и только тогда,
    когда она представляется в виде интеграла Лебега-Стилтьеса
    \begin{equation}
        \label{equation:stationarity:stationary_process_correlation_function_spectral_representation}
        R_X(\tau) = \int_\reals e^{i \lambda \tau} \, dS_X(\lambda),
    \end{equation}
    где функция $ S_X(\lambda) $ неотрицательная, монотонно неубывающая, ограниченная и непрерывная слева,
    то есть равная $ R_X(0) \cdot F(\lambda) $,
    где $ F $~--- функция распределения \uline{некоторой} случайной величины.
\end{theorem}

\begin{definition}
    \label{definition:stationarity:spectral_function}
    Функция $ S_X(\lambda) $ в \eqref{equation:stationarity:stationary_process_correlation_function_spectral_representation}
    теоремы \ref{theorem:stationarity:stationary_process_correlation_function_spectral_representation}
    называется \defemph{спектральной функцией стационарного процесса $ X $}.
    %\\[0.25\baselineskip]
    Если $ S_X(\lambda) $ абсолютно непрерывна, то есть
    \[
        \exists s_X(a): \quad S_X(\lambda) = \int\limits_{-\infty}^\lambda s_X(a) \, da,
    \]
    то функция $ s_X(\lambda) $ называется \defemph{спектральной плотностью}.
\end{definition}

\begin{remark}
    \label{remark:stationarity:spectral_density_Fourier}
    В случае, когда есть спектральная плотность, корреляционная функция есть её преобразование Фурье:
    \[
        R_X(\tau) = \int_\reals e^{i \lambda \tau} s(\lambda) \, d \lambda
    \]
\end{remark}

\begin{corollary}
    \label{corollary:stationarity:L1_Fourier}
    Если $ R_X(\tau) \in \lebesgue_1(\reals) $,
    то спектральная функция $ S_X $ обладает спектральной плотностью
    \[
        s_X(\lambda) = \frac{1}{2 \pi} \int_\reals e^{-i \lambda \tau} R_X(\tau) \, d\tau
    \]
\end{corollary}

\begin{proof}
    Это прямое следствие из \ref{remark:stationarity:spectral_density_Fourier} и теоремы о существовании преобразования Фурье для $ \lebesgue_1 $-функций.
\end{proof}


\begin{exercise}
    \label{exercise:stationarity:cosine_positive_semi_definite}
    Является ли $ R(\tau) = \cos(\tau) $ неотрицательно определённой функцией?
\end{exercise}

\begin{solution}
    Эрмитовость функции $ R(\tau) $ очевидна.
    Проверять неотрицательную определённость напрямую неудобно.
    Воспользуемся теоремой Бохнера-Хинчина:
    $ R $ непрерывна, поэтому если она является корреляционной функцией,
    то она имеет вид \eqref{equation:stationarity:stationary_process_correlation_function_spectral_representation}.
    Достаточно подобрать подходящую $ S(\lambda) $.
    Несложно заметить, что подходит ступенчатая функция с двумя скачками $ 1/2 $ в точках $ \pm 1 $.
    Тогда интеграл \eqref{equation:stationarity:stationary_process_correlation_function_spectral_representation}
    будет равен $ (e^{i \tau} + e^{-i \tau}) / 2 $, то есть $ \cos(\tau) $.
    Значит, $ R $ является корреляционной функцией некоторого стационарного случайного процесса,
    а потому неотрицательно определена.
\end{solution}

В чём смысл спектральной функции/спектральной плотности?
Стационарный процесс можно представить в виде совокупности множества нескоррелированных стационарных процессов
с гармоническими корреляционными функциями ($ e^{i \lambda \tau} $ или просто $ \cos(\lambda \tau) $).
Спектральная плотность обозначает <<вес>> (в смысле $ \lebesgue_2 $) той или иной <<гармоники>> в общем <<сигнале>>.
Если формализовать это интуитивное понимание, можно прийти к следующей теореме:

\begin{theorem}[Крамер]
    \label{theorem:stationarity:stochastic_spectral_decomposition}
    Любому стационарному в широком смысле процессу $ X $ соответствует процесс
    с ортогональными приращениями $ V_X(\lambda) $ такой, что с вероятностью $ 1 $ выполнено
    \[
        X_t = m_X + \int_\reals e^{i \lambda t} \, dV_X(\lambda),
    \]
    где $ m_X = \expect X_t = const $.
    Процесс $ V_X $ определён по $ X $ однозначно с точностью до аддитивной случайной величины.
\end{theorem}

Оказывается, что спектральная функция $ S_X(\lambda) $ в
\eqref{equation:stationarity:stationary_process_correlation_function_spectral_representation}
связана с процессом $ V_X(\lambda) $.
Выражение $ \nu([a, b]) = \expect |V_X(b) - V_X(a)|^2 $ можно рассматривать как меру.
Оказывается, это та же самая мера, по которой ведётся интегрирование в интеграле Стилтьеса (по $ dS_X(\lambda) $).
Это обстоятельство кратко записывают в виде $ \expect|dV_X(\lambda)|^2 = dS_X(\lambda) $.

\begin{example}
    \label{example:stationarity:Wiener_process_as_V}
    Если в качестве $ V_X $ взять винеровский процесс $ W $, то $ \nu([a, b]) = b - a $.
\end{example}

Приведём полезные теоремы о внесении функций и производных под знак интеграла
по случайному процессу с ортогональными приращениями:

\begin{theorem}
    \label{theorem:stationarity:spectral_functions_relation}
    Пусть $ X $~--- стационарный процесс со спектральной функцией $ S_X(\lambda) $ и спектральным разложением
    \[
        X_t = \int_\reals e^{i \lambda t} \, dV_X(\lambda)
    \]
    Пусть $ \Phi(\lambda) $~--- неслучайная комплексная функция такая,
    что $ \int_\reals |\Phi(\lambda)|^2 \, dS_X(\lambda) < +\infty $.
    Пусть, наконец, процесс $ Y $ раскладывается в интеграл по тому же процессу $ V_X $ следующим образом:
    \[
        Y_t = \int_\reals e^{i \lambda t} \Phi(\lambda) \, dV_X(\lambda)
    \]
    Тогда спектральная функция $ S_Y $ связана c $ S_X $ соотношением $ dS_Y(\lambda) = |\Phi(\lambda)|^2 \cdot dS_X(\lambda) $
    (в случае спектральных плотностей имеем $ s_Y(\lambda) = |\Phi(\lambda)|^2 \cdot s_X(\lambda) $).
\end{theorem}

\begin{theorem}
    \label{theorem:stationarity:spectral_differentiation}
    Если для некоторого $ k \geqslant 0 $ стационарный процесс $ X $ является $ k $ раз дифференцируемым в смысле среднего квадратичного,
    и существует конечный интеграл $ \int_\reals \lambda^{2 k} \, dS_X(\lambda) $, то
    \[
        X_t^{(k)} = \int_\reals \left( \frac{\partial^k}{\partial t^k} e^{i \lambda t} \right) \, dV_X(\lambda) = \int_\reals e^{i \lambda t} (i \lambda)^k \, dV_X(\lambda)
    \]
\end{theorem}

\begin{exercise}
    \label{exercise:stationarity:linear_system}
    Дана линейная система $ a_1 X_t' + a_0 X_t = Y_t $.
    В предположении о существовании стационарного решения уравнения
    вычислить спектральную плотность $ s_X(\lambda) $,
    если известна спектральная плотность $ s_Y(\lambda) $, а также что $ \expect Y_t = 0 $.
\end{exercise}

\begin{solution}
    Предполагая выполнение условий теорем \ref{theorem:stationarity:spectral_functions_relation} и \ref{theorem:stationarity:spectral_differentiation},
    \[
        X_t = \int_\reals e^{i \lambda t} \, dV_X(\lambda),
        \quad
        Y_t = \int_\reals e^{i \lambda t} \, dV_Y(\lambda)
        \quad \Longrightarrow \quad
        \int_\reals e^{i \lambda t} (a_1 i \lambda + a_0) \, dV_X(\lambda) = \int_\reals e^{i \lambda t} \, dV_Y(\lambda)
    \]
    Отсюда получаем, что $ s_Y(\lambda) = |a_1 i \lambda + a_0|^2 \cdot s_X(\lambda) $,
    то есть $ s_X(\lambda) = \frac{s_Y(\lambda)}{a_1^2 \lambda^2 + a_0^2} $.
\end{solution}
