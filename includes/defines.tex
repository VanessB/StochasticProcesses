%% Математические символы и прочие дефайны.

\def\defarr{\overset{\triangle}{\Longleftrightarrow}} % <<По определению>>
\def\defeq{\overset{\triangle}{=}}                    % <<По определению равно>>
\def\symdiff{\,\triangle\,}                           % <<Симметрическая разность>>
\def\connected{\leftrightsquigarrow}                  % Связность в графах.

% Большая черта в множествах.
\def\Mid{\;\middle|\;}

% Стрелки для сходимостей.
\newcommand{\limarrow}[2][\longrightarrow]{\underset{#2}{#1}}


% Математическое операторы.
\DeclareMathOperator{\diam}{\textnormal{diam}}
\DeclareMathOperator{\rad}{\textnormal{rad}}
\DeclareMathOperator*{\argmin}{\arg\min}
\DeclareMathOperator*{\argmax}{\arg\max}
\DeclareMathOperator*{\dom}{\textnormal{dom}}
\DeclareMathOperator*{\range}{\textnormal{range}}
\DeclareMathOperator*{\closure}{\textnormal{cl}}
\DeclareMathOperator*{\lowlim}{\underline{lim}}
\DeclareMathOperator*{\uplim}{\overline{lim}}

% Математические множества.
\def\naturals{\mathbb{N}}
\def\integers{\mathbb{Z}}
\def\rationals{\mathbb{Q}}
\def\reals{\mathbb{R}}
\def\complexes{\mathbb{C}}

\def\partition{\mathcal{T}} % Разбиение.

% Функции.
\def\blankarg{\, \cdot \,}

% Линейные пространства.
\newcommand{\dotprod}[2]{\left(#1, #2\right)}
